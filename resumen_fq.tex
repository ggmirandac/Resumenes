% Template:     Informe LaTeX
% Documento:    Archivo principal
% Versión:      8.1.0 (19/03/2022)
% Codificación: UTF-8
%
% Autor: Pablo Pizarro R.
%        pablo@ppizarror.com
%
% Manual template: [https://latex.ppizarror.com/informe]
% Licencia MIT:    [https://opensource.org/licenses/MIT]
% CREACIÓN DEL DOCUMENTO
\documentclass[
	spanish, % Idioma: spanish, english, etc.
	letterpaper, oneside
]{article}

% INFORMACIÓN DEL DOCUMENTO
\def\documenttitle {Resumen Fisicoquímica}
\def\documentsubtitle {}
\def\documentsubject {Físicoquímica}

\def\documentauthor {Gabriel Miranda}
\def\coursename {Fisicoquímica}
\def\coursecode {IIQ2043}

\def\universityname {Pontificia Universidad Católica de Chile}
\def\universityfaculty {Facultad de Ingeniería Química y Bioprocesos}
\def\universitydepartment {Escuela de Ingeniería}
\def\universitydepartmentimage {departamentos/LogoPUCING}
\def\universitydepartmentimagecfg {height=1.57cm}
\def\universitylocation {Santiago de Chile}

% INTEGRANTES, PROFESORES Y FECHAS
\def\authortable {
\begin{tabular}{ll}
	Por:
	& \begin{tabular}[t]{l}
		Gabriel Miranda
	\end{tabular} \\
	Profesor:
	& \begin{tabular}[t]{l}
		Roberto Canales
	\end{tabular} \\
	
	\multicolumn{2}{l}{Fecha de realización: \today} \\
	\multicolumn{2}{l}{Fecha de entrega: \today} \\
	\multicolumn{2}{l}{\universitylocation}
\end{tabular}
}

% IMPORTACIÓN DEL TEMPLATE
\input{template}

% INICIO DE PÁGINAS
\begin{document}
	
% PORTADA
\templatePortrait

% CONFIGURACIÓN DE PÁGINA Y ENCABEZADOS
\templatePagecfg

% RESUMEN O ABSTRACT
\begin{abstractd}
	En resumen, se describe la físicoquímica de la materia.
	% Párrafo ejemplo, se puede borrar
\end{abstractd}
\clearpage
% TABLA DE CONTENIDOS - ÍNDICE
\tableofcontents
% CONFIGURACIONES FINALES
\templateFinalcfg

% ======================= INICIO DEL DOCUMENTO =======================

\section{Repaso termodinámica}
\subsection{Conceptos termodinámicos fundamentales}
Hay ciertos conceptos que deben conocerse, y son fundamentales a la hora de estudiar fisicoquímica, estos son:
\begin{itemize}
    \item \textbf{Temperatura}: Esta caracteriza la transferencia de energía térmica, o calor, entre un sistema y otro. Es una medida de la energía cinética asociada a las colisiones de las partículas que componen el sistema.
    \item \textbf{Presión}: Acumulación de fuerzas de colisión en le área total de las parades del recipiente.
    \item \textbf{Calor y Trabajo}: Es la energía en tránsito. La fuerza motriz del flujo de calor es la diferencia de temperatura, mientras que la fuerza motriz del trabajo es el movimiento. Ambas son funciones de trayectoria, es decir, dependen del camino, no de los estados termodinámicos.
    \item \textbf{Energía Interna, U}: Energía total de todos los componentes de un sistema. Es la sumatoria de las energías de traslación, rotación, vibración, electrónica, nuclear y energía de interacción molecular.
    \item \textbf{Entalpía, H}: Cantidad termodinámica que se utiliza para describir cambios de calor que se efectúan a presión constante (esta definición es para un sistema cerrado).
    \item \textbf{Entropía, S}: Cantidad termodinámica que expresa el grado de desorden o de aleatoriedad de un sistema.
\end{itemize}

\subsection{Leyes de la termodinámica}

\textbf{Ley Cero:} Establece que, cuando dos cuerpos están en equilibrio térmico con un tercero, estos están a su vez en equilibrio térmico.

\insertimage[]{img/imagenes/ley0}{width=5cm}{Diagrama de la ley cero de la termodinámica.}

\textbf{Primera Ley:} La energía interna de un sistema no se crea ni se destruye, sólo se transforma. Esta ley puede ser representada por medio de la siguiente ecuación para un sistema cerrado:

\insertequation{dU=\delta Q+\delta W}{}

\textbf{Segunda Ley:} Si bien tdo el trabajo mecánico puede transformarse en calor, no todo el calor puede transformarse en trabajo mecánico. Esta ley restringe cuales procesos son posibles o no.

\textbf{Tercera Ley:} No se puede alcanzar el cero absoluto en un número finito de etapas.

\subsection{Ecuaciones de estado}

Una ecuación de estado es una ecuación constitutiva que describe el estado de agregación de la materia como una relación matemática entre la temperatura, la presión, el volumen, la densidad, la energía interna y posiblemente otras funciones de estado\footnote{Una función de estado es una cantidad que no depende del camino, solo del estado inicial y final.} asociadas con la materia. 
Estas ecuaciones nos permiten relacionar las variables del sistema, por lo general, utilizando valores medibles, como son la temperatura, la presión ,y el volumen.
Algunas ecuaciones de estado son:
\begin{itemize}
    \item \textbf{Ecuación del gas ideal:} En esta ecuacion se asume que es un gas ideal, es decir, en condiciones de baja presión, alto volumen y alta temperatura. Cuando se asume esta idealidad, los gases pueden ser descritos según la siguiente ecuación de estado:
    \insertequation{PV=RT}

    Donde P es la presión (Pa), V es el volumen molar ($\frac{m^3}{mol}$),
      es la constante de gas ideal  $(8.314\;\frac{J}{mol*K})$, y T es la temperatura (K).

    \item \textbf{Factor de compresibilidad generalizado:} Esta ecuación se utiliza para, además de predecir ciertas funciones de estado, también nos permite saber que tan ideal es un fluido. Esta ecuación es:
    \insertequation{PV=RTZ}

    Donde Z es el factor de compresibilidad, y a medida que Z sea más cercano a 1, más ideal es su comportamiento. Si $Z\noteq 1$ implica que es un fluido real.
    \item \textbf{Ecuación tipo Clausius:} Esta ecuación aproxima las funciones de estado por medio de:
    \insertequation{P(V-b)=RT}{}

    \item \textbf{Ecuación de Van der Waals:} Esta ecuación es una de las primeras que pude describir un fluido real, esta presenta una serie de variables que permiten que se pueda tener un mayor acercamiento al comportamiento de un fluido real:
    
    \insertalign{
        P&= \frac{RT}{V-b}-\frac{a}{V^2}\label{eqn:vdw}\\
        a &= \frac{27R^2 T_c^2}{64P_c}\\
        b&= \frac{RT_c}{8 P_c}
    }

    En esta ecuación $P_c$ y $T_c$ son la presión y temperatura crítica. Cabe mencionar que esta ecuación, cuando analizamos el volumen nos da 3 raices; cuando tenemos una raiz real y dos imaginarias, la real es la que describe al fluido; cuando tenemos 3 raices reales, la menor describe al líquido, la mayor al vapor y la del medio no tiene significado físico. Por lo cual hay que tener presente la fase del fluido al realizar la ecuación cúbica.

    Esta ecuación permite describir tanto gases como líquidos.
    
    \item \textbf{Ecuación de Soave-Redlich-Kwong:} Esta ecuación es la siguiente:
    
    \insertalign{
        P&=\frac{RT}{V-b}-\frac{a}{\sqrt{T}V(V+b)}\label{eqn:srk}\\
        a&=0.42748\frac{R^2 T_c^{2.5}}{P_c}\\
        b&=0.08664\frac{RT_c}{P_c}
    }
    
    Esta ecuación sigue los mismos principios que la ecuacion de Van der Waals.
    Esta ecuación permite describir tanto gases como líquidos.
    
    \item \textbf{Ecuación de Peng-Robinson:} Esta ecuación es la siguiente:
    
    \insertalign{
        P&= \frac{RT}{V-b}-\frac{a}{V^2+2bV-b^2}\label{eqn:pr}\\
        a&=0.45723\frac{R^2 T_c^{2}}{P_c} [1+(0.37464+1.54226\omega-0.26992\omega^2)(1-T_r^{0.5})]^2\\
        b&=0.07780\frac{RT_c}{P_c}\\
        T_r&=\frac{T}{T_c}
    }
    
    En esta ecuación se introducen las propiedades reducidas por medio del $T_r$ y el factor acéntrico, el cual nos permite cuantificar la esfericidad de las partículas del fluido.
    Esta ecuació permite describir tanto gases como líquidos.

    \item \textbf{Ecuación Virial:} Esta es una ecuación que, a diferencia de las anteriores, solo sirve para gases, y esta es:

    \insertalign{
        Z&= 1+\frac{B}{V}+\frac{C}{V^2} \label{eqn:virial}\\
        \frac{BP_c}{RT_c} &= B^0 + \omega B^1\\
        B^0&= 0.1445 - \frac{0.3300}{T_r} - \frac{0.1385}{T_r^2} - \frac{0.0121}{T_r^3} -\frac{0.000607}{T_r^8} \\
        B^1&= 0.0637 + \frac{0.331}{T_r^2}-\frac{0.423}{T_r^3}-\frac{0.0008}{T_r^8} \\
    }
\end{itemize}

\subsection{Grados de libertad termodinámicos}

Los grados de libertad de un problema termodinámico se define seín la \textbf{Regla de las fases de Gibs}:

\insertequation{GL=2-\phi +N}

Donde $\phi$ es el número de fases (líquido, gas o sólido) y $N$ es el número de especies químicas.

\subsection{Superficie P-V-T}

Al graficar de forma tridimensional las relaciones entre la presión, el volumen y la temperatura, podemos encontrar una superficie que nos describe el comportamiento de los fluidos en función de estas variables.
El gráfico general es:

\insertimage{img/imagenes/diagramapvt}{scale=0.5}{Diagrama tridimensional de las relaciones entre la presión, temperatura y volumen.}

Se pueden realizar análisis bidimensionales para poder entender mejor como se comportan los fluidos. En estos gráficos bidimensionales podemos encontrar el diagrama P-V:

\insertimage[]{img/imagenes/diagramapv}{width=10cm}{Diagrama que gráfica la relación entre la presión y el volumen.}

En este gráfico podemos encontrar zonas de líquido subenfriado, vapor sobrecalentado, fluido supercrítico y la campana líquido-vapor. Siendo la última la zona en la cual la muestra se encuentra en un equilibrio líquido vapor. Podemos encontrar también las isotermas, que son aquellas líneas que cruzan el gráfico. Y también podemos encontrar el punto crítico. Es en dicho punto donde se cumple que:

\insertalign{
    \left(\frac{\partial P}{\partial V}\right)_T=0\\
    \left(\frac{\partial^2 P}{\partial V^2}\right)_T=0
}

Estas derivadas parciales de arriba significan, la derivada parcial de la presión con respecto al volumen, a temperatura constante.

Dentro de la campana líquido-vapor se cumple la relga de la palaca, ésta nos permite analizar cuale son las fracciones molares de la fase vapor y la fase líquida:
\insertimage{img/imagenes/palanca}{width=10cm}{Diagrama representativo de la regla de la palanca.}

Esta regla se puede traducir a las siguientes ecuaciones:

\insertalign{
    V&= x_L V_L + x_V V_V\\
    x_V&= \frac{V-V_L}{V_V-V_L}\\
    x_L&= \frac{V-V_V}{V_L-V_V}
}

Donde $V$ es el volumen de la muestra, $V_V$ el volumen del vapor saturado, $V_L$ el volumen del líquido saturado, $x_L$ la fracción molar del líquido y $x_V$ la fracción molar del vapor.

A su vez, también tenemos el diagrama P-T:

\insertimage{img/imagenes/diagramapt}{width=10cm}{Diagrama de presión-temperatura.}

En este diagrama podemos encontrar el punto crítico en C. El punto triple, que es donde las tres fases (sólido, líquido y gaseos) están en equilibrio, y cada punto en la linea que va desde F a C es un punto en donde se entra a al campana líquido-vapor que se ve representada en el diagrama P-V.

\subsection{Balance de energía en un sistema abierto}

El valance de energía en un sistema abierto puede ser descrito por la siguiente ecuación:

\insertimage{img/imagenes/balanceE}{width=6cm}{Diagrama de balance de energía en un sistema abierto.}

\insertequation{\frac{dU}{dt}=\sum_{i=1}^N \dot{m}_i U_i + \dot{W}_s + \dot{Q} - P\frac{dV}{dt}+\sum_{i=1}^N \dot{m}_i (PV)_i}

El significado de cada uno de los parámetros es el siguiente:

\begin{itemize}
    \item Esta es la acumulación de enercía interna, este valor es diferente de cero en un sistema cerrado y en un sistema abierto con acumulación. Si no hay acumulación este término es 0. 
        \insertequation{\frac{dU}{dt}}
    \item  Esta expresión es el trabajo de eje, es el trabajo que realiza una máquina o turbina.
    \insertequation{\dot{W}_s}
    \item Calor que es entregado o retirado del sistema.
    
    \insertequation{\dot{Q}}

    \item Trabajo de compresión o expanción, debe de haber una diferencia de volumen.
    
    \insertequation{-P\frac{dV}{dt}}

    \item Energía interna que trae el flujo que entra, o se lleva el flujo que sale.
    
    \insertequation{ \sum_{i=1}^N \dot{m}_i U_i}

    \item Es la energía que se relaciona con la entrada o salida del fluido del sistema.
    
    \insertequation{\sum_{i=1}^N \dot{m}_i (PV)_i}
    
    \item Podemos juntar estas dos expresiones de forma de generar el término de la entalpía, dado que $U+PV=H$. De modo que este término es la entalpía de los flujos de salida y entrada.
    
    \insertalign{
        \sum_{i=1}^N \dot{m}_i U_i + \sum_{i=1}^N \dot{m}_i (PV)_i=\sum_{i=1}^N \dot{m}_i H_i
    }
\end{itemize}

Lo interesante de estos balances de energía es que dependiendo del sistema que se tiene, su balance de energía va a cambiar según cuantas de las expresiones de arriba se anulan. Estas se anulan en las siguientes condiciones:

\begin{itemize}
    \item Si el sistema no presenta acumulación. Cuando es un sistema abierto, la carencia de acumulación se denomina estado estacionario.
    \insertequation{\frac{dU}{dt}=0}

    \item Si el sistema es cerrado:
    \insertequation{\dot{m}_i=0}{}
    
    Lo que implica que:
    \insertalign{\sum_{i=1}^N \dot{m}_i (PV)_i&=0\\
    \sum_{i=1}^N \dot{m}_i U_i&=0
    }
    \item Si el sistema es adiabático:
    
    \insertequation{\dot{Q}=0}

    \item Si no hay trabajo de compresnsión o expansión:
    
    \insertequation{-P\frac{dV}{dt}}{}

    \item Si el sistema es rígido:
    
    \insertalign{
        dV&=0\\ 
        -P\frac{dV}{dt}&=0
    }

\end{itemize}

\clearpage
\section{Termodinámica clásica}

\subsection{Introducción}

¿Qué es la termodinámica clásica? La termodinámica clásica se refiere a el subtópico de la fisicoquímica que trabaja con las relaciones matemática
de energía y Entropía en los fluidos\footnote{Es imporante la utilización de ecuaciones de estado en este cálculo.}. A través de ella se pueden generalizar el conocimiento acerca de cualquier fluído en cualquier estado termodinámico.

La estimación de propiedades fisicoquímicas puede ser tanto para una fase como para procesos de cambios de fase, por ejemplo, la formación de condensado durante la expansión de un vapor en una turbina. Desarrollas estas habilidades ayudará a analizar de mejor manera la termodinámica no ideal y de mezclas.

\subsection{Fundamentos de las relaciones termodinámicas}

Se puede comenzar el análissi en un sistema simple cerrado. Con una sustancia pura y compresible el balance de energía es el siguiente:

\insertequation{ dU = Q + W_{EC}}

De forma diferencial esta ecuación queda:

\insertequation{du=dQ-PdV}

Donde podemos $-PdV$ hace referencia al trabajo de compresión/expansión reversible, no irreversible\footnote{Cuando es reversible la presión cambia lentamente, no hay roce. Mientras que cuando es irreversible hay roce, es un cambio brusco.}. Luego, por medio de la entropía tenemos la siguiente relación:

\insertequation{ dS=\frac{Q_{rev}}{T_{sys}}}

Dicha expresión viene del balance de entropía, que es el siguiente:

\insertequation{  \frac{dS}{dt}=\frac{{\dot{Q}}}{T}+\sum_{i=1}^N \dot{m}_i S_i + \dot{S}_{generada}    }{}

Lo cual nos deja la siguiente expresión para la energía interna en un sistema simple, cerrado e irreversible:

\insertequation[\label{eqn:du1}]{ dU=TdS -PdV}{}

En este caso se utiliza $-PdV$ dado a que se va desde lo reversiblea lo irreversible, permitiendo expresarlo de esta manera.

Como en la ecuación \ref{eqn:du1} puede escribirse el $dU$ a partir de $dS$ y $dV$, se denomnina función natural de $S$ y $V$. 

Podemos seguir trabajando con las expresiones de forma de poder generar más funciones naturales.

Teniendo la definición de la entalpía como $H=U+PV$ podemos generar la siguiente función natural de $H$:

\insertalign{
    H&=U+PV \\
    dH&=dU+VdP+PdV \nonumber\\ 
    dH&=TdS-PdV+VdP+PdV \nonumber\\ 
    dH&=TdS+VdP \label{eqn:dH1}
}

De modo que la ecuación \ref{eqn:dH1} muestra a la entalpía como una función natural de S y P. La manipulación que se realizó se denomina \textbf{Transformada de Legendre}.

La entalpía es una \textbf{propiedad conveniente} debido a que esta definida para que fuera útil en problemas donde el calor y la presión son manipuladas. El hecho que la entalpía relacione la transferencia de calor a presión constante en sistemas cerrados y la transferencia de calor con el trabajo en sistemas de flujo estacionario muestra el resultado de una buena elección de la definición.

En sistemas donde se pueden controlar T y V, como en situaciones de pistones o cilindros, la U no es función natural de T y V, por lo cual se requiere de otra definición para trabajar en estos sistemas. En esta situación se define la \textbf{energía de Helmholtz}, la cual se puede definir como un tipo de energía configuracional, también está relacionada con el trabajo de expanción/compresión en sistemas isotérmicos; esta energía está expresada por la siguiente ecuación:

\insertequation{A=U-TS}

De modo que al utilizar la transformada de Legendre podemos hacer lo siguiente:

\insertalign{
    dA&=dU-SdT-TdS \\
    dA&= TdS-PdV-SdT-TdS \nonumber\\
    dA&= -SdT-PdV \label{eqn:dA1}
}

De esta forma llegamos a la ecuación \ref{eqn:dA1} la cual es una expresión que muestra A como función natural de T y V, lo cual es muy beneficioso, ya que T y V son cantidades medibles.

El equilibrio ocurre cuando la derivada de Helmholtz es cero para V y T constante. A esta propiedad se le llama \textbf{energía de Gibbs}\footnote{También se le llama energía libre de Gibbs.}. Esta energía se define como:

\insertalign{
    G&=U-TS+PV\\
    G&=H-TS \\
    G&= A+PV
}

Luego, al utilizar la transformada de Legendre tenemos:

\insertalign{
    dG&=dU-SdT-TdS+PdV+VdP\\
    dG&=TdS-PdV-SdT-TdS+PdV+VdP \nonumber\\
    dG&=-SdT+VdP \label{eqn:dG1}
}

De esta forma, en la ecuación \ref{eqn:dG1} se puede ver que la energía de Gibbs es una función natural de T y P. Lo cual también es beneficioso ya que T y P son cantidades medibles.

La energía de Gibbs es usada específicamente en problemas de equilibrio de fase donde la temperatura y la presión son controladas. Cuando nos encontramos en un equilibrio de fases la temperatura y la presión son constantes, de forma que $dG=0$ en el equilibrio. También hay que mencionar que la energía de Helmholtz y Gibbs incluyen los efectos entrópicos de las fuerzas motrices, si la entroía aumenta le energía disminuye.

En resumen, las relaciones fundamentales para cada una de las relaciones importantes son:

\insertalign{
    dU&=TdS-PdV \\
    dH&=TdS+VdP \\
    dA&=-SdT-PdV \\
    dG&=-SdT+VdP
}

Como podemos notar de la ecuación \ref{eqn:dH1} y \ref{eqn:du1}, tanto la entalpía como la energía interna están en función de variables no medibles (de forma sencilla), como es la entropía. Por lo cual se quiere que estas esten en función de P o T, funciones medibles. Cabe recalcar que son \textbf{propiedades medibles}:

\begin{itemize}
    \item Presión, Volumen y Temperatura y derivadas que las incluyan.
    \item $C_p$ y $C_v$ que son funciones conocidas de la temperatura a baja presión.
    \item Se acepta también la entropía si no está dentro de un término derivativo. La entroía se puede carlcular desde propiedades medibles.
\end{itemize}

Por lo cual nuestro objetivo es encontrar una función que deje a H y U en función de propiedades medibles. Para esto es que es necesario introducir las siguientes propiedades matemáticas:

Supongamos que $F=F(x,y)$ entonces tenemos:

Identidades básica:

\insertequation{ \left(\frac{\partial x}{\partial y}\right)_z=\frac{1}{\left(\frac{\partial y}{\partial x}\right)_z}  }

\insertalign{
    \left(\frac{\partial x}{\partial y}\right)_x&=0  &  \left(\frac{\partial x}{\partial y}\right)_y&=\infty  &  \left(\frac{\partial x}{\partial x}\right)_y &=1 
}

Regla del producto triple:

\insertequation{  \left(\frac{\partial x}{\partial y}\right)_F \left(\frac{\partial y}{\partial F}\right)_x \left(\frac{\partial F}{\partial x}\right)_y = -1 }

\hfill

Regla de la cadena:

\insertequation{ \left(\frac{\partial x}{\partial y}\right)_F = \left(\frac{\partial x}{\partial z}\right)_F \left(\frac{\partial z}{\partial y}\right)_F }{}

Regla de la expansión:

\insertequation{\left(\frac{\partial x}{\partial y}\right)_z = \left(\frac{\partial x}{\partial k}\right)_m \left(\frac{\partial k}{\partial y}\right)_z  + \left(\frac{\partial x}{\partial m}\right)_k \left(\frac{\partial m}{\partial y}\right)_z }


Además de estas definiciones es imporante introducir la siguiente definición, debido a que nos ayudará a trabajar con las expresión del tipo $dx$ donde x es una variable termodinámica.

\textbf{Diferenciales exactas}. Podemos definir cualquier propiedad de estado en termodinámica a partir de otras dos propiedades. De modo que para una función que solo depende de dos variables se puede obtener la siguiente relación diferencial, lo que se llama en matemática \textbf{diferencial exacta}:
Por ejemplo, si definimos la energía interna como una función de la entropía y el volumen podemos generar la siguiente diferencial exacta.

\insertequation[\label{eqn:du2}]{ U=U(S,V) \Rightarrow dU=\left(\frac{\partial U}{\partial S}\right)_V dS+ \left(\frac{\partial U}{\partial V}\right)_S dV}

Sabiendo esta potente herramienta matemática podemos encontrar las expresiones de cualquier propiedad termodinámica a partir de diferenciales exactas y relaciones termodinámicas.

Ahora, ahondadno más en la ecuacion \ref{eqn:du2}, teniendo presente lo presentado en la ecuación \ref{eqn:du1} podemos notar que:

\insertalign{ T&=\left(\frac{\partial U}{\partial S}\right)_V & -P&=\left(\frac{\partial U}{\partial V}\right)_S }

A partir de esta información se pueden obtener las \textbf{Relaciones de Maxwell}, estas son:

\insertalign{
    -\left(\frac{\partial P}{\partial T}\right)_V&=-\left(\frac{\partial S}{\partial V}\right)_T \label{eqn:maxw1} \\
    \left(\frac{\partial T}{\partial P}\right)_S&=\left(\frac{\partial V}{\partial S}\right)_P \\
    \left(\frac{\partial V}{\partial T}\right)_P&=-\left(\frac{\partial S}{\partial P}\right)_T\\
    \left(\frac{\partial T}{\partial V}\right)_S&=-\left(\frac{\partial P}{\partial S}\right)_V
}

De esta forma podemos realizar conversiones por medio de estas relaciones para poder encontrar expresiones para cada propiedad termodinámica como una función natural de propiedades medibles P, T y V.

\subsubsection{Propiedades importantes}

Existen 3 propiedades típicamente usadas en termodinámica que están basadas en propiedades derivadas. Estas son:

\break

\textbf{Compresibilidad isotérmica}

\insertequation{ \kappa_T = \frac{-1}{V} \left( \frac{\partial V}{\partial P}   \right)_T= \frac{1}{\rho} \left( \frac{\partial \rho}{\partial P}\right)_T  }{}

Donde $\rho$ es la densidad molar de la sustancia.

\textbf{Coefficiente de expansión térmico}

\insertequation{\alpha_P = \frac{1}{V} \left( \frac{\partial V}{\partial T}   \right)_P= \frac{-1}{\rho} \left( \frac{\partial \rho}{\partial T}\right)_P  }{}

\textbf{Coeficiente de Joule-Thompson}

\insertequation{\mu_{JT}=\left( \frac{\partial T}{\partial P} \right)_H}{}

Como fue mencionado antes, se dice que el $C_p$ y el $C_v$ se consideran como propiedades medibles, por lo cual llego la hora de definir estas dos propiedades:

\insertalign{
    C_v=\left(\frac{\partial U}{\partial T}\right)_V \\
    C_p=\left(\frac{\partial H}{\partial T}\right)_P
}

\subsection{Propiedades Residuales}

Gracias a las relaciones de Maxwell, podemos dejar cualquier varible termodinámica en término de otras variables. Por lo cual, las propiedades residuales nos permiten manipular las propiedades de estado de forma de dejarlas en términos de variables conocidas y manipulables.

Sabemos en primera instancia que la energía interna puede ser expresada por medio de las siguientes varibales medibles:

\insertalign{
    dU&=\left( \frac{\partial U}{\partial T} \right)_V dT + \left( \frac{\partial U}{\partial V} \right)_T dV \nonumber\\
    dU&= C_v dT - \left[ T \left( \frac{\partial S}{\partial V} \right) + P \left( \frac{\partial V}{\partial V} \right) \right] dV \nonumber\\
    dU &= C_v dT + \left[ T \left( \frac{\partial P}{\partial T} \right)_v - P \right] dV \label{eqn:dutv}
}

Es a partir de la ecuación \ref{eqn:dutv} que podemos expresar la energía interna como función natural de la Temperatura y la Presión, las cuales son propiedades medibles. 
También podemos expresar el cambio de energía interna al integrar dicha ecuación, de forma que:

\insertequation{  \Delta U = \int_{T_1}^{T^2} C_v dT + \int_{V_1}^{V_2} \left[ T \left( \frac{\partial P}{\partial T} \right)_v - P \right] dV }{}

Para la entalpía podemos realizar una operación similar partiendo de la ecuación \ref{eqn:dH1}. Sabemos que :
\insertalign{
    dS&= \left( \frac{\partial S}{\partial T} \right)_V dT + \left( \frac{\partial S}{\partial V} \right)_T dV \nonumber\\
    dS&= \frac{C_p}{T} dT - \left( \frac{\partial V}{\partial T}  \right)_P dP \label{eqn:Stv}
}{}

El fundamento de estos reemplazos viene de que, para el primer término:

\insertalign{
    dH&= TdS + VdP \nonumber\\
\intertext{Al dividir por $\frac{1}{dT}$ a presión constante, nos queda que:}\\
    \left( \frac{\partial H}{\partial T} \right)_P &= T \left( \frac{\partial S}{\partial T} \right)_P + V \left( \frac{\partial P}{\partial T} \right)_P\\
\intertext{Y como estamos a presión constante, y que $\left( \frac{\partial H}{\partial T} \right)_P =C_P$, entonces:}\\
    \frac{C_P}{T} &=\left( \frac{\partial S}{\partial T} \right)_P
}

El segúndo término es reemplazado desde una de las relaciones de Maxwell, en específico de aquella en la ecuación \ref{eqn:maxw1}.

Luego al reemplazar la ecuación \ref{eqn:Stv} en la ecuación \ref{eqn:dH1} tenemos que:

\insertalign{
    dH&=TdS + VdP \nonumber\\
    dH&= T \left[ \frac{C_p}{T} dT - \left( \frac{\partial V}{\partial T}  \right)_P dP  \right] + VdP \nonumber\\
    dH &= C_p dT - T \left( \frac{\partial V}{\partial T}\right)_P dP + VdP \nonumber\\
    dH &= c_p dT + \left[ V - T \left( \frac{\partial V}{\partial T} \right)_P \right] dP
}

Integrando esta relación para poder encontrar el cambio de entalpía llegamos a que:

\insertequation{  \Delta H = \int_{T_1}^{T_2} c_p dT + \int_{P_1}^{P_2}\left[ V - T \left( \frac{\partial V}{\partial T} \right)_P \right] dP }{}

Podemos encontrar diferentes caminos para calcular el cambio de, por ejemplo, U, para llegar desde un estado ($V_L$,$T_L$) a un estado ($V_H$,$T_H$).
\insertimage[\label{path}]{img/imagenes/path}{width=5cm}{Podemos ver dos caminos obvios, los cuales se aprovechan de cambios tanto isocóricos como isotérmicos.}

Como podemos ver en la figura \ref{path}, los caminos obvios que se pueden tomar son dos.
De esta forma tenemos dos formas de calcular la energía interna:

\insertalign{
  \nonumber  \Delta U &= \int C_v \vert_{V_L}dT + \int \left[ T \left( \frac{\partial P}{\partial T} \right)_v - P \right]\vert_{T_H} dV \\
  \nonumber  \Delta U &= \int C_v \vert_{V_H} dT + \int \left[ T \left( \frac{\partial P}{\partial T} \right)_v - P \right]\vert_{T_L} dV 
}

Utilizando esta misma analogía es que definimos las propiedades residuales. Estas propiedades utilizan este principio de que al ser funciones de estado, no depende del camino, por lo cual buscamos evitar el uso de $C_v$ y $C_p$, para fluidos reales. Esto debido a que su cálculo se hace tedioso y poco confiable en situaciones lejanas de la idealidad.
Las propiedades residuales se utilizan como una forma de calcular los $Delta$ de propiedades por medio del cálculo de dicha propiedad en idealidad, y después sumarle una corrección.
De modo que se realiza el siguiente camino:

\insertimage[\label{img:resi}]{img/imagenes/residual}{width=5cm}{Camino recorrido para calcular cambios de cierta propiedad por medio de las propiedades residuales.}

Como podemos ver en la imagen \ref{img:resi} para calcular el cambio de la propiedad F, desde el punto A al punto B, se toma un camino donde \textbf{1} y \textbf{3} son caminos isotérmicos, por lo cual no se utiliza el $C_p$ ni el $C_v$; y el camino \textbf{2} es un camino que se hace en condiciones de gas ideal, por lo cual podemos usar $C_p^{ig}$ y $C_v^{ig}$ dado que se conoce el comportamiento de estos valores.
De modo que el $\Delta F$ queda definido como:

\insertalign{
    \Delta F_{12}= (F_2 - F_2^{ig})+(\Delta F^{ig}) + (F_1^{ig}-F_1)\\
    \Delta F_{12}= F_2^R +\Delta F^{ig} - F_1^R
}

Donde podemos darnos cuenta que la definición de $F^R$ es:

\insertequation{ F^R=F-F^{ig}}{}

Por medio de estas propiedades podemos calculas cualquier cambio de propiedades, por ejemplo:

\insertalign{
    \Delta H_{12}=\int_{T1}^{T2}C_p^{ig} dT + H_2^R - H_1^R\\
    \Delta S_{12}=\int_{T1}^{T2}\frac{C_p^{ig}}{T} dT - R\ln(\frac{P_2}{P_1}) +S_2^R-S_1^R
}

Ahora procederemos a calcular las propiedades residuales:
\vspace{0.5cm}

\textbf{Entalpía residual:}

Partimos desde que:

\insertequation{dH^R=dH-dH^{ig}}

Como en estas propiedades siempre tomaremos un camino isotérmico para llegar desde el punto "Real" al punto "ideal", podemos expresar la entalpía residual como:

\insertequation{dH^R=\left[ V-T\left(\frac{\partial V}{\partial T} \right)_P \right]dP}{}

Al integrar esta expresión, vamos a ir desde la  ´idealidad´, donde $P\approx 0$ hasta la presión del punto ´real´ $P$, quedando así que:

\insertequation{ H^R= \int_{0}^P  \left[ V-T\left(\frac{\partial V}{\partial T} \right)_P \right] dP }{}


\vspace{0.5cm}
\textbf{Entropía residual:}

De forma análoga a la entalpía, podemos calcular la entropía residual:
\insertequation{ dS^R=dS-dS^{ig}}{}
Como dT=0.
\insertequation{dS^R=\left[ \frac{R}{P}-\left( \frac{\partial V}{\partial T} \right)_P \right] dP}{}
Al igual que en la entalpía, tenemos que $P^{ig}\approx 0$, por lo cual tenemos que:
\insertequation{    dS^R=\int_{0}^P \left[ \frac{R}{P}-\left( \frac{\partial V}{\partial T} \right)_P \right] dP
}{}

\vspace{0.5cm}
\textbf{Volumen residual:}

Sabemos por la ecuación de los gases ideales que:
\insertequation{V^{ig}=\frac{RT}{P}}{}

Por lo cual la definición del volumen residual es:

\insertequation{V^R=V-\frac{RT}{P}}{}

Si utilizamos el coeficiente de compresión, entonces obtenemos que:

\insertequation{V^R=\frac{RT}{P}(Z-1)}

\vspace{0.5cm}

\textbf{Otras propiedades residuales:}

A partir de $V^R$, $H^R$ y $S^R$ podemos calcular el resto de propiedades residuales:

\insertalign{
    U^R=H^R-PV^R\\
    G^R=H^R-TS^R\\
    A^R=U^R-TS^R
}

\subsubsection{Ecuaciones de estado y propiedades residuales}

Podemos usar utilizar las ecuaciones de estado para predecir una propiedad residual. Como en este caso, las ecuaciones residuales antes presentadas tienen al volumen como variable, se deben utilizar las formas cúbicas de las ecuaciones de estado. Por lo cual es que dejar 
las expresiones de las propiedades residuales en términos de la presión y la temperatura puede ser más ventajoso. De modo que:

\insertalign{
    H^R=PV-RT+\int_{\infty}^V \left[ T\left( \frac{\partial P}{\partial T} \right)_V -P\right]dV\\
    S^R=R\ln(\frac{PV}{RT})+\int_{\infty}^V \left[ \left( \frac{\partial P}{\partial T} \right)_V - \frac{R}{V}\right]dV
}

Aquí la ventaja radica en que las ecuaciones cúbicas tienen una expresión explicita para la presión, por lo cual el cálculo de $\frac{\partial P}{\partial T}$ es más sencillo que $\frac{\partial V}{\partial T}$.

Luego, para cada ecuación de estado, podemos encontrar una expresión para las propiedades residuales.

\textbf{Van der Waals:}
\insertimage{img/imagenes/residualvdw}{width=4.5cm}{Propiedades residuales para una EoS tipo Van der Waals.}

\textbf{Soave-Redlich-Kwong:}
\insertimage{img/imagenes/residualsrk}{width=7.7cm}{Propiedades residuales para una EoS tipo SRK.}
\break
\textbf{Peng-Robinson:}
\insertimage{img/imagenes/residualpr}{width=10cm}{Propiedades residuales para una EoS tipo Peng-Robinson.}

\textbf{Virial:}
\insertimage{img/imagenes/residualvirial}{width=8cm}{Propiedades residuales para una EoS tipo Virial.}

\subsection{Equilibrio de fases}

Ek equilibrio de fases es una situación en donde hay equilibrio entre fases líquidas y gaseosas. En estas condiciones los balances de energía y materia se hacen insuficientes, por lo cual la determinación del equilibrio de fases
es una de las propiedades que son difíciles de predecir.

Para realizar este tipo de predicciones es necesario utilizar la energía libre de Gibbs, en función de la presión y la temperatura.

\insertequation{dG=-SdT + VdP}{}

\subsubsection{Críterio para el equilibrio de fases}

El volumen del vapor y el volumen del líquido se mantienen constantes, sin embargo, el volumen total cambia y por tanto lo hace la cantidad de líquido como de vapor, esta relación es:

\insertequation{\underline{V}=n^L V^{L,sat}+n^V V^{V,sat}}{}

En definitiva, los moles de líquido saturado y vapor saturado cambian.

Ahora, en términos de la energía de Gibbs, como en el equilibrio líquido-vapor, la isoterma también es isóbara, tenemos que $dT=0$ y $dP=0$. Por lo cual, tenemos que:

\insertequation{dG=0}{}

Lo cual se puede traducir a:

\insertequation{G^L = G^V}{}

Por lo cual, en condiciones de equilibrio en un compuesto puro, la presión, temperatura y energía libre de Gibbs molar son constante, sin importar la cantidad de fases.
De esta forma, tenemos que la energía de Gibbs molar se le llama también como \textbf{potencial químico}$\mu$.

\subsubsection{Ecuación de Clausius-Clapeyron}

Si queremos encontrar la pendiente de la curva de presión de vapor $\frac{dP^{sat}}{dT}$, entonces debemos notar que cuando estamos en el equilibrio de fases, tenemos que:

\insertequation{dG^L=dG^V}

A partir de la ecuación \ref{eqn:dG1} podemos reordenar esta relación y llegar a:

\insertequation[\label{eqn:precc}]{(V^V - V^L)dP^{sat}=(S^V-S^L)dT}{}

Luego vamos a tener que la entropía puede ser relacionada con la entalpía de vaporización por medio de la siguiente ecuación:

\insertequation{S^V-S^L=\Delta S^{vap}=\frac{H^V - H^L}{T}=\frac{\Delta H^{vap}}{T}}{}

Reemplazando esta expresión en \ref{eqn:precc} logramos obtener la \textbf{Ecuación de Clapeyron:}

\insertequation[\label{eqn:cla}]{\frac{dP^{sat}}{dT}=\frac{\Delta H^{vap}}{T(V^V - V^L)}}{}

Si puede multiplicar por $T^2$ y dividir por $P^{sat}$ para poder obtener la ecuacion \ref{eqn:cla} en función de Z.

\insertequation{\frac{T^2}{P^{sat}}\frac{dP^{sat}}{dT}=\frac{\Delta H^{vap}}{R(Z^V - Z^L)}}{}

Luego aplicando algunas reglas de cálculo\footnote{Magia matemática.} podemos obtener la siguiente expresión:

\insertequation{d\ln P^{sat}=\frac{-\Delta H^{vap}}{R(Z^V-Z^L)}d\left(\frac{1}{T}\right)}{}

Cuando estamos tratando con gases lejanos del punto crítico a baja temperatura reducida, tenemos que $Z^V - Z^L \approx Z^V$, y en presiones cercanas a 1 $bar$, donde estamos en condiciones cercanas a la idealidad, $Z^V \approx 1$. Entonces

\insertequation[\label{eqn:cc}]{d\ln P^{sat}=\frac{-\Delta H^{vap}}{R}d\left(\frac{1}{T}\right)}{}

La ecuación \ref{eqn:cc} se conoce como \textbf{ecuacion de Clausius-Clapeyron}\footnote{Para esta ecuación es más sencillo calcular el $\frac{dP^{sat}}{dT}$ por medio de calcular la pendiente.}.

La ecuación \ref{eqn:cc} es importante debido a que a partir de esta se puede generar la \textbf{ecuación de Antoine} por medio de ajustar los parámetros.
Esta ecuación es:

\insertequation[\label{eqn:Antoine}]{\log_{10} (P^{sat})= A-\frac{B}{T-C}}{}

La ecuación de Antoine (\ref{eqn:Antoine}) presenta 3 parámetros $A,B,C$, estos parámetros son extraidos desde bibliografía, y solo funcionan en los intervalos de temperatura que se encuentran estipulados en la bibliografía; a su vez, en la bibliografía podremos encontrar si es $ln$ o $log_{10}$ y cuales son las unidades de medida de la presión y temperatura.
Esta ecuación es muy importante, debido a que nos permite encontrar la presión de saturación a una temperatura dada de una forma más sencilla que la ecuacion \ref{eqn:cc}

\subsubsection{Cambios en la energía de Gibbs con la presión}

Partiendo desde la ecuación fundamental de la energía de Gibbs:

\insertequation{dG=-SdT + VdP}{}

Como queremos ver el efecto de la presión, asumimos $dT=0$, entonces tenemos que:

\insertequation{dG=VdP}{}

Esta ecuación es la base de la mayoría de las derivaciones en equilibrios de fase. Para evaluar los cambios de G necesitamos P-V-T de los fluidos\footnote{Pueden estar tabuladas o extraidas de EoS}. Integrando esta expresión tenemos que:

\insertequation{G_2-G_1=\int_{P1}^{P2}VdP \text{(T cte)}}{}

Cuando estamos trabajando con fluidos reales, podemos dejar $dG$ en función de Z a través de la siguiente ecuacion:

\insertequation{dG=RTZ \frac{dP}{P}}{}

Esto nos permite utilizar correlaciones generalizadas o EoS explicitas para Z en función de T y P. Cuando estamos en un gas ideal $Z=1$, por ende:

\insertequation{dG^{ig}=RT\frac{dP}{P}=RTd\ln P}{}

Luego

\insertequation{\Delta G^{ig}=\int_{P1}^{P2}\frac{RT}{P}dP = RT \ln \frac{P_2}{P_1}}{}

Tanto $dG$ como $dG^{ig}$ tienden a infinito cuando $P\approx 0$, por lo cual son difíciles de tratar a bajas presiones. Pero cuando estamos trabajando con fluidos reales, cuando $P\rightarrow 0 \Rightarrow Z\rightarrow 1$. De esta forma $dG-dG^{ig}$ se mantiene finito, y tiende a 0 cuando la presión tiende a 0.

A partir de $dG-dG^{ig}$ podemos obtener una nueva función residual:

\insertalign{dG-dG^{ig}&= (V-V^{ig})dP\\
    &=\left( \frac{ZRT}{P}-\frac{RT}{P}  \right) dP \\
    &= \frac{RT}{P} (Z-1) dP 
}
Con lo que llegamos a

\insertequation[\label{eqn:Dgibbs}]{\frac{d(G-G^{ig})}{RT}=\frac{Z-1}{P}dP}{}

Esta nueva propiedad residual se utiliza para definir una nueva propiedad, la \textbf{fugacidad}.

\subsubsection{Fugacidad}

En un principio la energía libre de Gibbs nos permite resolver todos los problemas de equilibrio de fase, sin embargo se introdujo la fugacidad como una propiedad que nos permite hacer esto mismo. Pero la fugacidad posee una ventaja por sobre G, y es que para mezclas es una sencilla extensión del trabajo para fluidos puros.

G.N. Lewis define la fugacidad como:

\insertequation{dG=VdP=RTd\ln f}{}

Por medio de la definición en \ref{eqn:Dgibbs} tenemos que:

\insertequation[\label{eqn:DG}]{d(G-G^{ig}) = RTd\ln \frac{f}{P}   }{}

Donde $f$ es la fugacidad del fluido. Y esta se define por:

\insertequation[\label{eqn:defuga}]{f=\varphi P}{}

Donde $\varphi$ se define como el coeficiente de fugacidad. Cuando estamos tratando con un gas ideal se cumple que $\varphi=1$ lo que implica que $f^{ig}=P$. Mientras que para un fluido real $\varphi \neq 1$.
Integrando la ecuacion \ref{eqn:DG} desde una presión baja, a temperatura constante, tenemos que:

\insertequation[\label{eqn:fuga}]{\frac{G-G^{ig}}{RT}=\ln(\frac{f}{P})=\ln \varphi}{}

De esta forma, el coeficiente de fugacidad es otra manera de caracterizar la energía residual de Gibbs a T y P fijas. 
Podemos seguir trabajando la ecuación \ref{eqn:fuga} de forma de llegar a que:

\insertequation[\label{eqn:fugav}]{\ln(\frac{f}{P})= \ln \varphi = \frac{1}{RT} \int_{0}^{P} \left( V-\frac{RT}{P}  \right) dP }{}

Utilizando el coeficiente de compresibilidad en la ecuación \ref{eqn:fugav} llegamos a que:

\insertequation{ \ln(\frac{f}{P})= \ln \varphi = \frac{1}{RT} \int_{\infty}^{V} \left( \frac{RT}{V}-P \right)dV + (Z-1)-\ln Z  }{}

La fugacidad nos permite evaluar la no-idealidad de un fluido mediante que tan lejano es este valor de 1.

\subsubsubsection{Fugacidad para equilibrio de fases}

Partiendo de que en el equilibrio de fases, tenemos que:

\insertequation{G^L = G^V }{}

Al restar a ambos lados la energía de Gibbs ideal, y dividir por RT, llegamos a que:

\insertequation{ \frac{(G^L-G^{ig})}{RT} = \frac{(G^V-G^{ig})}{RT} }{}

Reemplazando en la ecuación \ref{eqn:fuga} tenemos que:

\insertequation{ \ln(\frac{f^L}{P}) = \ln(\frac{f^V}{P}) }{}

Lo que permite llegar a que:
\insertequation{ f^V=f^L}{}

\insertequation{ \varphi^V  = \varphi^L}{}

De esta forma, tanto la fugacidad como el coeficiente de fugacidad nos van a permitir calcular las fases de equilibrio.

\subsubsubsection{Cálculo de fugacidad en gases}

Para el cálculo de la fugacidad en gases se debe proceder primero al cálculo del coeficiente de fugacidad, y luego utilizar la ecuación \ref{eqn:defuga}.

La forma de calcular $\varphi$ será diferente para cada ecuación de estado:

\textbf{Gas ideal:}

\insertequation{\varphi^{ig}=1 \text{ y } f^{ig}=P}{}

\textbf{Ecuación Virial:}

\insertalign{ \ln \varphi &= \frac{BP}{RT}\\
&= \frac{P_r}{T_r} (B^0 + \omega B^1)
}{}

Donde $B^0$ y $B^1$ son los coeficientes de la ecuación virial definidos en \ref{eqn:virial}.

\textbf{Van der Waals:}

\insertequation{ \ln \varphi = Z-1-\frac{a}{RTV}-\ln\left[ Z\left( 1-\frac{b}{V} \right)  \right]  }

Donde $a$ y $b$ son los coeficientes de la ecuación van der Waals definida en \ref{eqn:vdw}.

\textbf{Soave-Redlich-Kwong:}

\insertalign{
    \ln \varphi &= Z-1-\ln(Z-B') -\frac{A'}{B'}\ln\frac{Z+B'}{Z}\\
    A'&=\frac{aP}{(RT)^2}\\
    B'&=\frac{bP}{RT}
}

Donde $a$ y $b$ son los coeficientes de la EoS SRK en la ecuación \ref{eqn:srk}.

\textbf{Peng-Robinson:}

\insertalign{
    \ln \varphi &= Z-1-\ln(Z-B') - \frac{A'}{2\sqrt{2}B' \ln \frac{Z+(1+\sqrt{2}B')}{Z+(1-\sqrt{2}B'}}\\
    A'&=\frac{aP}{(RT)^2}\\
    B'&=\frac{bP}{RT}
}


Para casos generalizados podemos usar:

\insertequation{  \ln \varphi = \frac{1}{RT} \int_{0}^{P} \left( V-\frac{RT}{P}  \right) dP = \frac{1}{RT} \int_{\infty}^{V} \left( \frac{RT}{V}-P \right)dV + (Z-1)-\ln Z 
}{}

Y también, para aquellos casos que se disponga de gráficos tenemos que 

\insertequation{\ln \varphi = \ln \varphi^0 + \omega \ln \varphi^1}{}

Estos valores se obtienen de los siguiente gráficos:

\begin{images}{Diagramas que representan los dos parámetros de la ecuación de arriba.}
    \addimage{img/imagenes/lnphi0}{width=7cm}{$\ln \varphi^0$}{}
    \addimage{img/imagenes/lnphi1}{width=7cm}{$\ln \varphi^1$}{}
\end{images}

\subsubsection{Cálculo de fugacidad en líquidos}

Para entender como calcular la fugacidad de un líquido, hay que tener presente el siguiente diagramam:

\insertimage[\label{img:poy}]{img/imagenes/fugacity_liq}{width=7cm}{Diagrama que representa el cambio de estados a temperatura constante.}{}

Para poder obtener la fugacidad de un líquido se utiliza el \textbf{Método de Poynting}.
En primer lugar a partir de la definición de fugacidad podemos obtener que:

\insertequation{ RT\ln \frac{f_D}{f_{sat}}=\int_{P^{sat}}^{P_D}VdP }{}

Los líquidos por lo general al sufrir cambios grandes de presión, no sufren grandes cambios de volumen, como se puede ver en el cambio desde el estado D al C en la figura \ref{img:poy}.
Por lo cual, podemos asumir que el líquido es incompresible, lo que nos deja que:

De esta forma obtenemos:

\insertequation[\label{eqn:poy1}]{f = \varphi^{sat}P^{sat} \exp(\frac{V^{L,sat}(P-P^{sat})}{RT})  }{}

El volumen del líquido saturado puede ser obtenido de forma experimental (por medio de la densidad) o estimando con la ecuación de \textbf{Rackett}\footnote{Esta ecuación solo debe ser usada para este caso, en otros casos no tiende a funcionar bien la aproximación}. La ecuación de \textbf{Rackett} es:

\insertequation{V^{L,sat} = V_c Z_c^{(1-T_r)^{0.2857}} }{}

En la ecuación \ref{eqn:poy1} el exponencial presente se denomina como factor de Poynting:

\insertequation{POY= \exp(\frac{V^{L,sat}(P-P^{sat})}{RT})  }

Cuando estamos en presiones bajas tenemos que $\varphi^{sat}\approx 1$. Cuando $P\approx P^{sat}$ se tiene que $POY\approx 1$.
Luego, por esta razón, cuando tenemos presiones bajas y cercanas a la de saturación:

\insertequation{f=P^{sat}}{}

Cabe recalcar que para la mayoría de compuestos en condiciones normales se tiene que $POY\approx 1$, por lo cual:

\insertequation{f^L\approx \varphi^{sat}P^{sat}}{}

Cabe mencionar que el método de Poynting no solamente puede ser utilizado según plantea la ecuación \ref{eqn:poy1}. Este método en realidad lo que hace es hacer una corrección a la fugacidad en un punto isocórico e isotérmico (mismo volumen y temperatura), y le hace una corrección, la cual viene dada por el $POY$. De esta forma,
generalizando tenemos que, si sabemos la fugacidad en el punto A, la cual es $f_A=\varphi_A P_A$, podemos cálcular la fugacidad en el punto B, por medio de la siguiente ecuación:

\insertequation{ f_B= f_A \exp( \frac{V (P_B - P_A)}{RT} ) }{}

Siempre y cuando $V_B=V_A=V$ y $T_B=T_A=T$.
\subsubsection{Cálculo de fugacidad en sólidos}

Para el cálculo de la fugacidad en la fase sólida también se utiliza el método de Poynting, la única diferencia es que el volumen utilizado es $V^{S}$. Quedando la ecuación como:

\insertequation{f^S =   \varphi^{sat}P^{sat} \exp(\frac{V^{S}(P-P^{sat})}{RT})}{}

El coeficiente de fugacidad en la saturación se puede obtener con cualquiera de los métodos utilizados para la fase de vapor.

Al igual que con los líquidos, el factor de Poynting es usualmente cercano a 1, por lo que la fugacidad se puede aproximar a:

\insertequation{f^S \approx \varphi^{sat}P^{sat}}{} %i1

\section{Sistemas Multicomponentes}

\subsection{Introducción}

En primer lugar vale la pena introducir qué es un sistema múlticomponente. Un sistema múlticomponente es aquel que presenta más de dos sustancias diferentes, por lo cual ya no nos encontramos trabajando con compuestos puros.
Es en este caso donde tenemos sistemas compuestos por mezclas con elementos similares o completamente diferentes.

El comportamiento de estas mezclas es el componente básico en la industria donde ocurren separaciones. Lo que hace que las separaciones sean factibles es que podemos llevar la mezcla a un estado donde las distintas fases con diferentes composiciones pueden coexistir.

En primera instancia, para analizar estos sistemas hay que tener presenta la \textbf{Regla de las Fases de Gibbs}, la cuál viene dada por la siguiente ecuación.

\insertequation{GL=2-\phi+N}{}

Esta ecuación define los grados de libertad que se tienen en un problema. Donde $\phi$ es el número de fases en equilibrio, y $N$ corresponde al número de compuestos.

Por ejemplo, en una mezcla binaria (dos compuestos) existen dos grados de libertad. Es decir, a presión constante, tanto la temperatura como la composición\footnote{Este comcepto será introducido más adelante.} pueden variar. De esta manera, para resolver estos problemas
Cuando tenemos una variable constante, se requeriran de dos para resolverlo.

\subsubsection{Los diagramas de fases}

Cuando analizamos un sistema multicomponentes vamos a tener un gráfico en el cual vamos a mantener una de las variables (presión o temperatura), analizar el comportamiento P vs composición, o T vs composición. 

Este comportamiento se ve reflejado en los siguientes diagramas.

\begin{images}{Diagramas de fases}
    \addimage[]{img/imagenes/tvscomp1}{width=7.5cm}{Diagrama Temperatura v/s Composición.}
    \addimage[]{img/imagenes/pvscomp1}{width=7.5cm}{Diagrama Presión v/s Composición.}
\end{images}

La zona que podemos encontrar demarcada con blanco, en contraste con el gris del gráfico, es la zona donde coexisten las dos fases de ambos compuestos; en otras palabras,
esta zona se compone de los dos compuestos en un equilibrio líquido-vapor, esta zona se denomina \textbf{envoltura de fases} o clásicamente \textit{\textbf{la lenteja}}.

\subsubsubsection{Análisis de diagramas de fases}
\clearpage
\subsection{Separadores Flash}

Los separadores flash son frecuentemente usados en la industria para separar una corriente de vapor saturado de una de líquido saturado.

Este se puede diagramar por medio del siguiente diagrama.
\insertimage[\label{img:flash}]{img/imagenes/flash1}{width=7cm}{Diagrama de un separador flash, la corriente F hace referencia a la correinte de entrada, la corriente V a la corriente de salida de vapor saturado, y la corriente L a la corriente de salida de líquido saturado.}

\subsubsection{Cálculos Flash}
A la hora de realizar cálculos en separadores Flash tenemos que las cantidades molares de cada uno de los compuestos 
junto con un balance de masa nos permiten hacer cálculos sobre composiciones en la región de dos fases, esto se conoce como cálculos flash.

Tenemos en un principio que el número inicial de moles se denota F, los cuales se separan en L moles de líquido y V moles de vapor. Esto nos deja el siguiente balance global:

\insertequation[\label{eqn:flash1}]{F=L+V}{}

Con esto también podemos darnos cuenta que las fracciones de vapor y líquido suman uno, esto viene por la siguiente relación dada por dividir por F la ecuación \ref{eqn:flash1}.

\insertequation{1=\frac{L}{F}+\frac{V}{F}}{}

Podemos tomarlo por componentes, con lo cual llegariamos a que:

\insertequation{
    z_A F= y_a V+ x_A L
}

Lo que en palabras es "\textit{La composición global por el flujo de entrada es igual: a la composición de vapor por la correinte de vapor,mas la composición de líquido por la corriente de líquido}. 

Con esta ecuación podemos deducir una regla importante para este análisis, la \textbf{regla de la palanca}. Esta se deduce de la siguiente forma:

\insertalign{
    z_A F= y_a V+ x_A L \nonumber\\ 
    z_A=y_A\frac{V}{F}+x_A\frac{L}{F} \nonumber\\
    z_A=x_A\left( 1-\frac{V}{F} \right) + y_A \frac{V}{F}\nonumber\\
    z_A=x_A - x_A\frac{V}{F}+y_A\frac{V}{F}\nonumber\\
    \frac{z_A-x_A}{y_A-x_A}=\frac{V}{F} \nonumber\\
    \frac{V}{F}= \frac{z_A-x_A}{y_A-x_A} \label{eqn:palanca1}
}

Donde la última expresión, la ecuación \ref{eqn:palanca1} es la regla de la palanca. También esta se puede tomar por la siguiente relación:

\insertequation{ \frac{L}{F}=\frac{y_A-z_A}{y_A-x_A}  }{}

La cual se obtiene de una forma similar a la anterior.

\subsection{Equilibrio Líquido-Vapor}

Dependiendo de la información que se entrega se pueden realizar diferentes tipos de cálculos
para modelar la partición líquido-vapor. Los tipos de problemas son:

\begin{itemize}
    \item Presión de burbuja \textbf{BP}
    \item Presión de rocio \textbf{DP}
    \item Tempreratura de burbuja \textbf{BT}
    \item Tempreratura de rocio \textbf{DT}
    \item Flash isotérmico \textbf{FL}
    \item Flash adiabatico \textbf{FA}
\end{itemize}

\subsubsection{Principios de cálculo}

La mayoría de las aproximaciones que buscan resolver problemas en un equilibrio líquido-vapor (ELV) utilizan la razón entre la fracción molar del vapor con la del líquido
conocida como \textbf{Coeficiente de partición} o \textbf{K-Ratio}. El cual viene dado por la siguiente ecuación:

\insertequation{
    K_i=\frac{y_i}{x_i}
}{}

Este coeficiente se comporta de manera que nos indica si es que hay más cantida de vapor o de líquido en la mezcla para el compuesto $i$. 
Para un $K_i$ mayor nos indicará que hay más vapor en la mezcla, y para un $K_i$ menor hay más líquido.

La información sobre las propiedades físicas conocidad, combinando con el K-Ratio, nos permite resolver cada uno de los problemas anteriormente mencionados.

Los métodos usados para calcular el $K_i$ varían según el método a seguir. Además, estos varían con la composición, presión y temperatura.

\subsubsection{Estratégias para resolver problemas ELV}

Pueden notar que solo existen 6 tipos de problemas que involucran un ELV. Usualmente los problemas se resuelven relativamente rápido una vez
que se ubican dentro de la tabla. Se puede utilizar como estrategia general los siguientes puntos:

\begin{itemize}
    \item Decicidir si se conoce la composición del líquido, vapor o la global del enunciado.
    \item Identigicar si el fluido está en el punto de burbuja o rocío.
    \item Identificar si P, T o ambas son constantes. Decidir si el sistema es adiabático.
    \item Utilizando la informaicón anterior, decicir en que fila nos debemos posicionar.
\end{itemize}

\subsubsection{Diagrama xy}

Este diagrama es un diagrama que nos indica como se comporta la composición del vapor en función de la composición del líquido.
En este cuando se tiene una intersección entre la línea de la composición de y con la recta que nos indica x=y, se dice que hay un azeótropox. Este punto es aquel en el cual la composición del líquido es la misma que la del vapor.
\insertimage[]{img/imagenes/diagxy}{width=7.5cm}{Diagram de composición \textit{y} vs composición \textit{x}}
\clearpage
\subsection{Destilación}
\insertimage[]{img/imagenes/desti1}{width=10cm}{Diagrama del proceso se destilación.}

La destilación es un proceso continuo de separadores Flash lo cual nos permite purificar un compuesto de otro.
En este proceso un compuesto \textit{light} sube dado que es más volátil, mientras que otro \textit{heavy} va a bajar dado que es menor volátil. Esta separación de fases es fundamental y para esto se define la \textbf{volatilidad relativa}.

\insertequation{\alpha_{ij}=\frac{K_i}{K_j}}{}

Donde i hace referencia al más liviano, mientras que j al más pesado, quedando definida como:

\insertequation{\alpha_{LH}=\frac{K_LK}{K_HK}}{}

Para una buena separación es primordial que $\alpha_{LH}>1$.

Estos proceso de destilación se analizan por las siguientes curvas de destilación.

\begin{images}{Diagramas de Destilación.}
    \addimage[]{img/imagenes/desti2}{width=5cm}{Diagrama Temperatura-Composición de una destilación.}
    \addimage[]{img/imagenes/desti3}{width=5cm}{Diagrama Composición-Composición de una destialción.}
\end{images}

Hay que tener presente que dado que se hacen separaciones según las fases, cuando nos encontramos con un azeótropo, no se puede destilar sobre este punto. Por eso, por ejemplo, el etanol no se puede alcanzar una pureza del 100\%.

\subsection{Sistemas No Ideales}

La Ley de Raoult nos sirviría solamente en los casos que los compuestos son de similar función y estructura química. Sin embargo, en la realidad estos sistemas son escasos en la naturales.
Existen 4 casos de no-idealidad, estos son:
\begin{itemize}
    \item Desviación positiva de la Ley de Raoult: Hace referencia a cuando la curva del diagrama presión-composición esta por sobre lo estimado por la Ley de Raoult.
    \item Azeótropo de presión máxima: Es cuando se forma un azeótropo en condiciones de una Desviación positiva de la Ley de Raoult.
    \item Desviación negativa de la Ley de Raoult: Hace referencia a cuando la curva del diagrama presión-composición esta por debajo a lo estimado por la Ley de Raoult.
    \item Azeótropo de presión mínima: Es cuando se forma un azeótropo en condiciones de una Desviación negativa de la Ley de Raoult.
\end{itemize}

Siendo los dos últimos casos los más extraños. Podemos ver estos casos representados en las siguientes imágenes.

\begin{images}{Diagramas que representan la no-idealidad.}
    \addimage[]{img/imagenes/raoultdp}{width=8cm}{Diagrama que presenta una desviación positiva de la ley de Raoult, junto a un azeótropo de presión máxima.}
    \addimage[]{img/imagenes/raoultdn}{width=8cm}{Diagrama que presenta una desviación negativa de la ley de Raoult, junto a un azeótropo de presión mínima.}
\end{images}

\subsubsection{Conceptos para un equilibrio de fases generalizado}

La generalización desde los Principios de compuesto puro a multicomponente requiere que consideremos como las propiedades termodinámicas cambian con respecto a la
cantidad individual de cada componente. Para un fluido puro, las propiedades eran sencillamente una función de dos variables. Para una mezcla multicomponente, las energías
y entropía también dependen de la composición.
\insertalign{
    d\underline{U}(T,P,n1,n2,\ldots,ni)=\left(\frac{\partial \underline{U}}{\partial P}\right)_{T,n} dP+\left(\frac{\partial \underline{U}}{\partial T}\right)_{P,n}+ \sum_i \left( \frac{\partial \underline{U}}{\partial ni} \right)_{P,T,n_{j\noteq i}} dn_i\\
    d\underline{G}(T,P,n1,n2,\ldots,ni)=\left(\frac{\partial \underline{G}}{\partial P}\right)_{T,n} dP+\left(\frac{\partial \underline{G}}{\partial T}\right)_{P,n}+ \sum_i \left( \frac{\partial \underline{G}}{\partial ni} \right)_{P,T,n_{j\noteq i}} dn_i
}{}

A composición constante la mezcla debe seguir las mismas restricciones que un fluido puro. Esto se puede traducir en:
\insertalign{
    \left(\frac{\partial \underline{G}}{\partial P}\right)_{T,n}=\underline{V}\\
    \left(\frac{\partial \underline{G}}{\partial T}\right)_{P,n}=-\underline{S}
}

Con esto podemos reordenar la ecuación antes mencionada, quedando en:
\insertequation{
    d\underline{G}=\underline{V} dP-\underline{S}dT+\sum_i \left( \frac{\partial \underline{G}}{\partial ni} \right)_{P,T,n_{j\noteq i}} dn_i
}{}

La propiedad que se encuentra en la sumatoria se define como el potencial químico ($\mu_i=\left( \frac{\partial \underline{G}}{\partial ni} \right)$). Quedando finalmente la ecuación como:

\insertequation{
    d\underline{G}=\underline{V} dP-\underline{S}dT+\sum_i \mu_i dn_i
}{}

\subsubsection{Propiedades parciales molares}

Tenemos que el potencial químico queda definido como:
\insertequation{
    \mu_i=\left( \frac{\partial \underline{G}}{\partial ni} \right)_{P,T,n_{j\noteq i}}
}{}

Lo cual también tiene el nombre de \textbf{energía parcial molar de Gibbs}. Con esto se define una nueva propiedad llamada \textbf{propiedad parcial molar}, y cualquier propiedad extensiva se puede describir desde una propiedad parcial molar.
Esta propiedad se definde de la siguiente forma: Para una propiedad M cualquiera, se define una propiedad parcial molar M como:

\insertequation{
    \overline{M}=\left( \frac{\partial \underline{M}}{\partial ni} \right)_{P,T,n_{j\noteq i}}
}{}

Luego también podemos definir a partir de esta cantidad las siguientes cantidades:
\insertalign{
    \underline{M}=\sum_i n_i \overline{M}_i\\
    M = \sum_i x_i \overline{M}_i
}
Por lo cual podemos escribir la energía libre de Gibbs como:

\insertalign{
    \underline{G}=\sum_i n_i \overline{G}_i=\sum_i n_i \mu_i\\
    M = \sum_i x_i \overline{G}_i=\sum_i x_i \mu_i
}

\subsection{Criterios de Equilibrio}

Para equilibrio a T y P constantes, se debe minimizar la energía libre de Gibbs. De todas maneras, como dT y dP son cero, en un sistema cerrado, la condición de equilibrio indica que $dG=0$ en el equilibrio, para T y P constantes.

Esta ecuación la podemos utilizar para cualquier problema, quedando como:


\insertimage[]{img/imagenes/ec3}{width=10cm}{}

Lo cual se define por:
\insertalign{
    \mu_1^V=\mu_1^L//
    \mu_2^V=\mu_2^L//
}

\subsubsection{Potencial químico de un fluido puro}

Anteriormente se mostro que para un fluido puro la reacción de equilibrio se debe igualar con la energía molar de Gibbs para cada una de las fases.
\insertimage[]{img/imagenes/ec4}{width=10cm}{}

Para un fluido puro, solo hay un componente asi que $dn_i=dn$ y como $G(T,P)$ es intensiva $n(\partial G/\partial n)_{T,P}=0$. Luego:

\insertequation{
    \mu_{i,puro}=G_i
}

Con lo cual se demuestra que el potencial químico de un fluido puro es simplemente la energía molar de Gibbs. Los componentes puros pueden ser considerados un caso especial dentro del problema global de las restricciones de equilibrio. %i2
\clearpage
\subsection{Equilibrio líquido-líquido}

El equilibrio líquido-líquido se basa en el principio de la generación de dos fases cuando dos líquidos se mezclan.
Un ejemplo de esto es la mezcla del agua y el aceite, en donde se generan dos fases, una fase $\alpha$ y una fase $\beta$.

\insertimage[]{img/imagenes/ell.png}{width=3cm}{Diagrama de equilibrio líquido-líquido bifásico}.

Este equilibrio presenta las propiedades típicas de un equilibrio, \textbf{isofugacidad}. De esta forma podemos encontrar la siguiente expresión
\begin{align}
    \hat{f}_i^{\alpha}&=\hat{f}_i^{\beta}\\
    \gamma_i^{\alpha}x_i^{\alpha}P_i^{sat}&=\gamma_i^{\beta}x_i^{\beta}P_i^{sat}\\
    \gamma_i^{\alpha}x_i^{\alpha}&=\gamma_i^{\beta}x_i^{\beta}\\
\end{align}

Estas composiciones se conocen como \textbf{solubilidades mutuas}.

También se puede dar en el que coexistan tres fases, siendo dos líquidas y una gaseosa.
\insertimage[]{img/imagenes/ellv.png}{width=3cm}{Equilibrio trifásico, Líquido-líquido-vapor.}

En este tipo de equilibrio la isofugacidad toma la siguiente forma
\begin{align}
    \hat{f}_i^{\alpha}&=\hat{f}_i^{\beta}=\hat{f}_i^{V}\\
    \gamma_i^{\alpha}x_i^{\alpha}P_i^{sat}&=\gamma_i^{\beta}x_i^{\beta}P_i^{sat}=y_iP
\end{align}

\subsubsection{Estabilidad y Energía de Gibbs de exceso}

Cuando estamos hablando de equilibrio líquido-líquido hay que tener presente la 
inestabilidad del sistema, la cual puede medirse por medio de la enegía de Gibbs de exceso.
Esta energía sigue la siguiente ecuación.

\insertalign{
    G&=\Delta G_{mix}+\sum_i x_i G_i\\
    &=G^{E}+\Delta G_{mix}^{is}+\sum_i x_iG_i\\
    \Rightarrow \Delta G_{mix}&=G^{E}+\Delta G_{mix}^{is}\\
    &=G^{E}+RT\sum_i x_i \ln x_i
}

Es esta expresión del $\Delta G_{mix}$ la que nos va a ayudar a analizar el equilibrio líquido-líquido.

\begin{images}{Diagramas varios.}
    \addimage{img/imagenes/lle1.png}{width=5.5cm}{Diagrama G/RT vs composición.}
    \addimage{img/imagenes/lle2.png}{width=6cm}{Diagrama de las diferentes variables representadas en las líneas de los gráficos.}
\end{images}

Podemos notar del diagrama anterior que el $\Delta G_{mix}/RT$ presenta sectores convexos, curvas hacia abajo, las cuales van a representar los sectores en donde se encuentran los equilibrios líquido-líquido.
Ahondando más en esto, en la siguiente imagen se representa el diagrama $\Delta G_{mix}/RT$ vs composición. Podemos notar que hay nos guatitas, a partir de una de ellas se debe lanzar una línea tangencial y donde esta toque 
a la segunda guatita se indicará la composición de la fase líquida. Es imporante notar que los dos puntos que tocará esta línea tangente serán las composiciones de las dos fases líquidas en equilibrio líquido-líquido.
\insertimage[]{img/imagenes/dGmix1.png}{width=6cm}{Diagrama  $\Delta G_{mix}/RT$ vs composición.}

Otro ejemplo de lo anterior puede ser visto en la siguiente imagen. En donde se presenta la recta tangente que une los dos puntos estables, los cuales representan las composiciones de las dos fases en equilibrio. Sumado a lo anterior 
se presentan las fases estables y metaestables, se pueden encontrar al ser sectores cercanos a los puntos estables (metaestables) y sectores lejanos entre los estables (inestable.)
\insertimage[]{img/imagenes/dGmix2.png}{width=10cm}{Diagrama  $\Delta G_{mix}/RT$ vs composición.}

Estos gráficos antes presentados son calculados por medio del $\Delta G_{mix}/RT$, esta cantidad puede ser obtenida a partir de las siguiente ecuaciones:
\begin{align}
    \Delta G_{mix}&=G^{E}+RT\sum_i x_i \ln x_i\\
    \Delta G_{mix}&=RT \sum x_i \ln \gamma_i + RT\sum_i x_i \ln x_i \\
    \intertext{Por lo cual}
    \frac{\Delta G_{mix}}{RT}&=\sum_i x_i \ln \gamma_i + \sum_i x_i \ln x_i \\
    &= \sum_i x_i \ln \gamma_i x_i
\end{align}

\subsubsection{LLE usando actividades}

De manera análoga a VLE uno puede definir el coeficiente de partición en LLE como:
\begin{align}
    K_i=\frac{x_i^{\beta}}{x_i^{\alpha}}=\frac{\gamma_i^{\beta}}{\gamma_i^{\alpha}}
\end{align}

Se pueden ocupar los cálculos flash para encontrar estas condicones. Para esto es necesario un algoritmo apropiado. 
Asumiendo que $x_i^{\alpha}$ debe sumar 1 se puede llegar a la siguiente expresión
\begin{align}
    x_1^{\alpha}=\frac{1-K_2}{K1-K2} \text{ and } x_1^{\beta}=x_1^{\alpha}K_1
\end{align}
Utilizando esta ecuación podemos generar el siguiente algoritmo para encontrar las composiciones de las diferentes fases en el equilibrio
\begin{enumerate}
    \item En primer lugar se debe asumir que la fase $\alpha$ es pura en 1 y la fase $\beta$ es pura en 2. Así se calculan los coeficientes de actividad iniciales. Como son composiciones puras, son los coeficientes de actividad en dilusión infinita.
    \item Se calcula $K_{i,old}=\gamma_i^{\beta}/\gamma_i^{\alpha}$ donde $\gamma_i$ son evaluados en la composición inicial.
    \item Calcular un nuevo $ x_{1,new}^{\alpha}=(1-K_{2,old})/(K_{1,old}-K_{2,old})$ y $x_{2,new}^{\alpha}=1-x_{1,new}^{\alpha}$.
    \item Calcular $x_{1,new}^{\beta}=K_{1,old} x_{1,new}^{\alpha}$,$x_{2,new}^{\beta}=1-x_{1,new}^{\beta}$
    \item Determinar $\gamma_{i,new}$ para cada fase líquida desde los valores de $x_{i,new}$.
    \item Calcular $K_{i,new}=\gamma_i^{\beta}/\gamma_i^{\alpha}$.
    \item Actualizar $x_{i,old}$ y $K_{i,old}$.
    \item Iterar.
\end{enumerate}
\subsubsection{Diagramas de fase binarios}

Los equilibrio líquido-líquido, al igual que los equilibrios líquido-vapor, generan "campanas" en donde se generan los equilibrios, y donde fuera se encuentra en una fase miscible.
Estos diagramas tienen la siguiente forma.

\insertimage[]{img/imagenes/llediagram.png}{width=15cm}{Diagramas de fase binarios para equilibrio líquido-líquido bicomponente.}

Como podemos apreciar en estos diagramas las campanas de color blanco son sectores donde se generan dos fases, una $\alpha$ y otra $\beta$. Y estos diagramas se ven igual que las lentejas del equilibrio líquido-vapor.
Cabe mencionar que estos diagramas pueden presentar un Lower critical Solution temperatura (LCST) o Upper Critical Solution Temperature (UCST), estos puntos son la parte más arriba de la campana para el UCST y la parte más baja de la campana para el LCST, cabe mencionar que se pueden dar los dos en el mismo diagrama.

\subsubsection{Diagramas de fase ternarios}

Estos diagramas de fase son para mezclas líquido-líquido de tres componentes, estos pueden leerse de la siguiente manera.

\insertimage[]{img/imagenes/lle3.jpg}{width=15cm}{Diagrama ternario de equilibrio líquido-líquido.}

En este diagrama es importante entender que se genera una campana la cual genera un equilibrio pero las composiciones de estos presentan 3 compuestos. Luego las composiciones de estas fases en el equilibrio se ven de la misma forma que mirar cualquier punto en este diagrama.
Cabe mencionar que en este diagrama podemos ver también las fases puras del compuesto.

\insertimage[]{img/imagenes/lle4.png}{width=15cm}{Diagrama ternario de equilibrio líquido-líquido, fases puras. Fuente: Apuntes Clase 24-1 2022, autor Dr. Roberto Canales.}
\clearpage
\section{Reacciones químicas}
\subsection{Estequimetría}
Con el propósito del cálculo termodinámico, una reacción química representa un proceso en el cual los reactivos son convertidas en productos. Un ejemplo de esto es la formación del amoníaco:
\begin{equation}
    \frac{3}{2}H_2+\frac{1}{2}N_2\Leftrightarrow NH_3
\end{equation}

Los coeficientes estequimétricos que aparecen en la ecuación anterios nos aseguran que la ecuación está balanceada.
Se representan con la letra $\nu$ y se utiliza la siguiente convención: Negativo para los reactantes y Positivo para los productos. De esta manera para esta reacción tenemos:
\begin{equation}
    \nu_{H_2}=-\frac{3}{2}\quad\nu_{N_2}=-\frac{1}{2}\quad\nu_{NH_3}=+1
\end{equation}

Los coeficientes estequimétricos representan la razón por la cual el reactante y el producto participan en la reacción química. Como la razón se mantiene al multiplicar por el mismo número, estos coeficientes pueden ser representados de diferentes maneras. Es importante ser consistente con la Estequimetría que se está usando.

La suma de los coeficiente estequimétricos es:
\begin{equation}
    \nu = \sum_i \nu_i
\end{equation}

Este valor es el número total de moles cuando la reacción procede desde la izquierda a la derecha siguiendo la Estequimetría escogida. 
\subsection{Avance de la reacción}
No todos los cambios son independientes de otros, debido a que están relacionados mediante le Estequimetría. De esta menera podemos definir los siguientes cambios en las 
cantidad de moles:

\begin{align}
    |\delta n_{N_2}|=\frac{1}{2}\frac{|\delta n_{H_2}|}{\frac{3}{2}} \quad |\delta n_{NH_3}|=1\frac{|\delta n_{H_2}|}{\frac{3}{2}}
\end{align}

De esta manera podemos darnos cuenta que el cambio en los moles de $N_2$ y $NH_3$ va a depender de los cambios en los moles de $H_2$.

Este resultado puede ser visto desde el sigueinte punto de vista:
\begin{equation}
    \frac{\delta n_{H_2}}{\nu_{H_2}}= \frac{\delta n_{N_2}}{\nu_{N_2}}= \frac{\delta n_{NH_3}}{\nu_{NH_3}}
\end{equation}
El valor común de la razón $\delta n_i/\nu_i$ es llamado \textit{avance de la reacción $\zeta$}. La ecuación generalizada para el avance de la reacción es
\begin{align}
    \zeta=\frac{n_i-n_{i0}}{\nu_i}
\end{align}

Resolviendo para el número de moles de la especie $i$ después de la reacción:
\begin{align}
    n_i=n_{i0}+\zeta \nu_i
\end{align}

De esta manera el total de moles sumando todas las especies es

\begin{align}
    n=n_0 + \zeta \nu
\end{align}
Donde este $\nu$ corresponde a $\sum \nu_i$. 

Finalmente, la fracción molar de la especie $i$ después de la reacción es 
\begin{align}
    x_i=\frac{n_i}{n}=\frac{n_{i0}+\zeta \nu_i}{n_0 + \zeta \nu}
\end{align}

En la siguiente imagen se puede encontrar una manera de solucionar este tipo de problemas de una forma visual, sin tantas ecuaciones.

\insertimage[]{img/imagenes/reac1.jpg}{width=15cm}{Diagrama de resolución de problemas de avance de reacción.}

\subsection{Entalpía reacción}
\subsubsection{Entalpía estándar de reacción}

Cualquier proceso donde, desde un estado inicial, los reactantes se llevan a un estado final donde se transforman
en productos conlleva un cambio de entalpía. Este cambio se puede ejemplificar como:

\begin{align}
    \Delta H_{reacción}=H_{productos}-H_{reactantes}
\end{align}
Para realizar los cálculos se necesita de un estado de referencia. Para esto se definen los estados estándar de cada fase. De esta forma:

\begin{itemize}
    \item Para gases (g): El estado estándar es la sustancia pura en estado de gas ideal a 1 bar.
    \item Para líquidos (l): El estado estándar es la sustancia pura en fase líquida a 1 bar.
    \item Para sólidos (s): El estado estándar es la sustancia sólida a 1 bar:
    \item Para fase acuosa (aq): El estado estándar es la solución acousa a 1 bar que obedece la ley de Henry a una concentración a 1 molal.
\end{itemize}

El estado estándar específica la presión y pureza del componente. Es por esto que todas las propiedaddes del estado estándar son solo funciones de la temperatura.
Las entalías estándar de reacción están tabuladas para la reacción de formación de especies. Por convenciónn, las entalpías de formación estándar de los elementos puros es 0 (por ejemplo para $H_2$). Los valores tabulados permiten el cálculo de las entalpía de reacción de calquier reacción.
De forma que la entalpía de una reacción queda definida como:

\begin{align}
    \Delta H_{R}^\circ = \sum_i \nu_i \Delta H_{f,298,i}^\circ
\end{align}

Se dirá que la reacción es exotérmica (libera calor) cuando $\Delta H_R^\circ <0$ y en el caso contrario se dirá que es endotérmica (absorbe calor), $\Delta H_R^\circ>0$.

\subsubsection{Efecto de la temperatura en la entalpía}

Cuando la reacción ocurre en condiciones diferentes a las estándar tabuladas (25°C) se debe realizar un ajuste dado por los caminos que sigue la reacción. La reacción total viene dada por
\begin{align}
    \Delta H_R^\circ(T)=\Delta H_1+ \Delta H_{R,298}\circ+\Delta H_3
\end{align}
Donde 
\begin{align}
    \Delta H_1=\left(\sum_i \int_T^{298}\nu_iC_{P,i}^\circ dT\right)_{reactantes}\\
    \Delta H_3=\left(\sum_i \int_T^{298}\nu_iC_{P,i}^\circ dT\right)_{productos}
\end{align}

De manera que 
\begin{align}
    \Delta H_R^\circ(T)=\Delta H_{298}^\circ+\int_{298}^T\left(\sum_i \nu_iC_{P,i}^\circ \right) dT
\end{align}

Podemos hacer el siguiente arreglo
\begin{align}
    \Delta C_P^\circ=\sum_i \nu_iC_{P,i}^\circ
\end{align}
Quedando finalmente la siguiente expresión para el $\Delta H_R^\circ$.
\begin{align}
    \Delta H_R^\circ(T)=\Delta H_{298}^\circ+\int_{298}^T\left(\Delta C_P^\circ \right) dT
\end{align}

Donde por lo general el $C_P^\circ$ sigue una formula similar a la siguiente:
\begin{align}
    \frac{C_P^\circ}{RT}=c_0+c_1T+c_2T^2+c_3T^3+c_4T^4
\end{align}

\subsubsection{Balance de Energía en Reacciones}

Dada la propiedad antes definida, los balances de energía para los sistemas con reacciones quedan definidos por la siguiente expresión.
\begin{align}
    \Delta(\dot{n}H)=\zeta \Delta H_{rxn}^\circ (T_0)+\sum_{in}\int_{T_{in}}^{T_0}n_{0i}C_{Pi}dT+\sum_{out}\int_{T_0}^{T_{out}}n_i C_{Pi}dT
\end{align}

\subsection{Actividad}
Los cálculos de equilibrio químico se facilitan al ocupar el concepto de actividad introducido anteriormente. La actividad queda definida en función de la fugacidad del estado estándar de la siguiente manera:
\begin{align}
    a_i=\frac{\hat{f}_i}{f_i^\circ}
\end{align}
De esta manera, la actividad también puede relacionarse con el potencial químico mediante las siguientes ecuaciones:

\begin{align}
    \ln \frac{\hat{f}^B_i}{\hat{f}^A_i}=\frac{\mu_i^B-\mu_i^A}{RT}\\
    \mu_i=\mu_i^\circ+RT\ln a_i
\end{align}
De esta manera al conocer la actividad se puede calcular el potencial químico.

\subsubsection{Actividad en un Gas}

El estado estándar para un gas (g) es \textit{un gas puro en estado ideal a la temperatura del sistema y 1 bar}. Utilizando esta información se puede calcular la actividad:
\begin{align}
    \hat{f}_i=y_i\hat{\varphi}_iP\\
    f_i^\circ = P^\circ\\
    \intertext{de forma que la actividad es}
    a_i=y_i \hat{\varphi}_i\frac{P}{P^\circ}
\end{align}

De esta manera se puede calcular la actividad mediante el calculo del coeficiente de fugacidad mediante una EoS para la fase gaseosa.
\subsubsection{Actividad de un líquido}

A través de la fugacidad de un líquido
\begin{align}
    \hat{f}_i=\gamma_i x_i f_{i,puro}
\end{align}
De esta manera se calcula la actividad mediante
\begin{align}
    a_i=\frac{\hat{f}_i}{f_i^\circ}=\frac{\gamma_ix_if_{i,puro}}{f_i^\circ}
\end{align}

Luego, el factor de Poynting va a corresponder a la razón entre la fugacidad pura y la fugacidad estándar, por lo cual la actividad de esta fase es
\begin{align}
    a_i=\gamma_ix_i\exp\left(\frac{P-P^\circ}{RT}V_i\right)
\end{align}

Luego, mediante las asuncionciones tipicas vistas en secciones anteriores, llegamos a que 
\begin{align}
    a_i=\gamma_ix_i
\end{align}

\subsubsection{Actividad de un sólido}

Los sólidos no se mezclan con otras sustancias, incluso si son parte de una mezcla multicomponente, estos siempre permaneceran puros. En la mayoría de los casos, los sólidos se pueden catalogar como completamente inmiscibles. Bajo estas condiciones, la fugacidad del sólido es
\begin{align}
    f_i(x_i,T,P)=f_{i,puro}(T,P)
\end{align}
Ambas fugacidades están relacionadas por el factor de Poynting
\begin{align}
    f_i=f_i^\circ \exp\left( \frac{P-P^\circ}{RT}V_i \right)
\end{align}

Dejando como resultado que la actividad del sólido es 
\begin{align}
    a_i=\exp\left( \frac{P-P^\circ}{RT}V_i \right)
\end{align}

A menos que la presión sea sustancialmente mayor a 1 bar, el factor de Poynting puede ser despreciado para sólidos.

\subsection{Constante de equilibrio}

En principio, todas las reacciones son reversibles. Esto quiere decir que hay reactantes que tienen tendencia a convertirse en productos y productos que tienen tendencia a recombertirse en reactantes. En el equilibrio la 
transferencia neta cesa y la combinación del sistema se vuelve constante. A presión y temperatura costante el equilibrio es el estado donde se minimiza la energía libre de Gibbs.
Para el sistema de amoníaco:
\begin{equation}
    \frac{3}{2}H_2+\frac{1}{2}N_2\Leftrightarrow NH_3
\end{equation}
Si luego generamos una perturbación en el sistema entonces:

\begin{align}
    \Delta \underline{G}=-\frac{3\zeta}{2}\mu_{H_2}-\frac{\zeta}{2}\mu_{N_2}+\zeta \mu_{NH_3}=\left(-\frac{3}{2}\mu_{H_2}-\frac{1}{2}\mu_{N_2}+\mu_{NH_3}\right)\zeta
\end{align}

Luego este cambio en el $\Delta G$ puede ser gráficado de la siguiente manera:
\insertimage[]{img/imagenes/deltaGreac.png}{width=10cm}{Diagram de la energía libre de Gibbs a la hora de haber una reacción.}

\subsubsection{Cambio en la Energía Libre de Gibbs}

El pequeño cambio se mueve a tráves de una tangente, dejando la energía libre de Gibbs sin cambios o:

\begin{align}
    -\frac{3}{2}\mu_{H_2}-\frac{1}{2}\mu_{N_2}+\mu_{NH_3}=0
\end{align}

Para cualquier reacción se tendría entonces que:
\begin{align}
    \sum_i \nu_i \mu_i=0
\end{align}
Para los cambios en el equilibrio.

Aplicando la ecuación de potencial químico:
\begin{align}
    \sum_i \nu_i \mu_i^\circ + RT \sum_i \ln a_i^{\nu_i}=0
\end{align}

El primer término es el potencial químico estándar de la reaación a una temperatura T:

\begin{align}
    \sum_i \nu_i \mu_i^\circ = \Delta G^\circ
\end{align}

Por otro lado, la multiplicación de logaritmos posee la siguiente propiedad:
\begin{align}
    \ln a_i^{\nu_i} = \prod_i a_i^{\nu_i}
\end{align}

Donde $\prod$ es como una sumatoria, pero con multiplicación.
Luego, reescribiendo la ecuación llegamos a que

\begin{align}
    -\frac{\Delta G^\circ}{RT}=\ln\left(\prod_i a_i^{\nu_i}\right)
\end{align}

De esta forma se define la constante de equilibrio de una reacción a las condicones estándar como:
\begin{align}
    K=\exp\left( -\frac{\Delta G^\circ}{RT}\right)
\end{align}

Finalmente también esta puede ser calculada de la otra forma:
\begin{align}
    K=\prod_i a_i^{\nu_i}
\end{align}
\subsubsection{Constante de Equilibrio y Temperatura}

La constante de equilibrio a 298 K es calculada directamente de los datos tabulados de la energía libre de Gibss de formación. Una vez conocemos este valor, se puede extrapolar la información hacia otras temperaturas.

Cabe mencionar que la forma de calcular la energía libre de Gibbs de reacción es la siguiente:
\begin{align}
    \Delta G_R^\circ =\sum_i \nu_i G_{f,i}^\circ
\end{align}

Luegp, la extrapolación hacia otras temperaturas es la siguiente:
\begin{align}
    d\left(\frac{G^{tot}}{RT}\right)&=-\frac{H^{tot}}{RT^2}dT+\frac{V^{tot}}{RT}dP + \sum_i \frac{\mu_i}{RT} dn_i\\
    d\left(\frac{G^{\circ}}{RT}\right)&=-\frac{H^{\circ}}{RT^2}dT
\end{align}

Luego, utilizando el hecho de que $\frac{\Delta G^\circ}{RT}=-\ln K$ obtenemos que
\begin{align}
    \frac{d \ln K}{dT}=\frac{\Delta H^\circ}{RT^2}
\end{align}

Donde el $\Delta H^\circ$ va a depender de la temperatura a menos que se diga lo contrario.

\subsubsection{Ecuación de Van't Hoff}
Integrando la ecuación anterior podemos llegar a lo siguiente
\begin{align}
    \ln\frac{K(T)}{K(T_0)}=\int_{T_0}^{T}\frac{\Delta H^\circ}{RT^2}dT
\end{align}

Si asumimos que la entalpía de reacción no depende de la temperatura llegamos a que 
\begin{align}
    \int_{T_0}^{T}\frac{\Delta H^\circ}{RT^2}dT \approx -\frac{\Delta H^\circ }{R}\left(\frac{1}{T}-\frac{1}{T_0}\right)
\end{align}
Finalmente la constante de equibrio para una temperatura cualquiera queda definida por la siguiente ecuación:
\begin{align}
    K(T)\approx K(T_{298})\exp\left[ -\frac{\Delta H^\circ }{R}\left(\frac{1}{T}-\frac{1}{T_{298}}\right)  \right]
\end{align}

Notar que $T_0=298 K$.

\subsubsection{Constante de Equilibrio para  una Reacción Gaseosa}
Desde la realación
\begin{align}
    K=\prod_i a_i^{\nu_i}
\end{align}
LLegamos a que 
\begin{align}
    a_i=\frac{\hat{\varphi}y_i P}{P^\circ}
\end{align}
Lo que deja a la constante de equilibrio expresada como:
\begin{align}
    K=K_{\varphi}K_{y}\left(  \frac{P}{P_0}^{\nu}\right)=K_{\varphi}K_{y}\left(  P^{\nu}\right)
\end{align}
Se puede obviar $P_0$ dado que es 1 bar.

De esta ecuación los términos utilizados son
\begin{align}
    K_y=\prod_i y_i^{\nu_i}\quad K_\varphi =\prod_i \varphi_i^{\nu_i}
\end{align}

Donde por lo general se puede decir que el término relacionado con $\varphi$ es 1.

\subsubsection{Constante de Equilibrio para Líquidos y Sólidos}
Para líquidos y sólidos tenemos que 
\begin{align}
    K=\prod_i a_i^{\nu_i}
\end{align}

Donde para líquidos:
\begin{align}
    a_i\approx\gamma_ix_i
\end{align}
Y para sólidos 
\begin{align}
    a_i\approx 1
\end{align}

La manera para calcular las composiciones en el equilibrio para composiciones gaseosas es la siguiente:

\insertimage[]{img/imagenes/metodo1.jpeg}{width=12cm}{Metodología para determinar las composiciones en el equilibrio de composiciones gaseosas (perdón el desorden).}

Luego, en el caso de tener que hacerlo para más de una reacción se debe seguir la sigueinte metodología

\insertimage[]{img/imagenes/metodo2.jpeg}{width=12cm}{Metodología de resolución para más de una reacción química (perdón el desorden).}
 %examen
% FIN DEL DOCUMENTO
\end{document}