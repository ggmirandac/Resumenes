\section{Introducción}

La ingeniería de procesos estudia los procesos industriales en los que las materias primas se transforman en productos deseados
mediante procesos físicos, químicos, o biológicos.


Dentro de esto, la misión del ingeniero de procesos es desarrollar, diseñar, e implementar
tanto el rpoceso como los equipos, lo que implica elegir materias primas adecuadas, y operar las plantas con eficacia, seguridad y economía.


Los procesos químicose pueden definir como una sucesión de unidades para producir. Por lo cual, pueden separarse en operaciones unitarias.


Es aquí donde el ingeniero de procesos se encuentra en la capacidad de diseñar, mejorar y operar,  comprender los procesos de producción.

\subsection{Conceptos básicos}

Cualquier proceso puede ser descompuesto en acciones unitarias. EStas operaciones se repiten en distintos procesos,
poseen técnicas comunes y se basan en los mismo principios científicos.


En cada una de ellas se cambian las condiciones de una determinada cantidad de materia de una o más de las siguientes formas:
\begin{enumerate}
    \item Modificando su masa o composición.
    \item Modificando su energía.
\end{enumerate}
\subsubsection{Sistema}
Un sistema es una porción del universo que está bajo estudio. Este se encuentra separado de los alrededores por los límites del sistema. 
Y todo fenómeno que ocurre en un sistema se denomina \textbf{Proceso}.

Los diferentes tipos de sistemas que existen son:
\begin{enumerate}
    \item Sistema Abierto/Cerrado: Estos son sistemas que presentan una entrada o salida, cuando son abiertos, y carecen de esta cuando son cerrados.
    \item Sistema Reaccionante/No-Reaccionante: Aparace el término de generación. Cuando hay generación se dice que son reaccionantes, y cuando no hay generación, no-reaccionantes.
    \item Sistema Transiente/Estacionario: Los sistemas transientes hay un cambio en las condiciones con el tiempo, mientras que los estacionarios no. En los primeros aparece el término de acumulación.
    \item Sistemas de una unidad/multiples unidades: Los primeros se componen de una unidad (sistema) y la segunda de varios.
\end{enumerate}

A partir de estos sistemas se genera una ecuación que domina este proceso, y esta es la Ecuación General de Conservación.

\insertimage[]{img/imagenes/ecuacion_de_balance}{width=10cm}{Esquema general de un sistema.}

Este sistema se puede modelar por medio de la siguiente ecuación:

\insertequation{\text{Entrada} + \text{Generación} - \text{Consumo} - \text{Acumulación} = \text{Salida}}

Cuando no hay acumulación y no hay reacciones químicas de genración o consumo se tiene un: \textbf{Proceso Estacionario (Continuo)}. Y la ecuación de balance que lo describe es:
\insertequation{\text{Entrada} = \text{Salida}}

Si hauy reacciones químicas dentro del sistema la cantidad de generación del sistema se puede modelar de la siguiente forma:

\insertequation{ \text{Generación neta} = \text{Generación en el sistema} - \text{Consumo en el sistema}}

Y la ecuación de balance es:

\insertequation{\text{Entrada}+ \text{Generación neta}-\text{Acumulación} = \text{Salida}}

\clearpage
\section{Sistema Simple}

Este sistema es una unidad de proceso abierta, no reaccionante y estacionaria. No hay acumulación y no hay reacciones químicas de generación o comsumo.

Este sistema se puede definir por el siguiente diagrama:

\insertimage[]{img/imagenes/sist_simple}{width=10cm}{Diagrama representativo de un sistema simple.}

Y la ecuación de balance que lo describe es:

\insertequation{\text{Entrada}=\text{Salida}}{}

En este sistema se pueden definir las siguientes variables que pueden ayudarnos:

\begin{itemize}
    \item \textbf{S}: Número de compuestos (sustancias) presentes en el sistema.
    \item \textbf{Ne}: Número de corrientes de entrada al sistema. $Ne=N-Ns$
    \item \textbf{N}: Número de corrientes totales del sistema. $N=Ne+Ns$
    \item \textbf{Ns}: Número de corrientes de salida del sistema. $Ns=N-Ne$
\end{itemize}

\subsection{Definiciones importantes}

\begin{itemize}
    \item $F_{ij}$: Flujo másico del compuesto $i$ en la corriente $j$.
    \item $F_j$: Flujo másico de la corriente $j$.
    \item $w_{ij}$: Fracción másica del compuesto $i$ en la corriente $j$.
\end{itemize}

Y estas cantidades se relacionan por medio de la siguiente ecuación:

\insertequation{F_{ij}=F_j\cdot w_{ij}}

\subsection{Ecuaciones linealmente independientes}

En primera instancia tenemos las ecuaciones de balance de componentes:

\insertindexequation{\sum_{j=1}^{Ne}F_{ij}-\sum_{j=Ne+1}^{N}F{ij}=0}{i=1,2,...,S}

A su vez también la ecuación de balance general:

\insertequation{\sum_{j=1}^{Ne} F_{j}-\sum_{j=Ne+1}^{N}F{j}=0}{}

Solo S ecuaciones son linealmente independientes (LI). Si se quiere usar el balance global se debe eliminar el balance de uno de los componentes para
asegurar que no se utiliza una ecuacion de balance LD de las otras. De esta manera, de las (S+1) ecuaciones de balance que tenemos, solo S serán LI. Y tomamos solo S al momento de resolver el problema, 
porque una es redundante.

\section{Grados de Libertad}

Los grados de libertad son una erramienta que nos permiten establecer cuando el problema se puede resolver o no.
Es por esto que al momento de resolver un problema de balance, se debe tener en cuenta cuales son los grados de libertad.

Los grados de libertad se pueden calcular por medio de la siguiente ecuación.

\insertequation{GL=NV_c - NE - Rel - D - BC}

Estas variables significan lo siguiente:
\begin{itemize}
    \item $GL$: Grados de libertad.
    \item $NV_c$: Número de variables de corriente
    \item $NE$: Número de ecuaciones.
    \item $Rel$: Número de realaciones.
    \item $D$: Cantidad de datos entregados.
    \item $BC$: La base de cálculo.
\end{itemize}

¿Cómo establecer el valor de cada una de estas variables?

\begin{itemize}
    \item $NV_c$: Se puede calcular por medio de la sumatoria de la cantidad de sustancias que hay en cada corriente. Por lo cual: $NV_c=\sum_{i=1}^{N}S_i$
    \item $NE$: Se puede calcular como la cantidad de sustancias involucradas en el proceso, por lo cual: $NE=S$
    \item $Rel$: Es la cantidad de realciones que se pueden encontrar, ya sea por realciones que nacen de la naturaleza del proceso, o son dadas por el enunciado.
    \item $D$: Es la cantidad de datos dados por el enunciado.
    \item $BC$: Este valor es $1$ si no hay datos de las variables de corriente, y $0$ si hay datos. Cuando $BC=1$ se asume una cantidad para una de las corrientes para así resolver el problema.
\end{itemize}

A partir del valor de $GL$ podemos analizar si el problema se puede resolver o no. Si $GL=0$ el problema se puede resolver, si $GL>0$ no se puede resolver, y si $GL<0$ se puede resolver y existe información redundante.

\clearpage
\section{Extracción Sólido-Líquido (Lixiviación)}

Este proceso es una separación de los componentes de una mezcla, mediante la solubilidad selectiva de un compuesto en un solvente.
\insertimage[]{img/imagenes/diagrama_lixi}{scale=0.75}{Diagrama de un proceso de extracción sólido-líquido.}

Este proceso es aplicado en los siguientes procesos.

\begin{itemize}
    \item Extracción de aceites de semillas oleaginosas.
    \item Extracción de azúcar de remolacha.
    \item Extracción de raíces, hojas y tallos para productos farmacéuticos.
    \item Extracción de café para producción de café instantaneo
\end{itemize}

En este proceso el soluto (aceite) se va a extraer hasta que se alcanza el pseudo-equilibrio (la concentración no cambia en el tiempo). Pero este proceso posee la limitación de 
que en una operación real es imposible separar totalmente la solución del sólido (éste queda mojado, solución ocluida\footnote{Queda atrapada dentro del sólido.}). Aquí podemos distinguir claramente las dos corrientes de salida.
La primera es el \textbf{Refinado}, este consta del sólido, soluto (aceite) y el solvente utilizado. Y el segundo es el \textbf{Extracto}, el cual consta del soluto (aceite) y el solvente utilizado.

\insertimage[]{img/imagenes/diagrama_3_lix}{scale=0.5}{Diagrama que representa las denominaciones de las corrientes de entrada y salida del proceso de extracción sólido-líquido.}

Tomemos de ejemplo el siguiente proceso de extracción sólido-líquido.
\insertimage[]{img/imagenes/diag_4_lix}{scale=0.5}{Diagrama que representa una separación sólido-líquido del aceite de semillas, usando como solvente la acetona.}

En este tipo de procesos, donde hay un pseudo-equilibrio, se pueden definir una serie de relaciones que nos ayudan a resolver este problema. Estas realciones se conocen como relaciones de pseudo-equilibrio. Y estas se definen por las siguientes ecuaciones:

\insertequation{w_{23}=\frac{w_{24}}{w_{24}+w_{34}+w_{44}}}{}
\insertequation{w_{33}=\frac{w_{34}}{w_{24}+w_{34}+w_{44}}}{}
\insertequation{w_{43}=\frac{w_{44}}{w_{24}+w_{34}+w_{44}}}{}

Donde las sustancias son:
\begin{itemize}
    \item 2: Aceite
    \item 3: Agua
    \item 4: Solvente
\end{itemize}

Y las corrientes son:
\begin{itemize}
    \item 3: Extracto
    \item 4: Refinado
\end{itemize}

En estas relaciones nace de que en el pseudo-equilibrio, la concentración de una sustancia en el refinado es igual a la concentración en el refinado. 

La cantidad de relaciones que se agregan al sistema de ecuaciones que consta este proceso es igual a ($S_3$-1). Debido a que de las tres relaciones que se pueden obtener (en este caso), solo dos son LI.
De modo que en este tipo de procesos, la cantidad de relaciones por defecto es $Rel=S_3-1$. Donde $S_3$ es la cantidad de sustancias en el extracto.

Por otro lado, también tenemos una relación que aparece si y sólo si esta es mencionada en el enunciado del problema (o se pide cálcular), esta es la \textbf{relación de oclusión}. Esta hace referencia a cuantas partes de la solución el sólido atrapa. Esta se cálcula por medio de la siguiente ecuación:

\insertequation{\{\text{\% Oclusión}\}=\frac{\text{Sólución en refinado}}{\text{Sólido en refinado}}=\frac{F_4*(w_{w24}+w_{34}+w_{44})}{F_4*(w_{14})}
}

\clearpage
\section{Deshidratación Osmótica}

El proceso de deshidratración osmótica es imporatnte debido a que el agua es un agente de deterioro importante en los alimentos. De modo que removiendo el agua a través de un proceso de deshidratración se puede aumentar la vida útil del alimento.
Este proceso de remoción de agua de un trozo de alimento, se realiza por medio de la inmersión en una solución concentrada de un agente osmótico, que resulta de una simultanea salida de agua del producto y su impregnación con el agente osmótico (el alimento se impregna con el agente osmótico).
Los procesos continúan hasta que se alcanza el equilibrio.

Este sistema se puede representar de la siguiente manera:
\insertimage[]{img/imagenes/do_1}{width=13cm}{Diagrama representativo de una deshidratración osmótica.}

Aquí viene una denominación que es característica de este proceso. La solución osmótica concentrada es la que contiene los agentes que van a extraer el agua del alimento, mientras que la solución osmótica agotada es la solución osmótica después de deshidratar el alimento.

En proceso, al igual que en la extracción sólido-líquido, se generan relaciones de equilibrio. En este caso la composición de la fase acuosa ocluida en el tejido tratado (en base libre de sólidos insolubles) es igual a la de la fase líquida de la solución osmótica agotada.
De estos equilibrios podemos obtener las siguientes relaciones:

\insertequation{
    w_{14}=\frac{w_{13}}{1-w_{33}}=\frac{w_{13}}{w_{13}+w_{23}}
}
\insertequation{
    w_{24}=\frac{w_{23}}{1-w_{33}}=\frac{w_{23}}{w_{13}+w_{23}}
}

De estas ecuaciones solo se utiliza una, debido a que la otra es LD de la otra.

Por lo cual, el número de relaciones que se pueden considerar en este proceso es $Rel=S-2$.

\clearpage
\section{Contenido Energético de Corrientes}

\subsection{Primera ley de la termodinámica}

Esta ley estipula que la energía se conserva, no puede ni crearse ni destruirse. Para un sistema simple, podemos establecer que para un sistema simple, una ecuación única que definirá como se conserva la energía.
\insertimage[]{img/imagenes/energia_1}{width=8cm}{Diagrama que representa como se comporta la energía del sistema.}
\insertindexequation{\sum_{\text{flujos de entrada}}E_j+Q=\sum_{\text{flujos de salida}}E+W}{Ecuación de Balance de Energía, Entrada=Salida}

\subsection{Energía de corriente}

En una corriente podemos encontrar tres tipos de energía:
\begin{itemize}
    \item Energía Cinética (EC): Es la energía que posee un cuerpo o sistema en movimiento relativo al estado de reposo. $EC_j=\frac{1}{2}F_jv_j^2$
    \item Energía Potencial (EP): Es la energía debido a la posición de un sistema en cun campo potencial. En este caso, gravitacional. $EP_j=F_jgz$
    \item Energía Interna (U): Es la energía almacenada que posee un sistema debido a la energía atómica y molecular de la amteria. Esta incluye (1) energías de vibraciones de enlace, (2) energías rotacionales y (3) energía debido a fuerzas moleculares. $U_j=F_jU_j^*$
\end{itemize}
De esta forma la energía que trae un flujo es:
\insertequation{E_j=F_j\left(\frac{v_j^2}{2}+gz+U_j^*\right)}{}

En mayor profundidad la \textbf{Energía Interna} es una propiedad de estado, es decir, solo depende del estado del sistema, no del camino. Esta propiedad es extensiva, dado que depende del tamaño del sistema. Y además, depende del sistema de referencia.

\subsection{Trabajo}
El trabajo neto realizado por un sistema abierto se define por la siguiente ecuación:

\insertequation{W_{total}=W_f+W}
Donde $W$ es el trabajo externo, trabajo hecho por el fluido sobre una parte móvil dentro del sistema. Y $W_f$ es el trabajo de flujo, trabajo hecho sobre el fluido que entra más el efectuado por el fluido que sale del sistema.


\insertimage[]{img/imagenes/work_1}{width=8cm}{Diagrama que representa las corrientes de un sistema abierto.}
Dentro del $W_f$ podemos identificar dos trabajos, uno de salida y uno de entrada. En primera instancia el trabajo realizado sobre el fluido que entra al sistema es $W_{f(E)}$ y se calcula como:$W_{f(E)}=P_E\cdot V_E$. Por otro lado, el fluido que sale realiza trabajo sobre los alrededores,
y este trabajo se calcula como $W_{f(S)}=P_S\cdot V_S$. Dejando así que $W_f=W_{f}(E)+W_{f}(S)=P_S\cdot V_S-P_E\cdot V_E$.

De esta forma el trabajo total es:

\insertequation{W_{total}=W+\sum_S P_jF_jV_j^* - \sum_E P_jF_jV_j^*}{}

De esta forma el balance de energía queda expresado como:

\insertequation{\sum_E F_j \left(\frac{v_j^2}{2}+gz_j+U_j^*+P_jV_j \right)+Q=\sum_S F_j \left(\frac{v_j^2}{2}+gz_j+U_j^*+P_jV_j \right)+W}{}

Aquí podemos menospreciar los términos de la energía cinética y potencial. Y también podemos introducir un nuevo término conocido como Entalpía(H), la cual se cálcula como $H_j^*=U_j^*+P_jV_j$.
Lo cual nos deja la siguiente expresión para el balance de energía:

\insertequation{\sum_E F_jH_j^* + Q = \sum_S F_jH_j^* + W}{}

La determinación de la entalpía (H) y la energía interna (U) de una corriente solo puede realizarse mediante el cambio de esta propiedad respectp a un estado de referencia, es decir solo se pueden calcular los $\Delta U$ y $\Delta H$. 
Para realizar este cálculo es necesario especificar la temperatura, presión y fase del estado de referencia.

\subsection{Capacidad calorífica}

Cuando consideramos la entalpía específica de una fase de una sustancia pura, como una función de P y T, entonces para cambios isobáricos (dP=0) se cumple que:
\insertalign{
    dH &=\left(\frac{\partial H}{\partial T}\right)_P dT+\left(\frac{\partial H}{\partial P}\right)_T dP
    \\
    &= \frac{\partial H}{\partial T}_P dT
    \\
    &= C_P dT
    }{}
Esta nueva cantidad ($C_p$) se define como el calor específico a presión constante, el cual es:

\insertequation{\left(\frac{\partial H}{\partial T}\right)_P=C_P}

Luego el cambio de entalpía se puede calcular de la siguiente manera:
\insertequation{\Delta H = \int_{T1}^{T2} C_P(T) dT}

La cantidad del $C_P$ varia de compuesto a compuesto, y este puede ser cálculado de la siguiente manera:

\insertequation{C_P(T)=a+b\cdot T+c\cdot T^2 + d\cdot T^3+ e\cdot T^4}

Donde $a,b,c,d$ y $e$ son constantantes que están tabuladas.

Cuando se tienen cambios de temperatura menores a 50K (o cuando se indique), una aproximación adecuada de $\Delta H$ es considerar que la $C_P$ es una constante, dejando la ecuación de la entalpía de la siguiente forma:

\insertequation{\Delta H= \overline{C_P}\Delta T}{}

Donde $\overline{C_P}$ es la capacidad calorífica promedio.

\subsection{Entalpía de Cambio de fase}

El cambio de entalpía ($\Delta H$) que es necesario para el cambio de fase es el calor de vaporización. Este se compone de una componente que se refiere al cambio de temperatura, y una que hace referencia al cambio de estado en sí ($\Delta H_{lv}$).
Esta entalpía de vaporización es la siguiente:

\insertequation{\Delta H_{lv}\approx \Delta H_{lv}^° + \int_{T_0}^T (C_{pv}-C_{pl})dT}{}

Donde $H_{lv}°$ es la entalpía de vaporización estandar, es decir, a 25°C y a 1 atm. La expresión de la integral es la corrección que se debe hacer cuando no estamos en condiciones ideales.
En este caso $T_0$ hace referencia a la temperatura estándar (25°C) y $T$ es la temperatura de cambio de fase a las condiciones que se están, la cual puede ser cálculada por medio de la Ecuación de Antoine.
Estos valores están tabulados a presiones normales (1atm). Por lo cual se requiere ciertas correcciones cuando se trata de presiones diferentes a las atmosféricas. 
En este caso se debe utilizar la ecuación de Antoine:

\insertindexequation{\ln(P)=A-\frac{B}{T_e+C}}{Ecuación de Antoine}{}

Para esta ecuación el valor de la presión (P) debe estar en kPa (o lo que se mencione en la ecuación dada). Y a partir de esta ecuación podemos encontrar la temperatura de ebullición de un fluido a cualquier presión a partir de las constantes A, B y C, las cuales se encuentran tabuladas.
La ecuación utilizada para encontrar el valor de $T_e$ es:

\insertequation{T_e=\frac{B}{A-\ln(P)}-C}{}

Cuando queremos encontrar el cambio de entalpía de una sustancia, desde un punto de temperatura T1 hasta una temperatura T2, donde la temperatura de ebullición de esta sustancia ($T_e$) es tal que $T1<T_e<T2$.
De esta manera hay que considerar el cambio de fase que sufre la sustancia, por lo cual la ecuación que define esta cantidad es:

\insertequation{\Delta H= \int_{T1}^{T_e}C_{Pl}dT+\Delta H_{lv}(T_e)+\int_{T_e}^{T2}C_{Pv}dT}{}

Hay que notar que si estamos en condiciones diferentes a las estandar, la temperatura de ebullición cambia, y por ende, cambia el $\Delta H_{lv}$ también.

\section{Balances de Energía}

Como recordamos anteriormente, a partir de la primera ley de la termodinámica, nos podemos dar cuenta que la energía de un sistema no se crea ni se destruye. Por lo cual, cuando estamos analizando sistemas que tienen un cambio de temperatura se debe cumplir el siguiente balance:

\insertalign{
    \sum_{j=i}^{N_e} \sum_{i=1}^{S} F_{ij}\Delta H_{ij} (T_r) + \frac{\partial Q}{\partial T} = \sum_{j=N_e+1}^{N} \sum_{i=1}^{S} F_{ij}\Delta H_{ij} (T_r) + \frac{\partial W}{\partial T}
}

De esta ecuación, podemos darnos cuenta que hay que tomar un sistema de referencia, el cual se utiliza para poder calcular el $\Delta H$ de los compuestos que están intercambiando energía.
La energía de referncia puede ser escogida de tal forma que, si es la misma que la de la corriente, está genera que el $\Delta H=0$ a través de la misma definición de $\Delta H$.

Cuando seleccionamos nuestra temperatura de referencia, el $\Delta H$ queda como:

\insertequation{\Delta H = \int_{Tr}^{T_{syst}} C_P(T) dT}

Donde $T_r$ es la temperatura de referencia, y $T_{syst}$ es la temperatura del sistema (o la corriente). De aquí queda en evidencia la afirmación anterior que si $T_r = T_{syst}$ entonces $\Delta H=0$.
\clearpage
\subsection{Grados de libertad en un balance de energía}

El análisis de grados de libertad en un sistema con un intercambio de energía difiere del balance de grados de libertad de un sistema solamente definido por el balance de masas.
En estos sistemas hay que tener presente las variables energéticas del sistema.

Para el cálculo de los grados de libertad de un balance de energía tenemos que:

\insertalign{GL_{BC}=NV_c+NV_u-NE-Rel-D-BC}{}

Los grados de libertad se denominan $GL_{BC}$ dado que es un balance combinado, hay tanto un balance de masa como uno de energía. 
Las variables de ésta ecuación son:

\begin{itemize}
    \item $NV_c$: Número de variables de corriente. En este caso se cuentan tanto las cantidad de sustancias de cada corriente, como las temperaturas de estas corrientes. De modod que esta variable puede ser calculada por medio de la siguiente expresión. $\sum_{i=1}^{S} S_i+1$.
    \item $NV_u$: Número de variables de unidad. Este número indica algún tipo de cambio en el tiempo, como puede ser una corriente de calor que está entrando al sistema, o algún tipo de reacción. En este caso se tiene que $NV_u=0$ si el sistema es adiabático, y $NV_u=0$ en el caso contrario.
    \item $NE$: Número de ecuaciones. En este caso, dado que tenemos por un lado nuestro balance de masa, y además tenemos el balance de energía, podemos llegar a que $NE=S+1$.
\end{itemize}



El resto de valores permanece bajo la misma definición anterior, salvo que en este caso hay que considerar aquellos datos y relaciones enegéticas, es decir, que presenten componenetes como temperatura, calor, etc.

Cuando tenemos este tipo de balances, hay que realizar los dos, es decir, se debe realizar tanto el análisis de grados de libertad del balance de masa $GL_{BM}$, como el del balance combinado $GL_{BC}$.
Dependiendo de cuales son los resultados podemos sacar diferentes conclusiones.

\begin{table}[h]
    \begin{tabular}{|l|l|p{10cm}|}
    \hline
    $GL_{BM}$  & $GL_{BC}$  & Conclusión                                                                                                        \\ \hline
    1 o más    & 1 o más    & El problema no se puede resolver                                                                                  \\ \hline
    1          & 0          & El problema se puede resolver, pero primero se debe resolver el balance de energía.                               \\ \hline
    0          & 0          & El problema se puede resolver, y los balances están desacoplados.                                                 \\ \hline
    -1 o menos & 0          & El problema se puede resolver, y los balances están desacoplados. Pero el balance de masa está sobreespecificado. \\ \hline
    0          & -1 o menos & El problema se puede resolver, y los balances están desacoplados. Pero el balance de masa está sobreespecificado. \\ \hline
    -1 o menos & -1 o menos & El problema se puede resolver, y los balances están desacoplados. Pero ambos están sobreespecificados.            \\ \hline
    \end{tabular}
\end{table}
\clearpage
\subsection{Intercambiadores de Calor}

Los intercambiadores de calor tienen la finalidad de trnasferir energía calórica entre dos fluidos a distintas temperaturas.
\insertimage[]{img/imagenes/intercambiador1}{width=10cm}{Modelo de un intercambiador de calor.}
Estos son usados para:

\begin{itemize}
    \item Calefacción, refrigeración y acondicionamiento del aire.
    \item Pasterurización de leche y otros fluidos en la industria de alimentos.
    \item Obtención de vapor de agua en procesos de generación de electricidas y otras aplicaciones.
    \item Refinación del petroleo y otros procesos petroquímicos.
    \item Separación de gases a bajas temperaturas de criogenia.
\end{itemize}

Estos intercambiadores pueden ser clasificados en 3 tipos:

\begin{itemize}
    \item \textbf{Intercambiador de calor de tubos concéntricos}: Este intercambiador proporciona superficies de transferencia de calor a bajo costo siendo usado mayoritariamente cuando la superficie total requerida es pequeña.
    \item \textbf{Intercambiador de calor de tubos y carcasa}: Este es ampliamente usado en la indstria de procesos químicos y bioprocesos para transferencia de calor sensible y latente. Aporta mayores superficies de transferencia de calor que el de doble tuvo.
    \item \textbf{Intercambiador de calor de placas}: Este es un intercambiador de calor cmpacto con gran superficie de intercambio de calor por unidad de volumen, es de fácil limpieza y ampliación. Este siempre es a contraflujo.
\end{itemize}

\begin{images}[\label{imagenmultiple}]{Tipos de intercambiadores de calor.}
    \addimage[]{img/imagenes/intercambiadortubos}{width=5cm}{Intercambiador de calor de tubos concentricos.}
    \addimage{img/imagenes/intercambiadorcarca}{width=5cm}{Intercambiador de calor detubos y carcasa.}
    \addimage{img/imagenes/intercambiadorplacas}{width=5cm}{Intercambiador de de calor deplacas}
\end{images}

Estos intercambiadores pueden ser modelados por el siguiente sistema, dependiendo si los flujos que intercambian calor son a contra o co corriente.


\begin{images}[\label{imagenmultiple}]{Tipos de intercambiadores de calor.}
    \addimage[]{img/imagenes/cocorriente}{width=7.5cm}{Intercambiador de calor con flujos a cocorriente.}
    \addimage{img/imagenes/contracorriente1}{width=7.5cm}{Intercambiador de calor con flujos a contracorriente.}
\end{images}

Este tipo de procesos se basa en que hay un intercambio de calor entre los dos flujos. Por ende se tiene el siguiente balance de energía.

\insertequation{  \text{Energía flujo in 1} + \text{Energía flujo in 2} = \text{Energía flujo out 1} + \text{Energía flujo out 2}}

Dado que tenemos dos flujos separados, podemos asumir dos temperaturas de referencia, una para cada flujo del intercambiador de calor. 

Por otra parte, dado que no se mezclan los flujos, se puede hacer una trivialización del balance de masa de los flujos del intercambiador, de esta manera, se tienen que el flujo y composición de los flujos de cada compartimento del intercambiador
se mantienen constantes tanto a la entrada como a la salida del flujo. De esta forma simplemente tomamos las variables de corriente de una corriente, y asumimos que es la misma al otro lado. Así no usamos el balance de materia, de modo que los grados de libertad cambian.

\clearpage

\section{Sistemas Reaccionantes}

En los sistemas reaccionantes van a haber reacciones químicas que ocurren dentro de la unidad de procesos. En este caso va a haber un componente de generación y consumo. De esta forma el diagrama del procesos queda.

\insertimage[]{img/imagenes/Reaccion}{width=10cm}{Diagrama de un sistema reaccionante.}

En este sistema tenemos que el balance queda expresado como:

\insertalign{
    \text{Entrada} + \text{Generación Neta}&=\text{Salida} \\
    \text{Generación Neta}&=\text{Generación en el sistema}-\text{Consumo en el sistema}
}

\subsection{Velocidad de reacción}

En una reacción química vamos a tener un término que define la generación neta de esta. Esta se define como la velocidad neta de generación, esta sigue la siguiente ecuación:

\insertequation{R_i = \sigma_i r}

Donde:
\begin{itemize}
    \item $R_i$: Velocidad neta de generación del componente i.
    \item $\sigma_i$: Coeficiente estequiométrico del compuesto i.
    \item $r$: Velocidad de específica de la reacción.
\end{itemize}

Los valores de $R_i$ y $\sigma_i$ son negativos para las sustancias consumidas (reactantes) y positivos para las sustancias producidas (productos). También, se define $r$ como una variable de unidad en estos sistemas reaccionantes.

Un ejemplo de como extraer los $\sigma$ de una reacción química es:
\begin{center}
    \schemestart
    \chemfig{CH_4 + 2O_2} \arrow{->} \chemfig{2H_2O + CO_2}
    \schemestop 
\end{center}

En este caso si tomamos (1) \chemfig{CH_4}, (2) \chemfig{O_2}, (3)\chemfig{H_2O} y (4)\chemfig{CO_2}. Entonces:

\begin{itemize}
    \item $\sigma_1 = -1$
    \item $\sigma_2 = -2$
    \item $\sigma_3 = 2$
    \item $\sigma_4 = 1$
\end{itemize}

Cabe recalcar que el $\sigma$ de los reactantes es negativo, y el de los productos positivo.
\subsection{Balance de componentes}

El balance de componentes en un sistema reaccionante toma en cuanta la generación neta de los componentes. De esta menera se puede definir el balance de componentes para el compuesto i como:

\insertequation{
    \sum_{j=1}^{N_e} N_i x_{ij} + \sigma_i r - \sum_{j=N_e+1}^{N} N_j x_{ij}=0
}

Si una sustancia no participa de la reacción, entonces $\sigma_i=0$.

\subsubsection{Grados de libertad}

Los grados de libertad en estos sistemas toman en consideración la nueva variable de unidad que se agrega al sistema. De modo que los grados de libertad quedan como:

\insertequation{GL=NV_c +NV_u - NE - Rel - D - BC}

Estas variables significan lo siguiente:
\begin{itemize}
    \item $GL$: Grados de libertad.
    \item $NV_c$: Número de variables de corriente
    \item $NV_u$: Número de variables de unidad. En este caso este valor es igual al número de reacciones químicas.
    \item $NE$: Número de ecuaciones.
    \item $Rel$: Número de realaciones.
    \item $D$: Cantidad de datos entregados.
    \item $BC$: La base de cálculo.
\end{itemize}

\subsection{Relaciones en sistemas reaccionantes}

En los sistemas reaccionantes definimos diferentes relaciones para poder describir y estudiar mejor estos sistemas.
\subsubsection{Reactivo limitante}

Para definir cual es el reactivo limitante de la reacción se define la siguiente relación.

\insertequation{ 
    \frac{\sum_{j=1}^{N_e}N_{i,j}}{\sum_{j=1}^{N_e}N_{k,j}}=\frac{\sigma_i}{\sigma_k}
}

Donde $k$ es el reactivo limitante. De forma que, si esta relación se satisface para el reactivo limitante $k$ y el reactivo en exceso $i$, entonces tenemos que $k$ es el reactivo limitante. Si no se cumple y sólo hay dos reactantes, entonces el reactivo limitante es el otro compuesto.

En este caso, podemos  definir el exceso de i como:

\insertequation{
    \text{Exceso de i}=\sum_{j=1}^{N_e}N_{i,j}-\frac{\sigma_i}{\sigma_k}\sum_{j=1}^{N_e}N_{k,j}
}

\subsubsection{Porcentajes}

El \textbf{porcentaje de exceso} se define como:

\insertequation{
    \{\text{Porcentaje de exceso de 'i'}\}=100 \frac{\sum_{j=1}^{N_e}N_{i,j}-\frac{\sigma_i}{\sigma_k}\sum_{j=1}^{N_e}N_{k,j}}{\frac{\sigma_i}{\sigma_k}\sum_{j=1}^{N_e}N_{k,j}}
}

Este porcentaje describe la relación porcentual entre los  moles de exceso de un reactante 'i' y los moles requeridos para reaccionar completamente con el limitante 'k'.

Por otro lado, tenemos el \textbf{porcentaje de conversión} del compuesto 'i', este se puede definir como:

\insertalign{
    \{\text{\% conversión de 'i'}\}&=100 *\frac{\text{Lo que se convierte}}{\text{Lo que entra}}=100*\frac{(\sigma_i)r}{\sum_{j=1}^{N_e}N_{i,j}}\\
    &=100*\frac{\text{Lo que entra}-\text{Lo que sale}}{\text{Lo que entra}}=100*\frac{\sum_{j=1}^{N_e}N_{i,j}-\sum_{j=N_e+1}^{N}N_{i,j}}{\sum_{j=1}^{N_e}N_{i,j}}
}

En aquellos casos donde 'i' corresponde al reactivo limitante de la reacción, su porcentaje de conversión coincide con el grado de complementamiento de la reacción.
\subsection{Balance de energía}
\subsubsection{Reacciones de combustión}

Este tipo de reacciones se pueden definir como reacciones de oxidación rápidas a altas temperaturas. Estas reacciones poseen una gran importancia en la industría química dada la alta cantidad de calor generado para producir el vapor. Un ejemplo de una reacción de combustión es la siguiente:

\begin{center}
    \schemestart
    \chemfig{C_3H_8  + 5O_2} \arrow{->} \chemfig{4H_2O  + 3CO_2 }
    \schemestop 
\end{center}

En estas reacciones solo si la T° en algún punto del reactor excede los 500°C, la velocidad de generación de calor superará la velocidad de pérdida de calor en la zona de reacción.
El gas adyacente a esa zona se calienta por encima de los 500°C y así sucesivamente aumentando la T° del gas rápidamente hasta varios miles de grados en sólo una fracción de segundos.

Existen diferentes tipos de combustibles.
\begin{itemize}
    \item \textbf{Combustibles sólidos}: Principalmente carbón (mezcla de carbono, agua, cenizas no combustibles, hidrocarburos y azufre) y en menor medida madera y desperdicios sólidos.
    \item \textbf{Combustibles líquidos}: Principalmente hidrocarburos obtenidos por la destilación de petróleo. Existe un interés creciente en el uso de alcoholes que se obtienen por fermentación.
    \item \textbf{Combustibles gaseosos}: Principalmente gas natural (80-95\% de metano, el resto es etano, propano y butano), hidrocarburos ligeros obtenidos a partir del petróleo e hidrógeno.
\end{itemize}

\subsubsection{Calor de reacción}
 
Las reacciones químicas tienen asociado un componente energético (entálpico) que viene dado por la transformación de los reactantes a productos. Este se denomina \textbf{Calor de reacción $\Delta H_r$} el cual se expresa en unidades de energía por mol de reacción.
Esta cantidad depende de tanto la presión, temperatura y las fases en las que se encuentran los reactantes y productos.

En condiciones estándar este calor se cálcula de la siguiente manera:

\insertequation{
    \Delta H_{r}^{\circ}=\sum_{i=1}^{S}\sigma_i (\Delta H_f^{\circ})_i
}

En esta ecuación se hace presente el \textbf{calor de formación $\Delta H_f^{\circ}$}, este calor hace referencia al calor de la formación del compuesto S a partir de sus comstituyentes elementales en condiciones estandar (p°=1 atm, T°=25°C).
Este valor puede ser obtenido a partir de tablas.

Cuando no estamos en condiciones estandar el calor de reacción se cálcula mediante la siguiente ecuación:

\insertequation{
    \Delta H_r =  \Delta H_r^{\circ} + \sum_{i=1}^{S} \sigma_i [ H_i(T)-H_i° ]
}
Donde 

\insertequation[\label{eqn:cp_reac}]{
    H_i(T)-H_i°=\int_{T°}^{T} (c_p)_i dT
}

Considerar que T° y $H_i°$ son las respectivas cantidades en condiciones estándar.

Hay que tener cuidado cuando se utiliza la ecuación \ref{eqn:cp_reac} dado que el $c_p$ que se debe utilizar es aquel que corresponde a la fase a la cual se enuentra el compuesto en la temperatura $T$. Es decir, si por ejemplo el compuesto entra en una corriente a una temperatura $T_a$ y está en estado líquido, el $c_p$ es el $c_{pl}$.

\subsubsection{Ecuación de balance de energía}

El balance de energía en un sistema reaccionante puede ser modelado por la siguiente expresión:

\insertequation{
    \sum_{j=1}^{N_e} \sum_{i=1}^{S} N_{ij}\Delta H_{ij}+\frac{\partial Q}{\partial t} - r \Delta H_r=  \sum_{j=N_e+1}^{N} \sum_{i=1}^{S} N_{ij}\Delta H_{ij}
}

En este caso hay que tener presente que cuando estamos tratando con un sistema reaccionante la temperatura de referencia para calcular los $\Delta H$ es \textbf{siempre 25°C}.

\subsection{Calderas}

\insertimage[]{img/imagenes/caldera}{width=8cm}{Modelo de una caldera.}

Las calderas son utilizadas para calentar agua a temperatura superior a la ambiente y generar vapor a una presión mayor que la atmosférica.
Estas consisten de una cámara en la que se realiza la combustión usando aire. La transferencia de calor se realiza a través de una superficie como en los intercambiadores de calor.
Esta unidad de proceso puede ser diagramada por medio del siguiente diagrama de flujo.

\insertimage[]{img/imagenes/calderadiagra}{width=7.5cm}{Diagrama de flujo de una caldera.}

\subsubsection{Calor de combustión}

Este es el calor que se libera al oxidar una sustancia completamente con una cantidad estequimétrica de oxígeno atmosférico en condiciones estándar.
Cuando se genera la combustión de un combustible se genera a partir del carbono \chemfig{CO_2}, a partir del hidrógeno \chemfig{H_2O}, a partir del nitrógeno \chemfig{N_2} y a partir del azufre \chemfig{SO_2}. 
La mayor parte de estos productos están en estado gaseoso bajo condiciones estándar, \chemfig{H_2O} se encuentra en estado líquido aunque la combustión (muy exotérmica) normalmente produce vapor de agua.

A partir de esto último es que se definen dos calores de combustión.

\begin{itemize}
    \item \textbf{Calor alto de combustión}: Este calor es el que se genera cuando los compuestos alcanzan su estado estándar. Es a partir de esto que se aprovevha del calor latente del agua dado que esta se encuentra en estado líquido.
    \item \textbf{Calor bajo de combustión}: En este calor no se aprovecha el calor latente del agua, por lo cual que todos los productos de la combustión se encuentran en fase gaseosa.
\end{itemize}

\begin{images}[\label{imagenmultiple}]{Calores de combustión.}
    \addimage[]{img/imagenes/caloralto}{width=7.5cm}{Reacción del calor alto.}
    \addimage{img/imagenes/calorbajo}{width=7.5cm}{Reacción del calor bajo.}
\end{images}

Se puede generar una relación sobre como calcular el calor bajo a partir del calor alto, y viceversa. Esta es:

\insertequation[\label{eqn:calor_bajoyalto}]{
    \Delta H_{r1}°=\Delta H_{r2}°-21896 n
}

En esta ecuación $\Delta H_{r1}°$ es el calor de combustión del calor alto y $\Delta H_{r2}°$ es el calor de combustión del calor bajo, y $n$ es el factor estequiométrico del agua en la ecuación balanceada. En términos más sencillos, $n$ es el factor que está en la composición general del combustible.

\subsubsection{Temperatura de llama adiabática}

La temperatura de llama adiabática ($T_{ll}$) de un combustible corresponde a la máxima temperatura que se puede alcanzar al quemarlo, cuando tanto él como el aire se encuentran inicialmente en condiciones estándar (25°C @ 1 atm).

\insertimage{img/imagenes/llama1}{width=7.5cm}{Diagrama de la temperatura de llama adiabática.}

Luego tenemos diferentes condiciones de los sistemas en los cuales la temperatura de los flujos que salen del quemador cambian.

\begin{images}[\label{imagenmultiple}]{Casos de la temperatura de llama adiabática.}
    \addimage[]{img/imagenes/llama2}{width=7.5cm}{Combustión con conversión incompleta.}
    \addimage{img/imagenes/llama3}{width=7.5cm}{Combustión con exceso de aire.}
    \addimage{img/imagenes/llama4}{width=7.5cm}{Combustión con quemador no adiabático.}
\end{images}

\subsubsection{Composición atómica del carbón}

Si tenemos que el combustible que se utiliza en una combustión posee la siguiente composición \chemfig{C_mH_nO_oN_pS_q} el cual está compuesto por $w_C$ de carbono, $w_H$ de hidrógeno, $w_O$ de oxígeno, $w_N$ de nitrógeno, $w_S$ de azufre y $w_{Ceniza}$ de ceniza. Asumiendo
una masa molar de $100 g/mol$ de combustible libre de ceniza, entonces tenemos que:

\insertalign{
    m=\frac{100*w_C}{12(1-w_{Ceniza})}\\
    n=\frac{100*w_H}{1(1-w_{Ceniza})}\\
    o=\frac{100*w_O}{16(1-w_{Ceniza})}\\
    p=\frac{100*w_N}{14(1-w_{Ceniza})}\\
    q=\frac{100*w_S}{32(1-w_{Ceniza})}
}{}

Y la ecuación estequimétrica de la combustión queda definida por:

\insertequation{
    \chemfig{C_mH_nO_oN_pS_q}+(m+\frac{n}{4}-\frac{o}{2}+q)\chemfig{O_2} \rightarrow m \chemfig{CO_2}+ \frac{n}{2} \chemfig{H_2O}+\frac{p}{2} \chemfig{N_2}+q \chemfig{SO_2}
}

La composición atómica del carbón nos permite además predecir el calor de combustión alto del combustible usado.
Esta puede ser cálculada mediante la siguiente formula:

\insertequation{
    \Delta H_{r1}°=-PM_{carbon}*Q_{carbon}+RT°(-\frac{n}{4}+\frac{o}{2}+\frac{p}{2})
}

Donde
\insertequation{
    Q_{carbon}=34072*w_C+132210*w_H-11978(w_O+w_N)+6834*w_S-1529*w_{Cenizas}
}

Luego este valor puede ser utilizado en la ecuación \ref{eqn:calor_bajoyalto} para calcular el calor bajo de combustión $\Delta H_{r2}°$.

\subsection{Sistemas con Multiples Reacciones}

Si ocurre más de una reacción química independiente de forma simultánea, los términos de generación y consumo para cada sustancia reaccionante estarán relacionados por la estequiometría de las mismas de acuerdo a la siguiente ecuación:

\insertequation{
    R_i= \sum_{k=1}^{R} R_{ik}=\sum_{k=1}^{R}\sigma_{ik}r_k
}

De esta forma tenemos cual es la generación neta del compuesto 'i' a partir de las velocidades de reacción y coeficientes estequimétricos de cada compuesto en la respectiva reacción.

Dentro de un sistema de multiples reacciones tenemos relaciones que describen a este sistema.


\subsubsection{Selectividad}


Este es la razón molar entre las producciones de un componente deseado 'i' y uno indeseado 'j', es decir.

\insertalign{
    \text{Selectividad}_{i/j}&=\frac{\text{Producción deseado}}{\text{Producción indeseado}}\\
    &=\frac{\sum_{k=1}^{R}\sigma_{ik}r_k}{\sum_{k=1}^{R}\sigma_{jk}r_k}\\
    &=\frac{\sum_{k=1}^{N_e}N_{ik}-\sum_{k=N_e+1}^{N} N_{ik}}{\sum_{k=1}^{N_e}N_{jk}-\sum_{k=N_e+1}^{N} N_{jk}}
}{}

\subsubsection{Rendimiento fraccional}

Este valor es la razón porcentual entre la producción de un componente deseado 'i', a partir del reactante limitante 'k', y la máxima producción posible de 'i' a partir de 'k'.
\textit{Moles de producto formado por moles que se formarían, si el reactivo limitante se utilizara solo para producir el compuesto de interés.}

\insertalign{
    \text{Rendimiento fraccional}_{i/k}&=100 \frac{\sum_{j=1}^{R}\sigma_{ij}r_j}{\frac{\sigma_{ij} }{\sigma_{kj}} \left( \sum_{m=1}^{N_e} N_{ij}-\sum_{m=N_e+1}^{N} N_{ij}  \right) } \\
}{}

