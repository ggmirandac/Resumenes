
\section{Sistemas Multicomponentes}

\subsection{Introducción}

En primer lugar vale la pena introducir qué es un sistema múlticomponente. Un sistema múlticomponente es aquel que presenta más de dos sustancias diferentes, por lo cual ya no nos encontramos trabajando con compuestos puros.
Es en este caso donde tenemos sistemas compuestos por mezclas con elementos similares o completamente diferentes.

El comportamiento de estas mezclas es el componente básico en la industria donde ocurren separaciones. Lo que hace que las separaciones sean factibles es que podemos llevar la mezcla a un estado donde las distintas fases con diferentes composiciones pueden coexistir.

En primera instancia, para analizar estos sistemas hay que tener presenta la \textbf{Regla de las Fases de Gibbs}, la cuál viene dada por la siguiente ecuación.

\insertequation{GL=2-\phi+N}{}

Esta ecuación define los grados de libertad que se tienen en un problema. Donde $\phi$ es el número de fases en equilibrio, y $N$ corresponde al número de compuestos.

Por ejemplo, en una mezcla binaria (dos compuestos) existen dos grados de libertad. Es decir, a presión constante, tanto la temperatura como la composición\footnote{Este comcepto será introducido más adelante.} pueden variar. De esta manera, para resolver estos problemas
Cuando tenemos una variable constante, se requeriran de dos para resolverlo.

\subsubsection{Los diagramas de fases}

Cuando analizamos un sistema multicomponentes vamos a tener un gráfico en el cual vamos a mantener una de las variables (presión o temperatura), analizar el comportamiento P vs composición, o T vs composición. 

Este comportamiento se ve reflejado en los siguientes diagramas.

\begin{images}{Diagramas de fases}
    \addimage[]{img/imagenes/tvscomp1}{width=7.5cm}{Diagrama Temperatura v/s Composición.}
    \addimage[]{img/imagenes/pvscomp1}{width=7.5cm}{Diagrama Presión v/s Composición.}
\end{images}

La zona que podemos encontrar demarcada con blanco, en contraste con el gris del gráfico, es la zona donde coexisten las dos fases de ambos compuestos; en otras palabras,
esta zona se compone de los dos compuestos en un equilibrio líquido-vapor, esta zona se denomina \textbf{envoltura de fases} o clásicamente \textit{\textbf{la lenteja}}.

\subsubsubsection{Análisis de diagramas de fases}
\clearpage
\subsection{Separadores Flash}

Los separadores flash son frecuentemente usados en la industria para separar una corriente de vapor saturado de una de líquido saturado.

Este se puede diagramar por medio del siguiente diagrama.
\insertimage[\label{img:flash}]{img/imagenes/flash1}{width=7cm}{Diagrama de un separador flash, la corriente F hace referencia a la correinte de entrada, la corriente V a la corriente de salida de vapor saturado, y la corriente L a la corriente de salida de líquido saturado.}

\subsubsection{Cálculos Flash}
A la hora de realizar cálculos en separadores Flash tenemos que las cantidades molares de cada uno de los compuestos 
junto con un balance de masa nos permiten hacer cálculos sobre composiciones en la región de dos fases, esto se conoce como cálculos flash.

Tenemos en un principio que el número inicial de moles se denota F, los cuales se separan en L moles de líquido y V moles de vapor. Esto nos deja el siguiente balance global:

\insertequation[\label{eqn:flash1}]{F=L+V}{}

Con esto también podemos darnos cuenta que las fracciones de vapor y líquido suman uno, esto viene por la siguiente relación dada por dividir por F la ecuación \ref{eqn:flash1}.

\insertequation{1=\frac{L}{F}+\frac{V}{F}}{}

Podemos tomarlo por componentes, con lo cual llegariamos a que:

\insertequation{
    z_A F= y_a V+ x_A L
}

Lo que en palabras es "\textit{La composición global por el flujo de entrada es igual: a la composición de vapor por la correinte de vapor,mas la composición de líquido por la corriente de líquido}. 

Con esta ecuación podemos deducir una regla importante para este análisis, la \textbf{regla de la palanca}. Esta se deduce de la siguiente forma:

\insertalign{
    z_A F= y_a V+ x_A L \nonumber\\ 
    z_A=y_A\frac{V}{F}+x_A\frac{L}{F} \nonumber\\
    z_A=x_A\left( 1-\frac{V}{F} \right) + y_A \frac{V}{F}\nonumber\\
    z_A=x_A - x_A\frac{V}{F}+y_A\frac{V}{F}\nonumber\\
    \frac{z_A-x_A}{y_A-x_A}=\frac{V}{F} \nonumber\\
    \frac{V}{F}= \frac{z_A-x_A}{y_A-x_A} \label{eqn:palanca1}
}

Donde la última expresión, la ecuación \ref{eqn:palanca1} es la regla de la palanca. También esta se puede tomar por la siguiente relación:

\insertequation{ \frac{L}{F}=\frac{y_A-z_A}{y_A-x_A}  }{}

La cual se obtiene de una forma similar a la anterior.

\subsection{Equilibrio Líquido-Vapor}

Dependiendo de la información que se entrega se pueden realizar diferentes tipos de cálculos
para modelar la partición líquido-vapor. Los tipos de problemas son:

\begin{itemize}
    \item Presión de burbuja \textbf{BP}
    \item Presión de rocio \textbf{DP}
    \item Tempreratura de burbuja \textbf{BT}
    \item Tempreratura de rocio \textbf{DT}
    \item Flash isotérmico \textbf{FL}
    \item Flash adiabatico \textbf{FA}
\end{itemize}

\subsubsection{Principios de cálculo}

La mayoría de las aproximaciones que buscan resolver problemas en un equilibrio líquido-vapor (ELV) utilizan la razón entre la fracción molar del vapor con la del líquido
conocida como \textbf{Coeficiente de partición} o \textbf{K-Ratio}. El cual viene dado por la siguiente ecuación:

\insertequation{
    K_i=\frac{y_i}{x_i}
}{}

Este coeficiente se comporta de manera que nos indica si es que hay más cantida de vapor o de líquido en la mezcla para el compuesto $i$. 
Para un $K_i$ mayor nos indicará que hay más vapor en la mezcla, y para un $K_i$ menor hay más líquido.

La información sobre las propiedades físicas conocidad, combinando con el K-Ratio, nos permite resolver cada uno de los problemas anteriormente mencionados.

Los métodos usados para calcular el $K_i$ varían según el método a seguir. Además, estos varían con la composición, presión y temperatura.

\subsubsection{Estratégias para resolver problemas ELV}

Pueden notar que solo existen 6 tipos de problemas que involucran un ELV. Usualmente los problemas se resuelven relativamente rápido una vez
que se ubican dentro de la tabla. Se puede utilizar como estrategia general los siguientes puntos:

\begin{itemize}
    \item Decicidir si se conoce la composición del líquido, vapor o la global del enunciado.
    \item Identigicar si el fluido está en el punto de burbuja o rocío.
    \item Identificar si P, T o ambas son constantes. Decidir si el sistema es adiabático.
    \item Utilizando la informaicón anterior, decicir en que fila nos debemos posicionar.
\end{itemize}

\subsubsection{Diagrama xy}

Este diagrama es un diagrama que nos indica como se comporta la composición del vapor en función de la composición del líquido.
En este cuando se tiene una intersección entre la línea de la composición de y con la recta que nos indica x=y, se dice que hay un azeótropox. Este punto es aquel en el cual la composición del líquido es la misma que la del vapor.
\insertimage[]{img/imagenes/diagxy}{width=7.5cm}{Diagram de composición \textit{y} vs composición \textit{x}}
\clearpage
\subsection{Destilación}
\insertimage[]{img/imagenes/desti1}{width=10cm}{Diagrama del proceso se destilación.}

La destilación es un proceso continuo de separadores Flash lo cual nos permite purificar un compuesto de otro.
En este proceso un compuesto \textit{light} sube dado que es más volátil, mientras que otro \textit{heavy} va a bajar dado que es menor volátil. Esta separación de fases es fundamental y para esto se define la \textbf{volatilidad relativa}.

\insertequation{\alpha_{ij}=\frac{K_i}{K_j}}{}

Donde i hace referencia al más liviano, mientras que j al más pesado, quedando definida como:

\insertequation{\alpha_{LH}=\frac{K_LK}{K_HK}}{}

Para una buena separación es primordial que $\alpha_{LH}>1$.

Estos proceso de destilación se analizan por las siguientes curvas de destilación.

\begin{images}{Diagramas de Destilación.}
    \addimage[]{img/imagenes/desti2}{width=5cm}{Diagrama Temperatura-Composición de una destilación.}
    \addimage[]{img/imagenes/desti3}{width=5cm}{Diagrama Composición-Composición de una destialción.}
\end{images}

Hay que tener presente que dado que se hacen separaciones según las fases, cuando nos encontramos con un azeótropo, no se puede destilar sobre este punto. Por eso, por ejemplo, el etanol no se puede alcanzar una pureza del 100\%.

\subsection{Sistemas No Ideales}

La Ley de Raoult nos sirviría solamente en los casos que los compuestos son de similar función y estructura química. Sin embargo, en la realidad estos sistemas son escasos en la naturales.
Existen 4 casos de no-idealidad, estos son:
\begin{itemize}
    \item Desviación positiva de la Ley de Raoult: Hace referencia a cuando la curva del diagrama presión-composición esta por sobre lo estimado por la Ley de Raoult.
    \item Azeótropo de presión máxima: Es cuando se forma un azeótropo en condiciones de una Desviación positiva de la Ley de Raoult.
    \item Desviación negativa de la Ley de Raoult: Hace referencia a cuando la curva del diagrama presión-composición esta por debajo a lo estimado por la Ley de Raoult.
    \item Azeótropo de presión mínima: Es cuando se forma un azeótropo en condiciones de una Desviación negativa de la Ley de Raoult.
\end{itemize}

Siendo los dos últimos casos los más extraños. Podemos ver estos casos representados en las siguientes imágenes.

\begin{images}{Diagramas que representan la no-idealidad.}
    \addimage[]{img/imagenes/raoultdp}{width=8cm}{Diagrama que presenta una desviación positiva de la ley de Raoult, junto a un azeótropo de presión máxima.}
    \addimage[]{img/imagenes/raoultdn}{width=8cm}{Diagrama que presenta una desviación negativa de la ley de Raoult, junto a un azeótropo de presión mínima.}
\end{images}

\subsubsection{Conceptos para un equilibrio de fases generalizado}

La generalización desde los Principios de compuesto puro a multicomponente requiere que consideremos como las propiedades termodinámicas cambian con respecto a la
cantidad individual de cada componente. Para un fluido puro, las propiedades eran sencillamente una función de dos variables. Para una mezcla multicomponente, las energías
y entropía también dependen de la composición.
\insertalign{
    d\underline{U}(T,P,n1,n2,\ldots,ni)=\left(\frac{\partial \underline{U}}{\partial P}\right)_{T,n} dP+\left(\frac{\partial \underline{U}}{\partial T}\right)_{P,n}+ \sum_i \left( \frac{\partial \underline{U}}{\partial ni} \right)_{P,T,n_{j\noteq i}} dn_i\\
    d\underline{G}(T,P,n1,n2,\ldots,ni)=\left(\frac{\partial \underline{G}}{\partial P}\right)_{T,n} dP+\left(\frac{\partial \underline{G}}{\partial T}\right)_{P,n}+ \sum_i \left( \frac{\partial \underline{G}}{\partial ni} \right)_{P,T,n_{j\noteq i}} dn_i
}{}

A composición constante la mezcla debe seguir las mismas restricciones que un fluido puro. Esto se puede traducir en:
\insertalign{
    \left(\frac{\partial \underline{G}}{\partial P}\right)_{T,n}=\underline{V}\\
    \left(\frac{\partial \underline{G}}{\partial T}\right)_{P,n}=-\underline{S}
}

Con esto podemos reordenar la ecuación antes mencionada, quedando en:
\insertequation{
    d\underline{G}=\underline{V} dP-\underline{S}dT+\sum_i \left( \frac{\partial \underline{G}}{\partial ni} \right)_{P,T,n_{j\noteq i}} dn_i
}{}

La propiedad que se encuentra en la sumatoria se define como el potencial químico ($\mu_i=\left( \frac{\partial \underline{G}}{\partial ni} \right)$). Quedando finalmente la ecuación como:

\insertequation{
    d\underline{G}=\underline{V} dP-\underline{S}dT+\sum_i \mu_i dn_i
}{}

\subsubsection{Propiedades parciales molares}

Tenemos que el potencial químico queda definido como:
\insertequation{
    \mu_i=\left( \frac{\partial \underline{G}}{\partial ni} \right)_{P,T,n_{j\noteq i}} \label{eqn: muG}
}{}

Lo cual también tiene el nombre de \textbf{energía parcial molar de Gibbs}. Con esto se define una nueva propiedad llamada \textbf{propiedad parcial molar}, y cualquier propiedad extensiva se puede describir desde una propiedad parcial molar.
Esta propiedad se definde de la siguiente forma: Para una propiedad M cualquiera, se define una propiedad parcial molar M como:

\insertequation{
    \overline{M}=\left( \frac{\partial \underline{M}}{\partial ni} \right)_{P,T,n_{j\noteq i}}
}{}

Luego también podemos definir a partir de esta cantidad las siguientes cantidades:
\insertalign{
    \underline{M}=\sum_i n_i \overline{M}_i \label{eqn: parcialmolar2}\\
    M = \sum_i x_i \overline{M}_i \label{eqn: parcialmolar1}
}
Por lo cual podemos escribir la energía libre de Gibbs como:

\insertalign{
    \underline{G}=\sum_i n_i \overline{G}_i=\sum_i n_i \mu_i\\
    M = \sum_i x_i \overline{G}_i=\sum_i x_i \mu_i
}

También, las propiedades parciales molares se pueden calcular mediante las siguientes ecuaciones.

Teniendo 2 componentes en la mezcla, las propiedades parciales molares de cada uno se definen como:

\insertalign{
    \overline{M}_1&=M+(1-x_1)\left( \frac{\partial M}{\partial x1}  \right)_{P,T}\\
    \overline{M}_2&=M-x1\left( \frac{\partial M}{\partial x1}  \right)_{P,T}
}


\subsection{Criterios de Equilibrio}

Para equilibrio a T y P constantes, se debe minimizar la energía libre de Gibbs. De todas maneras, como dT y dP son cero, en un sistema cerrado, la condición de equilibrio indica que $dG=0$ en el equilibrio, para T y P constantes.

Esta ecuación la podemos utilizar para cualquier problema, quedando como:


\insertimage[]{img/imagenes/ec3}{width=10cm}{}

Lo cual se define por:
\insertalign{
    \mu_1^V=\mu_1^L \\
    \mu_2^V=\mu_2^L 
}

\subsubsection{Potencial químico de un fluido puro}

Anteriormente se mostro que para un fluido puro la reacción de equilibrio se debe igualar con la energía molar de Gibbs para cada una de las fases.
\insertimage[]{img/imagenes/ec4}{width=10cm}{}

Para un fluido puro, solo hay un componente asi que $dn_i=dn$ y como $G(T,P)$ es intensiva $n(\partial G/\partial n)_{T,P}=0$. Luego:

\insertequation{
    \mu_{i,puro}=G_i
}

Con lo cual se demuestra que el potencial químico de un fluido puro es simplemente la energía molar de Gibbs. Los componentes puros pueden ser considerados un caso especial dentro del problema global de las restricciones de equilibrio.

\subsubsection{Fugacidad de componente y equilibrio}

Para generalizar el concepto de fugacidad utilizaremos la ecuación que ocupamos para compuesto puro. A temperatura constante definimos $RT d\ln f = dG$. Lo cual puede ser generalizado a:

\insertequation{
    RT d\ln \hat{f}_i = d\mu_i
}{}

En esta ecuación $\hat{f}_i$ es la fugacdad del componente i en la mezcla y $\mu_i$ es el potencial químico del componente. La fugacidad del componente NO es una propiedad parcial molar.

Esta ecuación nos permite analizar como se comporta la fugacidad de componentes en el equilibrio. Al integrar dicha ecuación llegamos a:

\insertequation{
    \mu_i^{V}-\mu_{i, puro}=RT \ln \frac{\hat{f}_i^{V}}{f_i}
}

Y para el líquido:
\insertequation{
    \mu_i^{L}-\mu_{i, puro}=RT \ln \frac{\hat{f}_i^{L}}{f_i}
}

De esta forma llegamos a:

\insertequation{
    \mu_i^{V}-\mu_i^{L}=RT \ln \frac{\hat{f}_i^{V}}{\hat{f}_i^{L}}
}

Quedando finalmente que en el equilibrio

\insertequation{\hat{f}_i^{V}=\hat{f}_i^{L}}{}

\subsubsection{Propiedades de mezcla para gas ideal}

El potencial químico para un componente de una mezcla analizada como un gas ideal se define como

\insertequation{
    \mu_i^{ig}=G_i^{ig} +  RT \ln y_i
}{}

Utilizando las ecuaciones anteriores se puede llegar a una expresión para la fugacidad de un componente gas ideal

\insertalign{
    \mu_i^{ig}-\mu_{i,puro}^{ig}&= RT \ln \frac{\hat{f}_i^{ig}}{f_{i,puro}^{ig}}=RT \ln y_i\\
    \intertext{O lo que es equivalente a:}\\
    \hat{f}_i^{ig}&= yi f_{i,puro}^{ig} \cdot y_i
}

Como la fugacidad de un gas ideal se define como

\insertequation{
    \hat{f}_i^{ig}=yi P
}{}

Con una derivación similar, tenemos que se cumple la regla de Lewis-Randall:
\insertalign{
    \hat{f}_i^{is}/f_i &= x_i \\
    &\Rightarrow f_i^{is}=x_i f_i^{ig}
}

\subsubsection{Aproximación de la solución ideal}

En el equilibrio tenemos que

\insertequation{
    \hat{f}_i^{V}=\hat{f}_i^{L}
}{}

Luego usando la aproximación ideal para ambas fases, tenemos que:

\insertequation{y_i f_i^{V}= x_i f_i^{L}}{}

Recordando que

\insertalign{
    f_i^{V}=\varphi_i^{V} P \\
    f_i^{L}=\varphi_i^{L} P \\
}

Según el factor de Poynting

\insertequation[\label{eqn:poy1}]{f_i^{L} = \varphi^{sat}_i P^{sat}_i \exp(\frac{V^{L}_i(P-P^{sat}_i)}{RT})  }{}

Combinando en función del K-Ratio llegamos a:
\insertimage[]{img/imagenes/ec5}{width=9cm}{}

A bajas presiones y además a presiones cercanas a la de saturación tenemos lo siguiente:

\insertalign{
    POY_i=1 \\
    \varphi_i^{L}=1 \\
    \varphi_i^{V}=1
}

Quedando así la siguiente expresión

\insertequation[\label{eqn:raoult}]{y_iP=x_iP^{sat}_i }{}

Esta ecuación, la ecuación \ref{eqn:raoult} se denomina \textbf{Ley de Raoult}

\subsection{Ecuación de Raoult}
\subsubsection{Estimaciónes de presión con ley de Raoult}
Por medio de la ecuación de Raoult podemos calcular las condiciones de la primera burbuja o primera gota cuando se cálcula la presión de burbuja o de rocío.

En primera instancia, teniendo la presión de burbuja definica como:

\insertequation{
    P=x_1P_1^{sat}+x_2P_2^{sat}
}{}

Podemos calcular la fracción gaseosa como:

\insertequation{
    y_1=x_1P_1^{sat}/P \\
}{}

Mientras que para la presión de rocío, la cual se obtiene de la siguiente forma:

\insertequation{
    P=\frac{1}{
        \frac{y_1}{P_1^{sat}}+\frac{y_2}{P_2^{sat}}
    }
}{}

Y la fracción líquida es:
\insertequation{
    x_i=y_i P/P_i^{sat}
}{}
\subsubsection{Estimaciones de temperatura con ley de Raoult}

Para estos casos se debe hacer una iteración con respecto a la ecuación de Antoine, para esto se debe hacer que la presión definida anteriormente sea
en función de la temperatura, para esto se reemplaza el valor de la presión de saturación por su expresión de Antoine, la cual viene dada por:

\insertequation{P^{sat}= \exp \left(  A- \frac{B}{T+C}  \right)}{}

Aquí tener en consideración que la forma de esta ecuación varia de acuerdo a cual es la expresión de Antoine dada.

\subsubsection{Flash general}

Para un destilador Flash, tenemos que se pueden calcular las composiciones de la fracción líquida y gaseosa por medio de la siguiente ecuación:

\insertequation{
    x_i =  \frac{z_i}{1+(V/F)(K_1-1)}
}{}

Y su fracción gaseosa

\insertequation{
    y_i =  \frac{z_i K_i}{1+(V/F)(K_1-1)}
}{}

\subsubsection{Ecuación de Rachford y Rice}

La ecuación de Rachford y Rice se basa en que se puede obtener la razón $V/F$ a partir del hecho de que $\sum_i  x_i=1$ y que $\sum_i y_i=1$ en el destilador Flash.

Con esto se tiene la siguiente función objetivo $\sum_i x_i - \sum_i y_i = 0$, la cual se busca resolver. Para un sistema binario, esta ecuación se define como:

\insertequation{
    \frac{z_1(1-K_1)}{1+(V/F)(K_1-1)}+\frac{z_2(1-K_2)}{1+(V/F)(K_1-1)}
}{}

A partir de esto tenemos la siguiente tabla resumen que describe los diferentes problemas y su resolución:

\insertimage[]{img/imagenes/problems1}{width=15cm}{Tabla resumen de los diferentes problemas y como resolverlos.}
\clearpage
\subsection{Mezclas No Ideales}

Para mezclas no ideales, la ecuación de Raoult definida en \ref{eqn:raoult} no describe de forma correcta el comportamiento de la mezcla, para esto es que 
se genera la ecuación de Raoult modificada, la cual tiene en consideración los factores de coeficientes de actividad $\gamma$.

\insertimage[]{img/imagenes/raoultmodi}{width=10cm}{Comparación entre comportamiento entre lo modelado por la ecuación de Raoult y la ecuación de Raoult modificada.}

La ecuación de Raoult modificada se define como:

\insertequation{
    y_i P = x_i \gamma_i P_i^{sat}
}{}

Donde $\gamma$ es el \textbf{coeficiente de actividad}, el cual es una muestra de las desviaciones positivas o negativas de la ley de Raoult.

De esta manera, las ecuaciones para calcular la presión de burbuja y de rocío son respectivamente:

\insertalign{
    P=x_1\gamma_1P_1^{sat}+x_2\gamma_2P_2^{sat} \\
    P=\frac{1}{
        \frac{y_1}{\gamma_1 P_1^{sat}}+\frac{y_2}{ \gamma_2 P_2^{sat}}
    }
}

\subsubsection{Energía de Gibbs de Exceso}

Los coeficientes de actividad pueden ser obtenidos a partir de la ecuación de Raoult modificada por medio de la siguiente ecuación:

\insertequation{
    \gamma_i=\frac{y_i P }{x_i P_i^{sat}}
}{}

Aunque también se pueden obtener mediante el modelamiento a través de la energía de Gibbs de exceso.

\insertequation{
    \frac{G^E}{RT}=x_1 \ln \gamma_1+x_2 \ln \gamma_2
}{}

Esta cantidad, la energía de Gibbs de exceso, es una desviación del comportamiento energético de la mezcla.

Cuando se hace un modelamiento mediante estos modelos de exceso se sigue la siguiente estrategia.
\insertimage[]{img/imagenes/modelexces}{width=15cm}{En primera instancia se realiza un cálculo teórico de los valores de los coeficientes de actividad. Estos valores luego son analizados y se realiza un ajuste a modelos, con los cuales se pueden resolver los problemas planteados.}

Es gracias a estos modelos que podemos modelar sistemas que presentan azeótropos.

\subsubsection{Derivando la Ley de Raoult modificada}

El coeficiente de actividad se define como la razón entre la fugacidad de un componente con la fugacidad de la solución ideal a la misma fracción molar, en otras palabras:

\insertequation{
    \gamma_i = \frac{\hat{f}_i}{x_i f_i^{\circ}}
}

\textbf{Desarrollo de la Ley modificada}

Para la fase de vapor usamos el coeficiente de fugacidad

\insertequation{
    \hat{f}_i^{V}=y_i \hat{\varphi}_i P
}{}

Y para la fase líquida utilizamos el coeficiente de actividad

\insertequation{
    \hat{f}_i^{L}=x_i \gamma_i f_i^{L}
}{}

Usando la expresión de la fugacidad en la fase líquida:

\insertequation{
    f_i^{L} = \varphi_i^{sat}P_i^{sat} \exp(\frac{V^{L}_i(P-P^{sat}_i)}{RT})  
}{}

Combinando ambas expresiones tenemos que a partir del hecho de que $\hat{f}_i^{V}=\hat{f}_i^{L}$:

\insertalign{
    \hat{f}_i^{V}&=\hat{f}_i^{L} \\
    y_i \hat{\varphi_i} P &= x_i \gamma_i f_i^{L} \\
    y_i \hat{\varphi_i} P &= x_i \gamma_i \hat{\varphi_i} P_i^{sat} \exp(\frac{V^{L}_i(P-P^{sat}_i)}{RT}) 
}{}

Este método es denominado \textbf{gamma-phi}. Luego al escribir esta relación en función del K-Ratio, llegamos a que 
\insertimage[]{img/imagenes/gamaphi}{width=10cm}{}

Realizando las asunciones típicas que se realizan a bajas presiones ($\varphi_i^{sat}=1;\, P\approx P^{sat}_i;\, \hat{\varphi_i}=1$), con esto llegamos a:
\insertequation{
    y_iP=x_i\gamma_i P_i^{sat}
}{}

La cual es la expresión de la ecuacion de Raoult modificada.

\subsection{Propiedades de Exceso}

La desviación de una propiedad desde la solución es llamada \textbf{propiedad de exceso}, en otras palabras es como se desvia la solución real de la solución ideal. A partir de esto, el cálculo que se realiza 
para calcular estas propiedades es:

\insertequation{
    M^E=M-M^{is}
}{}

Donde M es cualquier propiedad, $M^E$ es la propiedad M de exceso, $M$ la propiedad M de la mezcla real y $M^{is}$ la propiedad M de la mezcla ideal.

Por ejemplo, el volumen de la solución ideal se define como 
\insertequation{
    V^{is}=\sum_i x_iV_i
}{}

Por lo cual, el volumen de exceso se define como
\insertequation[\label{eqn:vex}]{
    V^{E}=V-\sum_i x_iV_i
}{}

Y para calcular la propiedad parcial molar de una propiedad de exceso se realiza de la misma manera que cualquier otra propiedad.

\insertequation{
    \overline{M}^{E}_i=\left( \frac{\partial \underline{M}^E}{\partial n_i} \right)_{T,P, n_{j\noteq i}}
}{}

Estos valores pueden ser gráficados y estos nos van a dar cierta información respecto a el comportamiento de la mezcla real.

Para el caso del volumen de exceso tenemos el siguiente gráfico.

\insertimage[]{img/imagenes/vexceso}{width=10cm}{Diagrama de Volumen de exceso vs composición.}

De este gráfico podemos extraer la siguiente información:

\begin{itemize}
    \item El volumen de la mezcla es mayor que el de la mezcla ideal: Como podemos notar, el volumen de exceso es positivo para cualquier composición, por lo cual 
    a partir de la ecuación \ref{eqn:vex} podemos darnos cuenta que el volumen de la mezcla será mayor al de la solución ideal.
    \item Interacción negativa entre compuesto 1 y 2: Dado que hay una desviación positiva de la mezcla real con respecto a la ideal, podemos darnos cuenta que tienen una interacción negativa, es decir, se repelen.
    \item Volumenes de exceso parcial molar de 1 y 2: A partir de la reacta que tangente a cualquier composición (x1), podemos obtener los volumenes de exceso a tal composición mediante donde corta esta reacta a los puntos $x1=0$, el cual nos da el $\overline{V}_2^E$, y el corte de la reacta a $x1=1$, el cual nos da el $\overline{V}_1^E$.
\end{itemize}

También tenemos casos en los cuales el volumen de exceso es negativo, por lo cual el volumen de la mezcla real será menor al de la mezcla ideal.

\insertimage[]{img/imagenes/vexceso2}{width=10cm}{Diagramas de volumen de exceso vs composición, podemos ver un caso de volumen de exceso negativo.}

En este gráfico podemos ver un caso en el cual el volumen de exceso es negativo para cierta composición. Por lo cual, en estas composiciones el volumen de la mezcla es menor al de la mezcla ideal. En estos casos
se dice que hay una interacción positiva entre los compuestos, dado que estos interactuan, decrementando el volumen de la mezcla real.

\subsubsection{Entalpía de Exceso}

La entalpía de exceso se define de la misma manera que el volumen de exceso, de forma que

\insertequation{
    H^{E}=H-H^{is}=H-\sum_i x_iH_i
}{}

Cuando tenemos que $H^{E}>0$ el calor de la mezcla es endotérmico, es decir, al mezclarse se libera calor; mientras que si $H^{E}<0$ la mezcla es endotérmica, es decir, al mezclarse se absorve calor.

Esta entalpía puede verse de la siguiente forma en un gráfico Entalpía de exceso vs composición.

\insertimage[]{img/imagenes/entalpiaexceso}{width=10cm}{Gráfico de entalpía de exceso vs composición, podemos darnos cuenta que los cambios de presión son despresiables en esta propiedad.}

\subsubsection{Energía de Gibbs de Exceso}

La energía de Gibbs de exceso se puede definir de la siguiente manera

\insertequation{
    G^E=G-G^{is}
}{}

Dado que hay un componente entrópico en la energía de Gibbs, se da que $G^{is}≠ \sum_i x_iG_i$. Por lo cual, para encontrar la energía de Gibbs de exceso se realiza el siguiente procesdimiento.

\insertalign{
    G^{E}&=G-G^{is}\\
    &=\left(G-\sum_i x_iG_i\right)-\left(G^{is}-\sum_i x_iG_i\right) \\
    &= \Delta G_{mix}-\Delta G_{mix}^{is}\\
    &= \Delta G_{mix}-RT \sum_i x_i \ln x_i \label{eqn: ge1}
}

Aquí hay que tener presente que 
\insertalign{
    \Delta G_{mix}&=\left(G-\sum_i x_iG_i\right) \\
    \Delta G_{mix}^{is} &= \left(G^{is}-\sum_i x_iG_i\right) = RT \sum_i x_i \ln x_i
}

Hay que recordar la siguiente ecuación

\insertequation{
    \mu_i - \mu_i^{\circ}=RT \ln \frac{\hat{f}_i}{\hat{f}_i^{\circ}}
}

Donde esta última razón se define como \textbf{actividad}, la cual es la relación entre la fracción molar del líquido y el coeficiente de actividad. Esta sigue la siguiente ecuación:

\insertequation{
    a_i = \frac{\hat{f}_i}{\hat{f}_i^{\circ}} = x_i \gamma_i
}{}

Luego podemos desarrollar la expresión de $\Delta G_{mix}$ usando condiciones standard de T y P

\insertalign{
    \Delta G_{mix} &= G-\sum_i x_i G_i\\
    &= \sum_i x_i \left(\mu_i - G_i\right)\\
    &= RT \sum_i x_i \ln \frac{\hat{f}_i}{f_i}\\
    &= RT \sum_i x_i \ln a_i\\
    &=  RT \sum_i x_i \ln x_i \gamma_i \label{eqn: ge2}
}

Luego, sustituyendo en la ecuación \ref{eqn: ge1}, tenemos
\insertalign{
    G^{E} &= RT \sum_i x_i \ln x_i \gamma_i - RT \sum_i x_i \ln x_i \\
    G^{E}&= RT \sum_i x_i \ln \gamma_i
}

Los coeficientes de actividad están relacionadas con las derivadas de la energía de Gibbs de exceso mediante las propiedades parciales molares.
Podemos definir la energía de Gibbs de exceso de la siguiente manera, por medio de la ecuación \ref{eqn: parcialmolar1}
\insertequation{
    G^{E}= \sum_i x_i \overline{G}_i^{E}
}

Luego, mediante la ecuación \ref{eqn: muG} y \ref{eqn: ge2} tenemos que

\insertequation{
    \left( 
        \frac{\partial \underline{G}^{E}}{\partial n_i} 
    \right)_{T,P,n_{j≠i}} =  \overline{G}_i^{E} = \mu_i^{E}=RT \ln \gamma_i \label{eqn: ge3mu}
}{}

Este método va a ser útil para el análisis de los diferentes modelos de $G^{E}$.
\clearpage
\subsection{Modelos de $G^{E}$}
\subsubsection{Margules de un parámetro}
El modelo más simple para la expresión del Gibbs de exceso es el de la ecuación de Margules de un parámetro.

Este modelo viene de que el $G^{E}$ se comporta como una función cuadrática, como podemos ver en la siguiente gráfica.

\insertimage[]{img/imagenes/gexes}{width=10cm}{Gráficos de $G^{E}$ versus composición, en suma de deferentes modelos para el $G^{E}$.}

Como podemos ver, esta gráfica esboza una parábola, por lo cual el modelo de Margules de un parámetro, para un sistema binario, es el siguiente

\insertequation{
    \frac{G^{E}}{RT} = A_{12}x_1x_2
}{}

Y podemos también, a partir de este modelo, predecir el coeficiente de actividad

\insertequation{
    \ln \gamma_i = A_{12}(1-x_i)^2
}{}

\subsubsubsection{Demostración del modelo de coeficiente de actividad}

Esta demostración se realizará para el modelo de Margules de un parámetro, pero es importante entender que esta metodología 
es transversal a los diferentes modelos.

Partiendo de la ecuación del modelo de Margules de un parámetro tenemos que, a partir de la ecuación \ref{eqn: parcialmolar2}

\insertalign{
    \frac{\underline{G}^{E}}{RT} &= n\frac{G^{E}}{RT}\\
    &= n A_{12}x_1x_2\\
    &= (A_{12}n_2)\left(\frac{n_1}{n}\right)
}

Luego, mediante la ecuación \ref{eqn: ge3mu} tenemos que

\insertalign{
    \frac{1}{RT} \left( 
        \frac{\partial \underline{G}^{E}}{\partial n_i} 
        \right)_{T,P,n_2} &= \ln \gamma_i \\
        \frac{1}{RT} \left(\frac{\partial (A_{12}n_2)\left(\frac{n_1}{n}\right)}{\partial n_i} \right) &= \ln \gamma_i \\
        A_{12}n_2 \left[ \frac{1}{n}-\frac{n_1}{n^2} \right]    &= \ln \gamma_i \\
        A_{12} \frac{n_2}{n}\left[1-\frac{n_1}{n}\right]&= \ln \gamma_i \\
        A_{12} x_2 (1-x1) &= \ln \gamma_i \\
        A_{12} (1-x_1)^2 &= \ln \gamma_i \\
}

Ya teniendo el parámetro de la ecuación de Margules de un parámetro, podemos predecir los coeficientes de actividad, y a partir de esto los diferentes parámetros que definen al equilibrio líquido-vapor. 
Esto puede ser visto en el siguiente diagrama.

\insertimage{img/imagenes/model1}{width=5cm}{Protocolo de trabajo de los diferentes modelos para el $G^{E}$}

\subsubsection{Redlich-Kister}

En el diagrama antes mostrado, si bien el $G^{E}$ se comporta \textbf{como} una parábola, no es una parábola como tal. A partir de esto, Redlich-Kister desarrollaron
un modelo completamente empírico para el ajuste del $G^{E}$, esta se define por la siguiente ecuación

\insertequation{
    \frac{G^{E}}{RT} = x_1x_2 (B_{12} + C_{12}(x_1-x_2)+D_{12}(x_1-x_2)^2+ ...)
}{}

A partir de esta ecuación tambien se puede definir el modelo de Margules de dos parámetros, dado que es una simplificación de Redlich-Kister.
El modelo de Margules de dos parámetros es 

\insertequation{
    \frac{G^{E}}{RT} = x_1x_2 (A_{21}x_1+A_{12}x_2)
}{}

Y además
\insertalign{
    \ln \gamma_1 = x_2^2 \left( A_{12} + 2 (A_{21}-A_{12})x_1 \right)\\
    \ln \gamma_2 = x_1^2 \left( A_{21} + 2 (A_{12}-A_{21})x_2 \right)
}

\subsubsection{Ajustes de modelos}

Los modelos que se tienen se pueden ajustar por medio de datos experimentales, los cuales son utilizados como insumo en sistemas de ecuaciones, de los cuales hay que despejar los parámetros de los modelos para el $G^{E}$.

Uno de los datos para el ajuste más usados son los de dilución infinita.

\subsubsubsection{Ajuste a dilución infinita}

Un componente se puede catalogar en dilución infinita cuando solo una traza de este está presente. Esto es, cuando la mezcla binaria es 
cercana a la pureza de algún componente, pero no aún. Los coeficientes de actividad toman valores especiales cerca de la dilución infinita.

\insertalign{
    \lim_{x_i\rightarrow1 }\gamma_i &=1 \\
    \lim_{x_i\rightarrow0 }\gamma_i &=\gamma_i^{\infty} \\
}

Esto quiere decir que, cuando tenemos una mezcla cercana a la pureza del compuesto $i$, su coeficiente de actividad es 1. Pero cuando la pureza es cercana a 0 (en dilución infinita) su coeficiente de actividad es $\gamma_i^{\infty}$.

De modo que cuando calculamos estos límitar para, por ejemplo, el modelo de Margules de dos parámetros, llegamos a que
\insertalign{
    A_{12}=\ln \gamma_1^{\infty}\\
    A_{21}=\ln \gamma_2^{\infty}\\
}

Lo relevante de este tipo de ajuste es que el coeficiente de actividad en dilución infinita es medible.

\subsubsection{Funciones Objetivo}

Cuando estamos ajustando los modelos de $G^{E}$ a datos experimentales generamos funciones objetivo, las cuales son funciones que por medio de su minimización nos permiten saber los parámetros de los 
modelos para cualquier composición.

Las funciones objetivo que usualmente se usan son
\insertalign{
    FO_p=\sum_i \left(P_{calc}-P_{exp}\right)^2_i \\
    FO_{G^{E}}=\sum_i \left(G^{E}_{calc}-G^{E}_{exp}\right)^2_i \\
    FO_\gamma = \left[  \left( \frac{\gamma_a^{exp}-\gamma_a^{calc}}{\gamma_a^{exp}} \right)^2 +   \left( \frac{\gamma_b^{exp}-\gamma_b^{calc}}{\gamma_b^{exp}} \right)^2   \right]
}

\subsubsection{Coeficiente de Actividad Tipo Van der Waals}

La ecuación de estado de Van der Waals es un punto de partida para la modelación de los coeficientes de actividad
de forma teórica, sin la necesidad de datos empíricos. Para esto, antes hay que entender las reglas de mezclado

\subsubsubsection{Reglas de Mezclado}

Las ecuaciones que se introducen para representar las mezclas se denominan \textbf{reglas de mezclado}.

Para una ecuación de estado tipo Van der Waals, el parámetro $b$ representa el tamaño finito de una molécula:

\insertequation{
    b=\sum_i x_i b_i
}{}

Mientras que el parámetro $a$ presenta un análisis más complejo, el cual tiene que ver con la energía.

\insertequation{
    U-U^{ig}=-a\rho= -a/V
}{}

Si tomamos en cuanta las probabilidades de interacción, tendrémos una \textbf{regla de mezclado cuadrática}:

\insertalign{
    a=x_1^2 a_{11}+2x_1x_2 a_{12}+x_2^2 a_{22} = \sum_i \sum_j x_i x_j a_{ij}\\
    a_{12}=(1-k_{12})(a_{11}a_{22})^{1/2}
}{}

Desde esta métodología derivan varios modelos. Estos se distiguen por las distintas aproximaciones en términos de las ecuaciones resultantes.
Dentro de estas podemos encontrar los modelos de solución regular:
\begin{itemize}
    \item Modelo de van Laar
    \item Modelo de Scatchard-Hildebrand
\end{itemize}

Cabe mencionar que un modelo de solución regular asume moléculas pequeñas y sin mucha interacción, las cuales cumplen que para $G^{E}$
\insertequation{
    G^{E}=U^{E}+PV^{E}-TS^{E}
}{}
Los valores de $V^{E}$ y $S^{E}$ cumplen que $V^{E}=S^{E}=0$. Por lo cual $G^{E}=U^{E}$.

\subsubsection{Modelo de van Laar}

En base a un modelo de solución regular, podemos llegar a que el $U^{E}$ se puede calcular de la siguiente manera:

\insertimage[]{img/imagenes/ue1}{width=7.5cm}{}
\insertimage[]{img/imagenes/ue2}{width=12.5cm}{}

De esta manera tenemos que 
\insertalign{
    U^{E}&=\frac{x_{1} x_{2} V_{1} V_{2}}{x_{1} V_{1}+x_{2} V_{2}} Q \\
    Q &=\left(\frac{a_{11}}{V_{1}^{2}}+\frac{a_{22}}{V_{2}^{2}}-2 \frac{a_{12}}{V_{1} V_{2}}\right)
}
Donde los parámetros se definen como:

\insertalign{
    A_{12}=\frac{Q V_1}{RT} \\
    A_{21}=\frac{Q V_2}{RT} \\
    \frac{A_{12}}{A_{21}}=\frac{V_1}{V_2}
}

Por lo cual tenemos que 

\insertalign{
    \frac{G^{E}}{RT}&=\frac{U^{E}}{RT}=\frac{A_{12}A_{21}x_1x_2}{(x_1A_{12}+x_2A_{21})}\\
    &=\frac{A_{12}A_{21}n_1n_2}{(n_1A_{12}+n_2A_{21})}
}

Realizando el mismo procesdimiento que se realizó en la sección de demostración de coeficiente de actividad llegamos a que 
\insertalign{
    \ln \gamma_1&=\frac{A_{12}}{\left[1+\frac{A_{12}x_1}{A_{21}x_2}\right]^2}\\
    \ln \gamma_2&=\frac{A_{21}}{\left[
        1+\frac{A_{21}x_2}{A_{12}x_1} \right]^2}
}

Estas ecuaciones son útiles dado que se puede obtener el parámetro desde un punto del equilibrio líquido-vapor.
\subsubsection{Modelo de Scatchard-Hilderbrand}

La teoría de Scatchar-Hilderbrand asume que $k_{12}$  y por tanto $a_{12}=\sqrt{a_{11}a_{22}}$. Por lo tanto se tiene que:

\begin{align}
    U^{E}=\frac{x_{1} x_{2} V_{1} V_{2}}{x_{1} V_{1}+x_{2} V_{2}}\left(\frac{a_{11}}{V_{1}^{2}}+\frac{a_{22}}{V_{2}^{2}}-2 \sqrt{\frac{a_{11}}{V_{1}^{2}} \frac{a_{22}}{V_{2}^{2}}} \frac{x_{1} x_{2} V_{1} V_{2}}{x_{1} V_{1}+x_{2} V_{2}}\left(\frac{\sqrt{a_{11}}}{V_{1}}-\frac{\sqrt{a_{22}}}{V_{2}}\right)^{2}\right)
\end{align}

Los creadores del modelo reconocieron los parámetros como una relación entre la fracción de volumen y la energía de dispersión-atracción del compuesto puro. A este término lo llamaron \textbf{parámetro de solubilidad}. Volvieron a escribir la ecuación de la siguiente manera:

\begin{align}
    U^{E}=\phi_1\phi_2(\delta_1-\delta_2)^2 (x_1V_1+x_2V_2)
\end{align}
Se define de esta menera:
\textbf{Fracción de volumen}:
\begin{align}
    \phi_i\equiv x_iV_i / \sum_i x_i V_i
\end{align}
\textbf{Parámetro de solubulidad}:
\begin{align}
    \delta_i\equiv \sqrt{a_{ii}}/V_i
\end{align}
Desde datos experimentales llegamos a que:

\begin{align}
    \delta_i\equiv \sqrt{\frac{\Delta U_i^{vap}}{V_i}}=\sqrt{\frac{\Delta H_i^{vap}-RT}{V_i}}
\end{align}
De esta manera, los coeficientes de actividad por medio del modelo de Scatchard-Hilderbrand es:

\begin{align}
    RT\ln \gamma_1 = V_1 \phi_2^2 (\delta_1-\delta_2)\\
    RT\ln \gamma_2 = V_2 \phi_2^2 (\delta_1-\delta_2)
\end{align}

Estos parámetros se pueden ver en la siguiente tabla:

\insertimage[]{img/imagenes/stachard.png}{width=15cm}{Los parámetros de solubilidad en $(J/cm^3)^{1/2}$ y el volumen molar en $(cm^3/mol)$}

\subsubsection{Teoría de composición local}

Uno de los mayores supuestos de las mezclas de Van der Waals fue que las interacciones eran independientes unas de las otras, y por lo tanto la regla de mezclado cuadrática nos daría buenos resultados. 
Sin embargo, en mezclas donde las fuerzas de interacción son radicalmente opuestas (hay variadas interacciones) no trae buenos resultados.

Para este tipo de mezclas utilizaremos la teoría de la composición local, esta reconoce la posibilidad que composiciones locales del compuesto se desvien desde la composicion esperada.
Esta teoría se puede ver en la siguiente imagen:

\insertimage[]{img/imagenes/complocal.png}{width=15cm}{Se pueden ver los diferentes interaccioens distintas que se dan cuando hay composiciones locales diferentes.}

En esta teoría se asumen pesos relativos de estar cerca de una o de la otra molécula:
\begin{align}
    \frac{x_{21}{x_11}}=\frac{x_2}{x_1}\Omega_{21}\\
    \frac{x_{12}}{x_{22}}=\frac{x_1}{x_2}\Omega_{12}
\end{align}
Desarrollando la expresión cerca de moléculas 1:
\begin{align}
    x_{11}+x_{21}=1\\
    x_{21}=x_{11}\frac{x_{2}}{x_{1}}\Omega_{21}
\end{align}

De esta forma llegamos a que:
\begin{align}
    x_{11}=\frac{x_1}{x_1+x_2\Omega_{21}}\\
    x_{21}=\frac{x_2\Omega_{21}}{x_1+x_2\Omega_{21}}
\end{align}

De esta forma, tenemos que se generan dos sectores, y se genera la siguiente expresión

\begin{equation}
       (M-M^{ig})=x_1(M-M^{ig})^{(1)}+x_2(M-M^{ig})^{(2)}
\end{equation}

Con esto llegamos a:

\begin{align}
    U-U^{i g}&=\frac{N_{A}}{2}\left[x_{1} N_{c, 1}\left(x_{11} \varepsilon_{11}+x_{21} \varepsilon_{21}\right)+x_{2} N_{c, 2}\left(x_{12} \varepsilon_{12}+x_{22} \varepsilon_{22}\right)\right] \\
U^{E}&=\frac{N_{A}}{2}\left[\frac{x_{1} x_{2} \Omega_{21} N_{c, 1}\left(\varepsilon_{21}-\varepsilon_{11}\right)}{x_{1}+x_{2} \Omega_{21}}+\frac{x_{2} x_{1} \Omega_{12} N_{c, 2}\left(\varepsilon_{12}-\varepsilon_{22}\right)}{x_{1} \Omega_{12}+x_{2}}\right]
\end{align}

Con esto podemos llegar a diferentes modelos de $G^{E}$.

\subsubsubsection{Ecuación de Wilson}

Esta ecuación es la primera aproximación para el cálculo de los coeficientes de actividad mediante la teórica de composición local. Esta se expresa como:

\begin{align}
    \frac{G^{E}}{R T}&=-x_{1} \ln \left(x_{1}+x_{2} \Lambda_{12}\right)-x_{2} \ln \left(x_{1} \Lambda_{21}+x_{2}\right)\\
    \ln \gamma_{1}&=-\ln \left(x_{1}+x_{2} \Lambda_{12}\right)+x_{2}\left(\frac{\Lambda_{12}}{x_{1}+x_{2} \Lambda_{12}}-\frac{\Lambda_{21}}{x_{1} \Lambda_{21}+x_{2}}\right)\\
    \ln \gamma_{2}&=-\ln \left(x_{1} \Lambda_{21}+x_{2}\right)-x_{1}\left(\frac{\Lambda_{12}}{x_{1}+x_{2} \Lambda_{12}}-\frac{\Lambda_{21}}{x_{1} \Lambda_{21}+x_{2}}\right)\\
    \Lambda_{12}&=\frac{V_{2}}{V_{1}} \exp \left(\frac{-A_{12}}{R T}\right) \text{ y } \Lambda_{21}=\frac{V_{1}}{V_{2}} \exp \left(\frac{-A_{21}}{R T}\right)\\
\end{align}
Esta ecuación no describe de buena forma el equilibrio líquido-líquido dado que este requiera de la presencia de cierta inestabilidad en el sistema, la cual no puede ser expresada por esta ecuación.

\subsubsubsection{Non-Random Two Liquid - NRTL}

Este es otro modelo del cálculo de coeficientes de actividad:
\begin{align}
    \frac{G^{E}}{R T}&=x_{1} x_{2}\left[\frac{\tau_{12} G_{12}}{x_{1} G_{12}+x_{2}}+\frac{\tau_{21} G_{21}}{x_{1}+x_{2} G_{21}}\right] \\
    \ln \gamma_{1}&=x_{2}^{2}\left[\frac{\tau_{12} G_{12}}{\left(x_{1} G_{12}+x_{2}\right)^{2}}+\tau_{21}\left(\frac{G_{21}}{x_{1}+x_{2} G_{21}}\right)^{2}\right]\\
    \ln \gamma_{2}&=x_{1}^{2}\left[\frac{\tau_{21} G_{21}}{\left(x_{1}+x_{2} G_{21}\right)^{2}}+\tau_{12}\left(\frac{G_{12}}{x_{1} G_{12}+x_{2}}\right)^{2}\right]\\
    G_{i j}&=\exp \left(-\alpha_{i j} \tau_{i j}\right)\\
    \tau_{i j}&=\frac{\Delta g_{i j}}{R T}
\end{align}
Este modelo es ajustado a datos experimentales, se tiene que realizar una optimización del factor $\tau_{12},\tau_{21}$ asumiendo un $\alpha \in [0-0.47]$.

\subsubsubsection{UNIversal QUAsi Chemical model - UNIQUAC}

Este es un modelo de ajuste a los coeficientes de activadad de gran potencial.

Este modelo se compone de dos partes, una parte combinatoria y una residual.

\begin{align}
    \left(\frac{G^{E}}{R T}\right)^{C}&=\left(x_{1} \ln \frac{\Phi_{1}}{x_{1}}+x_{2} \ln \frac{\Phi_{2}}{x_{2}}\right)-5\left[q_{1} x_{1} \ln \left(\frac{\Phi_{1}}{\theta_{1}}\right)+q_{2} x_{2} \ln \left(\frac{\Phi_{2}}{\theta_{2}}\right)\right]\\
    \left(\frac{G^{E}}{R T}\right)^{R}&=-x_{1} q_{1} \ln \left(\theta_{1}+\theta_{2} \tau_{21}\right)-x_{2} q_{2} \ln \left(\theta_{1} \tau_{12}+\theta_{2}\right)
\end{align}

Y para el cálculo de los coeficientes de actividad:


\begin{align}
    \ln \gamma_{i}&=\ln \gamma_{i}^{C}+\ln \gamma_{i}^{R} \\
    \ln \gamma_{1}^{C}&=\ln \frac{\Phi_{1}}{x_{1}}+1-\frac{\Phi_{1}}{x_{1}}-5 q_{1}\left(\ln \frac{\Phi_{1}}{\theta_{1}}+1-\frac{\Phi_{1}}{\theta_{1}}\right) \\
    \ln \gamma_{1}^{R}&=q_{1}\left(1-\ln \left(\theta_{1}+\theta_{2} \tau_{21}\right)-\frac{\theta_{1}}{\theta_{1}+\theta_{2} \tau_{21}}-\frac{\theta_{2} \tau_{12}}{\theta_{1} \tau_{12}+\theta_{2}}\right) \\
    \Phi_{j} &\equiv \frac{x_{j} r_{j}}{\sum_{i} x_{i} r_{i}} \quad \theta_{j} \equiv \frac{x_{j} q_{j}}{\sum_{i} x_{i} q_{i}}\\
    r_{j}&=\sum_{k} v_{k}^{(j)} R_{k}  \quad q_{j}=\sum_{k} v_{k}^{(j)} Q_{k}
\end{align}

En este modelo tenemos los parámetros R y Q los cuales son parámetros dados, estos pueden ser encontrados en la Dormund Data Bank.

De esta forma, queda que los coeficientes de actividad para una mezcla binaria son:

\begin{align}
    &\ln \gamma_{1}=\ln \left(\Phi_{1} / x_{1}\right)+\left(1-\Phi_{1} / x_{1}\right)-5 q_{1}\left[\ln \left(\Phi_{1} / \theta_{1}\right)+\left(1-\Phi_{1} / \theta_{1}\right)\right] \\
&+q_{1}\left[1-\ln \left(\theta_{1}+\theta_{2} \tau_{21}\right)-\theta_{1} /\left(\theta_{1}+\theta_{2} \tau_{21}\right)-\theta_{2} \tau_{12} /\left(\theta_{1} \tau_{12}+\theta_{2}\right)\right] \nonumber\\
&\ln \gamma_{2}=\ln \left(\Phi_{2} / x_{2}\right)+\left(1-\Phi_{2} / x_{2}\right)-5 q_{2}\left[\ln \left(\Phi_{2} / \theta_{2}\right)+\left(1-\Phi_{2} / \theta_{2}\right)\right] \\
&+q_{2}\left[1-\ln \left(\theta_{1} \tau_{12}+\theta_{2}\right)-\theta_{1} \tau_{21} /\left(\theta_{1}+\theta_{2} \tau_{21}\right)-\theta_{2} /\left(\theta_{1} \tau_{12}+\theta_{2}\right)\right] \nonumber
\end{align}

\subsubsubsection{UNIversal Functional Activity Coefficient model - UNIFAC}

UNIFAC es una extensión de UNICUAQ sin parámetros ajustables a datos experimentales. Los datos para calcular los coeficientes de actividad mediante unifac se obtienen a partir de grandes bases de datos. Este tipo de modelos se denominan \textbf{modelos de contribución de grupos} dado que cada grupo de la molécula contribuye a calcular los R, Q, y $A_{ij}$-$A_{ji}$ de la molécula.
Este modelo se calcula de la misma forma que UNICUAQ:
\begin{align}
    \ln \gamma_k=\ln\gamma_k^{C}+\ln \gamma_k^{R}
\end{align}
De esta el termino combinatorial es el mismo que el de UNICUAQ, mientras que el término residual se calcula como:

\begin{align}
    \ln \gamma_{1}^{R}&=\frac{\mu_{1}-\mu_{1}^{o}}{R T}=\sum_{m} v_{m}^{(1)}\left[\ln \Gamma_{m}-\ln \Gamma_{m}^{(1)}\right]\\
    \ln \Gamma_{m}&=Q_{m}\left[1-\ln \sum_{i} \Theta_{i} \Psi_{i m}-\sum_{j} \frac{\Theta_{j} \Psi_{m j}}{\sum_{i} \Theta_{i} \Psi_{i j}}\right]\\
    \Theta_{j}&\equiv \frac{X_{j} Q_{j}}{\sum X_{i} Q_{i}}\\
    \Psi_{m j}&=\exp \left(\frac{-a_{m j}}{T}\right)\\
    X_{j}&=\frac{\sum_{\text {molecules } i} v_{j}^{(i)} x_{i}}{\sum_{\text {molecules } i} \sum_{\text {groups } k} v_{k}^{(i)} x_{i}}
\end{align}

En este término UNIFAC considera energías de interacción entre grupos funcionales y no entre moléculas. Las moléculas se dividen en subgrupos y se consideran interacciones
entre grupos funcionales y subgrupos de moléculas.
\insertimage[]{img/imagenes/unifac.png}{width=15cm}{Podemos ver como se segmentan los grupos de la molécula, generando los parámetros UNIFAC de la molécula, los cuales son obtenidos a partir de una base de datos como Dortmund Data Bank.}

En la siguiente tabla podemos ver también el parametro $A_{ij}$ que define la interacción entre los grupos funcionales de la molécula.
\insertimage[]{img/imagenes/unifac2.png}{width=15cm}{}


