
\section{Sistemas Multicomponentes}

\subsection{Introducción}

En primer lugar vale la pena introducir qué es un sistema múlticomponente. Un sistema múlticomponente es aquel que presenta más de dos sustancias diferentes, por lo cual ya no nos encontramos trabajando con compuestos puros.
Es en este caso donde tenemos sistemas compuestos por mezclas con elementos similares o completamente diferentes.

El comportamiento de estas mezclas es el componente básico en la industria donde ocurren separaciones. Lo que hace que las separaciones sean factibles es que podemos llevar la mezcla a un estado donde las distintas fases con diferentes composiciones pueden coexistir.

En primera instancia, para analizar estos sistemas hay que tener presenta la \textbf{Regla de las Fases de Gibbs}, la cuál viene dada por la siguiente ecuación.

\insertequation{GL=2-\phi+N}{}

Esta ecuación define los grados de libertad que se tienen en un problema. Donde $\phi$ es el número de fases en equilibrio, y $N$ corresponde al número de compuestos.

Por ejemplo, en una mezcla binaria (dos compuestos) existen dos grados de libertad. Es decir, a presión constante, tanto la temperatura como la composición\footnote{Este comcepto será introducido más adelante.} pueden variar. De esta manera, para resolver estos problemas
Cuando tenemos una variable constante, se requeriran de dos para resolverlo.

\subsubsection{Los diagramas de fases}

Cuando analizamos un sistema multicomponentes vamos a tener un gráfico en el cual vamos a mantener una de las variables (presión o temperatura), analizar el comportamiento P vs composición, o T vs composición. 

Este comportamiento se ve reflejado en los siguientes diagramas.

\begin{images}{Diagramas de fases}
    \addimage[]{img/imagenes/tvscomp1}{width=7.5cm}{Diagrama Temperatura v/s Composición.}
    \addimage[]{img/imagenes/pvscomp1}{width=7.5cm}{Diagrama Presión v/s Composición.}
\end{images}

La zona que podemos encontrar demarcada con blanco, en contraste con el gris del gráfico, es la zona donde coexisten las dos fases de ambos compuestos; en otras palabras,
esta zona se compone de los dos compuestos en un equilibrio líquido-vapor, esta zona se denomina \textbf{envoltura de fases} o clásicamente \textit{\textbf{la lenteja}}.

\subsubsubsection{Análisis de diagramas de fases}
\clearpage
\subsection{Separadores Flash}

Los separadores flash son frecuentemente usados en la industria para separar una corriente de vapor saturado de una de líquido saturado.

Este se puede diagramar por medio del siguiente diagrama.
\insertimage[\label{img:flash}]{img/imagenes/flash1}{width=7cm}{Diagrama de un separador flash, la corriente F hace referencia a la correinte de entrada, la corriente V a la corriente de salida de vapor saturado, y la corriente L a la corriente de salida de líquido saturado.}

\subsubsection{Cálculos Flash}
A la hora de realizar cálculos en separadores Flash tenemos que las cantidades molares de cada uno de los compuestos 
junto con un balance de masa nos permiten hacer cálculos sobre composiciones en la región de dos fases, esto se conoce como cálculos flash.

Tenemos en un principio que el número inicial de moles se denota F, los cuales se separan en L moles de líquido y V moles de vapor. Esto nos deja el siguiente balance global:

\insertequation[\label{eqn:flash1}]{F=L+V}{}

Con esto también podemos darnos cuenta que las fracciones de vapor y líquido suman uno, esto viene por la siguiente relación dada por dividir por F la ecuación \ref{eqn:flash1}.

\insertequation{1=\frac{L}{F}+\frac{V}{F}}{}

Podemos tomarlo por componentes, con lo cual llegariamos a que:

\insertequation{
    z_A F= y_a V+ x_A L
}

Lo que en palabras es "\textit{La composición global por el flujo de entrada es igual: a la composición de vapor por la correinte de vapor,mas la composición de líquido por la corriente de líquido}. 

Con esta ecuación podemos deducir una regla importante para este análisis, la \textbf{regla de la palanca}. Esta se deduce de la siguiente forma:

\insertalign{
    z_A F= y_a V+ x_A L \nonumber\\ 
    z_A=y_A\frac{V}{F}+x_A\frac{L}{F} \nonumber\\
    z_A=x_A\left( 1-\frac{V}{F} \right) + y_A \frac{V}{F}\nonumber\\
    z_A=x_A - x_A\frac{V}{F}+y_A\frac{V}{F}\nonumber\\
    \frac{z_A-x_A}{y_A-x_A}=\frac{V}{F} \nonumber\\
    \frac{V}{F}= \frac{z_A-x_A}{y_A-x_A} \label{eqn:palanca1}
}

Donde la última expresión, la ecuación \ref{eqn:palanca1} es la regla de la palanca. También esta se puede tomar por la siguiente relación:

\insertequation{ \frac{L}{F}=\frac{y_A-z_A}{y_A-x_A}  }{}

La cual se obtiene de una forma similar a la anterior.

\subsection{Equilibrio Líquido-Vapor}

Dependiendo de la información que se entrega se pueden realizar diferentes tipos de cálculos
para modelar la partición líquido-vapor. Los tipos de problemas son:

\begin{itemize}
    \item Presión de burbuja \textbf{BP}
    \item Presión de rocio \textbf{DP}
    \item Tempreratura de burbuja \textbf{BT}
    \item Tempreratura de rocio \textbf{DT}
    \item Flash isotérmico \textbf{FL}
    \item Flash adiabatico \textbf{FA}
\end{itemize}

\subsubsection{Principios de cálculo}

La mayoría de las aproximaciones que buscan resolver problemas en un equilibrio líquido-vapor (ELV) utilizan la razón entre la fracción molar del vapor con la del líquido
conocida como \textbf{Coeficiente de partición} o \textbf{K-Ratio}. El cual viene dado por la siguiente ecuación:

\insertequation{
    K_i=\frac{y_i}{x_i}
}{}

Este coeficiente se comporta de manera que nos indica si es que hay más cantida de vapor o de líquido en la mezcla para el compuesto $i$. 
Para un $K_i$ mayor nos indicará que hay más vapor en la mezcla, y para un $K_i$ menor hay más líquido.

La información sobre las propiedades físicas conocidad, combinando con el K-Ratio, nos permite resolver cada uno de los problemas anteriormente mencionados.

Los métodos usados para calcular el $K_i$ varían según el método a seguir. Además, estos varían con la composición, presión y temperatura.

\subsubsection{Estratégias para resolver problemas ELV}

Pueden notar que solo existen 6 tipos de problemas que involucran un ELV. Usualmente los problemas se resuelven relativamente rápido una vez
que se ubican dentro de la tabla. Se puede utilizar como estrategia general los siguientes puntos:

\begin{itemize}
    \item Decicidir si se conoce la composición del líquido, vapor o la global del enunciado.
    \item Identigicar si el fluido está en el punto de burbuja o rocío.
    \item Identificar si P, T o ambas son constantes. Decidir si el sistema es adiabático.
    \item Utilizando la informaicón anterior, decicir en que fila nos debemos posicionar.
\end{itemize}

\subsubsection{Diagrama xy}

Este diagrama es un diagrama que nos indica como se comporta la composición del vapor en función de la composición del líquido.
En este cuando se tiene una intersección entre la línea de la composición de y con la recta que nos indica x=y, se dice que hay un azeótropox. Este punto es aquel en el cual la composición del líquido es la misma que la del vapor.
\insertimage[]{img/imagenes/diagxy}{width=7.5cm}{Diagram de composición \textit{y} vs composición \textit{x}}
\clearpage
\subsection{Destilación}
\insertimage[]{img/imagenes/desti1}{width=10cm}{Diagrama del proceso se destilación.}

La destilación es un proceso continuo de separadores Flash lo cual nos permite purificar un compuesto de otro.
En este proceso un compuesto \textit{light} sube dado que es más volátil, mientras que otro \textit{heavy} va a bajar dado que es menor volátil. Esta separación de fases es fundamental y para esto se define la \textbf{volatilidad relativa}.

\insertequation{\alpha_{ij}=\frac{K_i}{K_j}}{}

Donde i hace referencia al más liviano, mientras que j al más pesado, quedando definida como:

\insertequation{\alpha_{LH}=\frac{K_LK}{K_HK}}{}

Para una buena separación es primordial que $\alpha_{LH}>1$.

Estos proceso de destilación se analizan por las siguientes curvas de destilación.

\begin{images}{Diagramas de Destilación.}
    \addimage[]{img/imagenes/desti2}{width=5cm}{Diagrama Temperatura-Composición de una destilación.}
    \addimage[]{img/imagenes/desti3}{width=5cm}{Diagrama Composición-Composición de una destialción.}
\end{images}

Hay que tener presente que dado que se hacen separaciones según las fases, cuando nos encontramos con un azeótropo, no se puede destilar sobre este punto. Por eso, por ejemplo, el etanol no se puede alcanzar una pureza del 100\%.

\subsection{Sistemas No Ideales}

La Ley de Raoult nos sirviría solamente en los casos que los compuestos son de similar función y estructura química. Sin embargo, en la realidad estos sistemas son escasos en la naturales.
Existen 4 casos de no-idealidad, estos son:
\begin{itemize}
    \item Desviación positiva de la Ley de Raoult: Hace referencia a cuando la curva del diagrama presión-composición esta por sobre lo estimado por la Ley de Raoult.
    \item Azeótropo de presión máxima: Es cuando se forma un azeótropo en condiciones de una Desviación positiva de la Ley de Raoult.
    \item Desviación negativa de la Ley de Raoult: Hace referencia a cuando la curva del diagrama presión-composición esta por debajo a lo estimado por la Ley de Raoult.
    \item Azeótropo de presión mínima: Es cuando se forma un azeótropo en condiciones de una Desviación negativa de la Ley de Raoult.
\end{itemize}

Siendo los dos últimos casos los más extraños. Podemos ver estos casos representados en las siguientes imágenes.

\begin{images}{Diagramas que representan la no-idealidad.}
    \addimage[]{img/imagenes/raoultdp}{width=8cm}{Diagrama que presenta una desviación positiva de la ley de Raoult, junto a un azeótropo de presión máxima.}
    \addimage[]{img/imagenes/raoultdn}{width=8cm}{Diagrama que presenta una desviación negativa de la ley de Raoult, junto a un azeótropo de presión mínima.}
\end{images}

\subsubsection{Conceptos para un equilibrio de fases generalizado}

La generalización desde los Principios de compuesto puro a multicomponente requiere que consideremos como las propiedades termodinámicas cambian con respecto a la
cantidad individual de cada componente. Para un fluido puro, las propiedades eran sencillamente una función de dos variables. Para una mezcla multicomponente, las energías
y entropía también dependen de la composición.
\insertalign{
    d\underline{U}(T,P,n1,n2,\ldots,ni)=\left(\frac{\partial \underline{U}}{\partial P}\right)_{T,n} dP+\left(\frac{\partial \underline{U}}{\partial T}\right)_{P,n}+ \sum_i \left( \frac{\partial \underline{U}}{\partial ni} \right)_{P,T,n_{j\noteq i}} dn_i\\
    d\underline{G}(T,P,n1,n2,\ldots,ni)=\left(\frac{\partial \underline{G}}{\partial P}\right)_{T,n} dP+\left(\frac{\partial \underline{G}}{\partial T}\right)_{P,n}+ \sum_i \left( \frac{\partial \underline{G}}{\partial ni} \right)_{P,T,n_{j\noteq i}} dn_i
}{}

A composición constante la mezcla debe seguir las mismas restricciones que un fluido puro. Esto se puede traducir en:
\insertalign{
    \left(\frac{\partial \underline{G}}{\partial P}\right)_{T,n}=\underline{V}\\
    \left(\frac{\partial \underline{G}}{\partial T}\right)_{P,n}=-\underline{S}
}

Con esto podemos reordenar la ecuación antes mencionada, quedando en:
\insertequation{
    d\underline{G}=\underline{V} dP-\underline{S}dT+\sum_i \left( \frac{\partial \underline{G}}{\partial ni} \right)_{P,T,n_{j\noteq i}} dn_i
}{}

La propiedad que se encuentra en la sumatoria se define como el potencial químico ($\mu_i=\left( \frac{\partial \underline{G}}{\partial ni} \right)$). Quedando finalmente la ecuación como:

\insertequation{
    d\underline{G}=\underline{V} dP-\underline{S}dT+\sum_i \mu_i dn_i
}{}

\subsubsection{Propiedades parciales molares}

Tenemos que el potencial químico queda definido como:
\insertequation{
    \mu_i=\left( \frac{\partial \underline{G}}{\partial ni} \right)_{P,T,n_{j\noteq i}}
}{}

Lo cual también tiene el nombre de \textbf{energía parcial molar de Gibbs}. Con esto se define una nueva propiedad llamada \textbf{propiedad parcial molar}, y cualquier propiedad extensiva se puede describir desde una propiedad parcial molar.
Esta propiedad se definde de la siguiente forma: Para una propiedad M cualquiera, se define una propiedad parcial molar M como:

\insertequation{
    \overline{M}=\left( \frac{\partial \underline{M}}{\partial ni} \right)_{P,T,n_{j\noteq i}}
}{}

Luego también podemos definir a partir de esta cantidad las siguientes cantidades:
\insertalign{
    \underline{M}=\sum_i n_i \overline{M}_i\\
    M = \sum_i x_i \overline{M}_i
}
Por lo cual podemos escribir la energía libre de Gibbs como:

\insertalign{
    \underline{G}=\sum_i n_i \overline{G}_i=\sum_i n_i \mu_i\\
    M = \sum_i x_i \overline{G}_i=\sum_i x_i \mu_i
}

\subsection{Criterios de Equilibrio}

Para equilibrio a T y P constantes, se debe minimizar la energía libre de Gibbs. De todas maneras, como dT y dP son cero, en un sistema cerrado, la condición de equilibrio indica que $dG=0$ en el equilibrio, para T y P constantes.

Esta ecuación la podemos utilizar para cualquier problema, quedando como:


\insertimage[]{img/imagenes/ec3}{width=10cm}{}

Lo cual se define por:
\insertalign{
    \mu_1^V=\mu_1^L//
    \mu_2^V=\mu_2^L//
}

\subsubsection{Potencial químico de un fluido puro}

Anteriormente se mostro que para un fluido puro la reacción de equilibrio se debe igualar con la energía molar de Gibbs para cada una de las fases.
\insertimage[]{img/imagenes/ec4}{width=10cm}{}

Para un fluido puro, solo hay un componente asi que dn_i=dn y como $G(T,P)$ es intensiva $n(\partial G/\partial n)_{T,P}=0$. Luego:

\insertequation{
    \mu_{i,puro}=G_i
}

Con lo cual se demuestra que el potencial químico de un fluido puro es simplemente la energía molar de Gibbs. Los componentes puros pueden ser considerados un caso especial dentro del problema global de las restricciones de equilibrio.